\documentclass{article}

\usepackage{amsmath}
\usepackage{amssymb}
\usepackage[
  left=1cm,
  right=1cm,
  top=2cm,
  bottom=2cm,
]{geometry}

\begin{document}
    \section*{Woche 1}
    \subsection*{1. Rechenoperationen}
    Stellen Sie in der Form a + ib dar:
    \begin{equation*}
        \frac{1+i}{1-i}
    \end{equation*}
    \begin{equation*}
        \frac{1}{3i}\left(6-5i+\frac{1+5i}{1+i}\right)
    \end{equation*}
    \begin{equation*}
        \left(1-i\right)^{14}
    \end{equation*}
    \\
    a) \begin{equation*}
        \frac{1+i}{1-i} = \frac{\left(1+i\right)\left(1+i\right)}{\left(1-i\right)\left(1+i\right)}=\frac{1+2i+i^2}{1-i^2}=\frac{1+2i-1}{1-\left(-1\right)}=\frac{2i}{2}=i
    \end{equation*}
    b) \begin{equation*}
        \frac{1}{3i}\left(6-5i+\frac{1+5i}{1+i}\right)=\frac{1}{3i}\left(9-3i\right)=-1-3i
    \end{equation*}
    c) \begin{equation*}
        \left(1-i\right)^{14}=\left(1-i\right)^2\left(1-i\right)^2\left(1-i\right)^2\left(1-i\right)^2\left(1-i\right)^2\left(1-i\right)^2\left(1-i\right)^2
    \end{equation*}
    \begin{equation*}
        =2i\cdot 2i\cdot 2i\cdot 2i\cdot 2i\cdot 2i\cdot 2i=128i^7=128i^2\cdot i^2\cdot i^2\cdot i
    \end{equation*}
    \begin{equation*}
        -128\cdot i^2\cdot i^2\cdot i=128\cdot i^2\cdot i=-128i
    \end{equation*}

    \subsection*{2. Eulersche Formel}
    Stellen Sie in Polarkoordinaten $z = re^{i\varphi}$ dar:
    a) $1-i$ \begin{equation*}
        r=\sqrt{1^2+\left(-1\right)^2}=\sqrt{2}; \varphi=\arctan(\frac{1}{1})=45^{\circ}=\frac{\pi}{4}
    \end{equation*}
    \begin{equation*}
        \sqrt{2}e^{\frac{\pi i}{4}}
    \end{equation*}
    b) $-\sqrt{3}+3i$\begin{equation*}
        r=\sqrt{\left(-\sqrt{3}\right)^2+3^2}=\sqrt{12}; \varphi=\arctan(-\frac{3}{\sqrt{3}})=-\frac{\pi}{3}
    \end{equation*}
    \begin{equation*}
        \sqrt{12}e^{\frac{\pi i}{3}}
    \end{equation*}
    c) $\sqrt{2}i$\begin{equation*}
        r=\sqrt{2};\varphi=90^\circ=\frac{\pi}{2}
    \end{equation*}
    \begin{equation*}
        \sqrt{2}e^{\frac{\pi i}{2}}
    \end{equation*}

    \subsection*{3. Eulersche Formel}
    a) Welche Kurve in der komplexen Zahlenebene wird durch folgende Gleichung dargestellt?
    \begin{equation*}
        | z + 1 - i | = 2 
    \end{equation*}
    \begin{equation*}
        z = x+yi
    \end{equation*}
    \begin{equation*}
        |x+yi+1-i|=2
    \end{equation*}
    \begin{equation*}
        |i(y-1)+(x+1)|
    \end{equation*}
    wobei $i(y-1)$ den imaginären Teil dargestellt und $(x+1)$ den reellen Teil\\
    Es gilt:
    \begin{equation*}
        |z_0| = |x_0 + y_0i| = \sqrt{x_0^2 + y_0^2}
    \end{equation*}
    Somit:
    \begin{equation*}
        \sqrt{\left(y-1\right)^2+\left(x+1\right)^2} = 2
    \end{equation*}
    \begin{equation*}
        \left(y-1\right)^2+\left(x+1\right)^2=4
    \end{equation*}
    dies ist die Kreisfunktion mit einem Radius von $r=2$ und dem Mittelpunkt M(-1,  1).\\\\

    b)  Substituieren Sie z in der obigen Kurvengleichung durch die neue Variable
    \begin{equation*}
        w=\frac{1}{z+i+1}; z=\frac{1}{w}-i-1
    \end{equation*}
    \begin{equation*}
        |\frac{1}{w}-2i|=2
    \end{equation*}
    \begin{equation*}
        |\frac{1-2i(x+yi)}{x+yi}|=2
    \end{equation*}
     \begin{equation*}
        \sqrt{\left(1+2y\right)^2+{\left(-2x\right)^2}} =2\sqrt{x^2+y^2}
     \end{equation*}
     \begin{equation*}
        1+4y+4y^2+4x^2=4x^2+4y^2
     \end{equation*}
     \begin{equation*}
        1+4y=0
     \end{equation*}
     \begin{equation*}
        y=-\frac{1}{4}
     \end{equation*}

     \section*{Woche 2}

     \subsection*{4. Wurzeln}
     Berechnen Sie folgende Wurzeln der komplexen Zahlen.\\
     Benötigt werden hierfür:
    \begin{equation*}
        z = a + ib, r = \sqrt{a^2 + b^2}, \varphi=\arctan\left(\frac{a}{b}\right)
    \end{equation*}
    \begin{equation*}
        \sqrt[n]{z} = \sqrt[n]{r}e^{i\frac{\varphi+2k\pi}{n}}
     \end{equation*}
     a) $\sqrt{-i}$
     \begin{equation*}
        z_0 = \sqrt{r}\cdot e^{\left(\frac{i}{2}\left(\varphi + 2 \pi \cdot 0\right)\right)}
     \end{equation*}
     \begin{equation*}
        z_1 = \sqrt{r} \cdot e^{\left(\frac{i}{2}\left(\varphi + 2 \pi \cdot 1\right)\right)}
     \end{equation*}
     \begin{equation*}
        z_0 = \sqrt{1} \cdot e^{\left(-\frac{i}{2}\left(\frac{\pi}{2} + 2 \pi \cdot 0\right)\right)} = e^{-\frac{\pi}{4} \cdot i}
     \end{equation*}
     \begin{equation*}
        z_1 = \sqrt{1} \cdot e^{\left(\frac{i}{2}\left(-\frac{\pi}{2} + 2 \pi\right)\right)} = e^{\frac{3\pi}{4}\cdot i}
     \end{equation*}
     b) $\sqrt{1+i}$
     \begin{equation*}
        r = \sqrt{1^2 + 1^2} = \sqrt{2}
     \end{equation*}
     \begin{equation*}
        \varphi = \frac{\pi}{4}
     \end{equation*}
     \begin{equation*}
        z_0 = \sqrt[4]{2}e^{i\frac{\pi}{8}+0\pi} = \sqrt[4]{2}e^{\frac{i\pi}{8}}
     \end{equation*}
    \begin{equation*}
        z_1 = \sqrt[4]{2}e^{i \left( \frac{\pi}{8} + \pi\right)} = \sqrt[4]{2}e^{\frac{9i\pi}{8}}
    \end{equation*}

    c) $\sqrt[3]{i}$
    \begin{equation*}
        r = \sqrt{1} = 1
    \end{equation*}
    \begin{equation*}
        \varphi = \frac{\pi}{2}
    \end{equation*}
    \begin{equation*}
        z_0 = \sqrt[3]{1}e^{i\left(\frac{\frac{\pi}{2} + 2\cdot0\cdot\pi}{3}\right)} = e^{\frac{i\pi}{6}}
    \end{equation*}
    \begin{equation*}
        z_1 = e^{i\left(\frac{\frac{\pi}{2} + 2\pi}{3}\right)} = e^{\frac{5i\pi}{6}}
    \end{equation*}
    \begin{equation*}
        z_2 = e^{i\left(\frac{\frac{\pi}{2} + 4\pi}{3}\right)} = e^{\frac{3i\pi}{2}}
    \end{equation*}

    \subsection*{5. Reihen}
    Geometrische Reihe:
    \begin{equation*}
        z = \sum_{n} e^{-\frac{E_n}{kT}}
    \end{equation*}
    \begin{equation*}
        z = \sum_{n}^{\infty} e^{-\frac{(n+\frac{1}{2})h\nu}{kT}}
    \end{equation*}
    \begin{equation*}
        z = \sum_{n}^{\infty} e^{-\frac{nh\nu}{kT}}\cdot e^{-\frac{h\nu}{2kT}}
    \end{equation*}
    \begin{equation*}
        z = e^{-\frac{h\nu}{2kT}} \sum_{n}^{\infty} (e^{-\frac{h\nu}{kT}})^n
    \end{equation*}
    Da nun die Form der geometrischen Reihe vorliegt, wird diese angewandt:
    \begin{equation*}
        z = e^{-\frac{h\nu}{kT}} \cdot \frac{1}{1 - e^{\frac{h\nu}{kT}}}
    \end{equation*}
    \begin{equation*}
        \frac{h\nu}{kT} = a
    \end{equation*}
    \begin{equation*}
        z = \frac{e^{-\frac{a}{2}}}{1-e^{-a}}
    \end{equation*}
    Erweitern mit $e^a$
    \begin{equation*}
        z = \frac{e^{\frac{a}{2}}\cdot e^a}{e^a-e^{-a} \cdot e^a}
    \end{equation*}
    \begin{equation*}
        z = \frac{e^\frac{a}{2}}{e^a-1}
    \end{equation*}
    b)
    \begin{equation*}
        C_V = \frac{\partial}{\partial T}\left(kT^2\frac{\partial}{\partial T}\ln (Z)\right)
    \end{equation*}
    \begin{equation*}
        \ln (Z) = \ln\left(\frac{e^\frac{a}{2}}{e^{a}-1}\right) = \ln\left(e^\frac{h\nu}{2kT}-\ln\left(e^a-1\right)\right) = \frac{a}{2} - \ln\left(e^\frac{h\nu}{kT}-1\right)
    \end{equation*}
    \begin{equation*}
        -\frac{\partial \ln(Z)}{\partial T} = \frac{h\nu}{2kT^2} - \frac{1}{e^{\frac{h\nu}{kT}}-1}\left(-\frac{h\nu}{kT^2}\right)e^{\frac{h\nu}{kT}}
    \end{equation*}
    \begin{equation*}
        -\frac{h\nu}{2kT^2}+\frac{h\nu}{kT^2}\frac{e^{\frac{h\nu}{kT}}}{e^{\frac{h\nu}{kT}}-1}
    \end{equation*}
    \begin{equation*}
        \frac{\partial}{\partial T}\left(kT^2 \left(-\frac{h\nu}{2kT^2}+\frac{h\nu}{kT^2}\frac{e^{\frac{h\nu}{kT}}}{e^{\frac{h\nu}{kT}}-1}\right)\right)
    \end{equation*}
    \begin{equation*}
        h\nu\frac{\partial}{\partial T}\left(-\frac{1}{2}+\frac{e^{\frac{h\nu}{kT}}}{e^{\frac{h\nu}{kT}}-1}\right)
    \end{equation*}
    \begin{equation*}
        h\nu\frac{\partial}{\partial T}\left(-\frac{1}{2}+1+\frac{1}{e^{\frac{h\nu}{kT}}-1}\right)
    \end{equation*}
    \begin{equation*}
        h\nu\frac{\partial}{\partial T}\left(\frac{1}{2}+\frac{1}{e^{\frac{h\nu}{kT}}-1}\right)
    \end{equation*}
    \begin{equation*}
        h\nu\left(-\left(e^{\frac{h\nu}{kT}}-1\right)^2\left(-\frac{h\nu}{kT^2}\right)e^{\frac{h\nu}{kT}}\right)
    \end{equation*}
    \begin{equation*}
        \frac{h^2\nu^2}{kT^2}\frac{e^a}{(e^a-1)^2}
    \end{equation*}
    \begin{equation*}
        \frac{ka^2e^a}{(e^a-1)^2}
    \end{equation*}
        Da $3N$:
        \begin{equation*}
            C_V = 3N\frac{ka^2e^a}{(e^a-1)^2}
        \end{equation*}
    c)\begin{equation*}
        T \to 0: C_V \to 0
    \end{equation*}
    \begin{equation*}
        T \to \infty: C_V \to 3Nk
    \end{equation*}
    \subsection*{6. Reihen}
    Das Quotientenkriterium lautet:
    \begin{equation*}
        \lim_{k \to \infty} \frac{a_{k+1}}{a_k}
    \end{equation*}
    Wobei von $a$ der Summand der Reihe ist. $k$ ist eine frei wählbare Variable nach die Reihe summiert wird.\\
    a)
    \begin{equation*}
        \sum_{n=0}^{\infty} \frac{n^4}{e^{(nx)^2}}
    \end{equation*}
    Die Variablen sind:
    \begin{equation*}
        a = \frac{n^4}{e^{(nx)^2}}, k = n
    \end{equation*}
    Somit:
    \begin{equation*}
        \lim_{n \to \infty} \frac{\frac{(n+1)^4}{e^{((n+1)x)^2}}}{\frac{n^4}{e^{(nx)^2}}} = \lim_{x \to \infty} \frac{(n+1)^4}{e^{((n+1)x)^2}} \cdot \frac{e^{(nx)^2}}{n^4}
    \end{equation*}
    Der limes wird erstmal weggelassen...
    \begin{equation*}
        \frac{(n+1)^4}{n^4} \cdot \frac{e^{(nx)^2}}{e^{(xn+x)^2}} = \left(\frac{(n+1)}{n}\right)^4 \cdot \left(\frac{e^{(nx)}}{e^{(xn+x)}}\right)^2
    \end{equation*}
    Betrachtung des Termes mit $e$:
    \begin{equation*}
        \frac{e^{(nx)}}{e^{(xn+x)}} = \frac{e^{nx}}{e^{nx} \cdot e^x} = \frac{e^{nx}}{e^{nx}} e^{-x} = 1e^{-x} = e^{-x}
    \end{equation*}
    somit:
    \begin{equation*}
        \left(\frac{(n+1)}{n}\right)^4 \cdot \left(e^{-x}\right)^2
    \end{equation*}
    Betrachtung des anderen Termes:
    \begin{equation*}
        \lim_{n \to \infty} \frac{n+1}{n} = \frac{\infty + 1}{\infty} = 1
    \end{equation*}
    Somit:
    \begin{equation*}
        \left(e^{-x}\right)^2 = e^{-2x}; x \in \mathbb{R} 
    \end{equation*}
    für alle $x > 0$, für $x =0$ divergiert diese.\\\\
    b) $\sum_{n=0}^{\infty}\frac{(\sqrt{5}-x)^n}{n^2+x}$
    Die Variablen sind
    \begin{equation*}
        k = n, a = \frac{(\sqrt{5}-x)^n}{n^2+x}
    \end{equation*}
    Somit:
    \begin{equation*}
        \frac{(\sqrt{5}-x)^{n+1}}{(n+1)^2 + x} \cdot \frac{n^2 + x}{(\sqrt{5}-x)^n} = \frac{(\sqrt{5}-x)^{n+1}}{(\sqrt{5}-x)^n} \cdot \frac{n^2 + x}{(n+1)^2 + x}
    \end{equation*}
    \begin{equation*}
        (\sqrt{5}-x) \cdot \frac{n^2 + x}{(n+1)^2 + x}
    \end{equation*}
    Es wird der Term mit $n$ betrachtet:
    \begin{equation*}
        \lim_{n \to \infty} \frac{n^2 + x}{(n+1)^2 + x} = \frac{\infty^2 x}{(\infty+1)^2 x} = \frac{\infty x}{\infty x} = 1
    \end{equation*}
    Somit:
    \begin{equation*}
        (\sqrt{5}-x)\cdot 1 = \sqrt{5}-x; x \in \mathbb{R}
    \end{equation*}
    Die Funktion konvergiert für Werte von $x \in (\sqrt{5}-1, \sqrt{5}+1)$\\
    Für $x = \sqrt{5}$ konvergiert sie auch, da dann 0 bei herauskommt.


    \section*{Woche 3}

    \subsection*{7. Reihen}
    \begin{equation*}
        y(x) = \sum_{k=0}^{\infty} \frac{(-1)^k}{2k+1}x^{2k+2} = x \sum_{k=0}^{\infty} \frac{(-1)^k}{2k+1}x^{2k+1} = x \int \frac{\partial}{\partial x} \left(\sum_{k=0}^{\infty} \frac{(-1)^k}{2k+1}x^{2k+1}\right) \, dx  
    \end{equation*}
    \begin{equation*}
        = x \left(\arctan(x) + c\right) = x \int \sum_{k=0}^{\infty} (-x^2)^k \, dx = x \int \frac{1}{1+x^2} \, dx
    \end{equation*}

    \subsection*{8. Reihen}
    \begin{equation*}
        f(x) = x\pi
    \end{equation*}
    mit $[0; 2\pi[$:
    \begin{equation*}
        f(x) = \frac{a_0}{2} + \sum_{n=1}^{\infty} a_n \cos(nx)+b_n\sin(nx)
    \end{equation*}
    \begin{equation*}
        f(x)=\frac{1}{\sqrt{2\pi}}\sum_{k = -\infty}^{\infty} c_k e^{ikx}
    \end{equation*}
    \begin{equation*}
        c_0 = \frac{1}{\sqrt{2\pi}} \int_{0}^{2\pi} f(x) e^0 \,dx
    \end{equation*}
    \begin{equation*}
        \frac{1}{\sqrt{2\pi}} \left[\frac{1}{2} 4\pi^2 - 2\pi^2\right] = 0
    \end{equation*}
    \begin{equation*}
        k \neq 0, \, c_k = \frac{1}{\sqrt{2\pi}} \int_{0}^{1} (x-\pi)e^{ikx} \, dx = \frac{1}{\sqrt{2\pi}} \int_{0}^{1} \left[(x-\pi)\frac{1}{ik}e^{ikx}\right]^{2\pi}_0 + \frac{1}{ik} \int_{0}^{2\pi} \frac{e^{-ikx}}{-ik} \, dx
    \end{equation*}
    \begin{equation*}
        = \frac{1}{\sqrt{2\pi}}-\left(\frac{\pi}{-ik} - \frac{-\pi}{-ik} + \frac{1}{ik}\left(\frac{1}{-ik}-\frac{1}{-ik}\right)\right) = \frac{i\sqrt{2\pi}}{k}
    \end{equation*}
    \begin{equation*}
        \tilde{f} = \sum_{k=-\infty}^{\infty'} \frac{i}{k} e^{ikx} = \sum_{k=-\infty}^{\infty'} \frac{i}{k} \left(\cos(kx)+i\sin(kx)\right) = \sum_{k=1}^{\infty} \frac{-2}{k}\sin(kx)
    \end{equation*}
    \begin{equation*}
        \frac{\pi}{2} = 2(1-\frac{1}{3}+\frac{1}{5}-\dots)
    \end{equation*}
    \begin{equation*}
        \frac{\pi}{2} = \sum_{k=1}^{\infty} \frac{-2}{k} \sin\left(\frac{3\pi}{2}\right) \rightarrow 1, 0, -1, 0, 1, \dots
    \end{equation*}
    \begin{equation*}
        \sum_{k=1}^{\infty} \frac{-2}{2k+1}\left((-1)^k\right)
    \end{equation*}

\section*{BEI FOLGENDEN AUFGABEN WAR ICH NICHT DA:}

\section*{Woche 4}
    \subsection*{9.}
    \begin{eqnarray*}
        f\left(x\right)=\frac{1}{2\pi}\frac{1}{\sqrt{2\pi}}\sum_{-\infty}^{\infty} \frac{\sin(k\varphi)}{\sqrt{2\pi}k\varphi}e^{ikx}\\
            =\frac{1}{2\pi\varphi}\sum_{-\infty}^{\infty}\frac{\sin(k\varphi)}{k}e^{ikx}+\varphi\\
             \lim_{\varphi}\frac{1}{2\pi}\sum_{-\infty}^{\infty}\frac{k\cos(k\varphi)}{k}e^{ikx}\\
            =\frac{1}{2\pi}\sum_{-\infty}^{\infty}e^{ikx}
    \end{eqnarray*}

\section*{Woche 5}
\subsection*{Aufgabe 11}
\begin{equation*}
    p(x) = a_0+a_1(x-x_0)+a_2(x-x_0)(x-x_1)\\
\end{equation*}
\begin{eqnarray*}
    a_0 = 0\\
    a_1 = \frac{1}{60}
    a_2 = \frac{2\sqrt{2}-3}{2700}
\end{eqnarray*}
\begin{equation*}
    p(x) = -6.35\cdot10^{5}x^2+1.86\cdot 10^{-2}x
\end{equation*}
\begin{equation*}
    p(35) = 0.57
\end{equation*}
Absolut kein Sinn Diggi

\subsection*{Aufgabe 12}
\begin{equation*}
    P_2(x)=c_0+c_1x+c_2x^2
\end{equation*}
\begin{center}
    \begin{tabular}{c | c c c c}
        \hline
        $i$ & 1 & 2 & 3 & 4\\
        \hline
        $x_i$ & 2 &0&-1&3\\
        $\phi_1(x_i)$&1&1&1&1\\
        $\phi_2(x_i)$&2&0&-1&3\\
        $\phi_3(x_i)$&4&0&1&9\\
        $y_i$&4&2&11&7\\
        \hline
    \end{tabular}
\end{center}
\begin{center}
    \begin{tabular}{c c c | c}
        $A_{11}$&$A_{12}$&$A_{13}$&$b_1$\\
        $A_{21}$&$A_{22}$&$A_{23}$&$b_2$\\
        $A_{31}$&$A_{32}$&$A_{33}$&$b_3$\\
    \end{tabular}
\end{center}
\begin{eqnarray*}
    A_{LK}=\sum_{i=1}^{4} \phi_L(x_i)\phi_K(x_i)\\
    A_{11} = \sum_{i=1}^{4} \phi_1(x_i)\phi_1(x_i)\\
    A_{12}= \sum_{i=1}^{4} \phi_1(x_i)\phi_2(x_i) = A_21\\
\end{eqnarray*}
\begin{eqnarray*}
    A_{11} = 4 = A_{12} = A_{21}\\
    A_{22} = 14\\
    A_{23} =34 = A_{32}\\
    A_{33} = 98
\end{eqnarray*}
\begin{eqnarray*}
    \vec{b_L} = \sum_{i=1}^{4} y_i \phi_L(x_i)\\
    \vec{b_1} = 4+2+11+7 = 24\\
    \vec{b_2} = 18\\
    \vec{b_3} = 90
\end{eqnarray*}
\begin{center}
    \begin{tabular}{c c c | c}
        4&4&14&24\\
        4&14&34&18\\
        14&34&98&90\\
    \end{tabular}
\end{center}
\begin{eqnarray*}
    c_1 = 3.6\\
    c_2 = -4.6\\
    c_3 = 2\\
    P_2(x)=2x^2-4.6x+3.6
\end{eqnarray*}
\subsection*{Aufgabe 13}
a)\begin{equation*}
    \left(\begin{array}{c c c | c}
        3&3&-3&0\\
        2&-1&-8&0\\
    \end{array}\right)
\end{equation*}
3 Vektoren in 2 Dimensionen haben unendlich viele Lösungen, deshalb: linear abhängig\\
b)\begin{equation*}
    \left|\begin{array}{c c c}3&3&-3\\2&-1&-8\\1&2&1\end{array}\right| = 0
\end{equation*}
Linear abhängig\\
c)\begin{equation*}
    \left|\begin{array}{c c c c}3&3&-3&-3\\2&-1&-8&-8\\1&2&1&0\\0&0&1&0\end{array}\right| = \left|\begin{array}{c c c}3&3&-3\\2&-1&-8\\1&2&0\end{array}\right| = 3
\end{equation*}
Nicht linear Abhängig.

\section*{Woche 6}
\subsection*{Aufgabe 14}
Lineare Unabhängigkeit:
\begin{equation*}
    \begin{array}{| c c c |}
        1&2&2\\
        1&3&2\\
        0&2&3\\
        1&1&2
    \end{array} = 9+8-4-6 = 7
\end{equation*}
Normierung von $\vec{a_1}$:
\begin{eqnarray*}
    \hat{a_1} = \frac{\vec{a_1}}{||\vec{a_1}||} = \frac{(1,1,0,1)}{\sqrt{3}} = \frac{1}{\sqrt{3}}\cdot \left(\begin{array}{c}1\\1\\0\\1\end{array}\right)
\end{eqnarray*}
Orthogonalisierung von $\vec{a_2}$ auf $\hat{a_1}$:
\begin{eqnarray*}
    \vec{a_2'} = \vec{a_2} - (\hat{a_1}\cdot\vec{a_2}\cdot\hat{a_1}) = \left(\begin{array}{c}2\\3\\2\\1\end{array}\right)-\left(\begin{array}{c}2\\2\\0\\2\end{array}\right) = \left(\begin{array}{c}0\\1\\2\\-1\end{array}\right)
\end{eqnarray*}
Normieren bon $\vec{a_2'}$:
\begin{equation*}
    \hat{a_2'}=\frac{\vec{a_2'}}{||\vec{a_2'}||} = \frac{1}{\sqrt{6}} \left(\begin{array}{c}0\\1\\2\\-1\end{array}\right)
\end{equation*}
Orthogonalisierung von $\vec{a_3'}$ auf $\hat{a_1}$ und $\hat{a_2}$:
\begin{eqnarray*}
    \vec{a_3'}=\vec{a_3}-(\hat{a_1}\vec{a_3})\hat{a_1}-(\hat{a_2'}\vec{a_3})\hat{a_2'} = 2\left(\begin{array}{c}1\\1\\0\\1\end{array}\right) - \left(\begin{array}{c}0\\1\\2\\-1\end{array}\right)\\
    \left(\begin{array}{c}2\\2\\3\\2\end{array}\right)-\left(\begin{array}{c}2\\2\\0\\2\end{array}\right)-\left(\begin{array}{c}0\\1\\2\\-1\end{array}\right) = \left(\begin{array}{c}0\\-1\\1\\1\end{array}\right)
\end{eqnarray*}
Normierung:
\begin{equation*}
    \vec{a_3'}=\frac{1}{\sqrt{3}}\left(\begin{array}{c}0\\-1\\1\\1\end{array}\right)
\end{equation*}

\subsection*{Aufgabe 15}
a) Operator: Spiegelung an $\left(\begin{array}{c}0\\1\\1\end{array}\right)$ somit:
\begin{equation*}
    \left(\begin{array}{c c c}
        -1&0&0\\0&1&0\\0&0&1
    \end{array}\right)
\end{equation*}
b) Drehung um $z$-Achse um Winkel $\varphi$
\begin{eqnarray*}
    B= \left(\begin{array}{c c c}
        \cos(\varphi)&-\sin(\varphi)&0\\
        \sin(\varphi)&\cos(\varphi)&0\\
        0&0&1
    \end{array}\right)
\end{eqnarray*}
c) Winkel in $0<\varphi<2\pi$ für die $A,B$ vertauschbar sind. Somit $AB - BA = 0$
\begin{equation*}
    AB = \left(\begin{array}{c c c}
        -\cos(\varphi)&\sin(\varphi)&0\\
        \sin(\varphi)&\cos(\varphi)&0\\
        0&0&1
    \end{array}\right)
\end{equation*}
\begin{equation*}
    BA = \left(\begin{array}{c c c}
        -\cos(\varphi)&-\sin(\varphi)&0\\
        -\sin(\varphi)&\cos(\varphi)&0\\
        0&0&1
    \end{array}\right)
\end{equation*}
Aus den Operatoren ist zu sehen, dass sich nur das Vorzeichen des $\sin$ verändert, somit muss gelte $\sin(\varphi) = -\sin(\varphi)$\\
Somit $\varphi = k\pi,\, k \in \mathbb{Z}$\\
Also: $\varphi = \pi$ 

\section*{Woche 6}
\subsection*{1}
\begin{align*}
    \left(\begin{array}{c c c}
        1&1&0\\0&0&\sqrt{2}\\1&-1&0
    \end{array}\right)\\
    0&=\left(\begin{array}{c c c}
        1-\lambda&1&0\\0&0-\lambda&\sqrt{2}\\1&-1&-\lambda
    \end{array}\right)\\
    &= (1-\lambda)\lambda^2+\sqrt{2}+\sqrt{2}(1-\lambda)\\
    &= \lambda^2-\lambda^3+\sqrt{2}+\sqrt{2}-\sqrt{2}\lambda\\
    &= -\lambda^3+\lambda^2-\sqrt{2}\lambda+2\sqrt{2}\\
    -(\sqrt{2})^3+(\sqrt{2}^2)-\sqrt{2}\cdot\sqrt{2}+2\sqrt{2}\\
    -2\sqrt{2}+2-2+2\sqrt{2}&=0\\
\end{align*}
$\lambda =\sqrt{2}$:
\begin{align*}
    I &(1-\sqrt{2})x+y=0\\
    II &-\sqrt{2}y+\sqrt{2}z=0\\
    II &x-y-\sqrt{2}z=0
\end{align*}

\section*{Woche 7}
\subsection*{Aufgabe 1}
\begin{align*}
    k_1c_A(t) &= k_2c_B(t)\\
    \frac{1}{\mathrm{s}}c_A(t)&=\frac{0.2}{\mathrm{s}}c_B(t)\\
    5 = \frac{c_B(t)}{c_A(t)}\\
    c_B(t) &= 1\,\mathrm{\frac{mol}{l}} - c_A(t)\\
    5 &= \frac{1\,\mathrm{\frac{mol}{l}} - c_A(t)}{c_A(t)}\\
    c_A(t) &= \frac{1}{6}\,\mathrm{\frac{mol}{l}}\\
    c_B(t) &= \frac{5}{6}\,\mathrm{\frac{mol}{l}}\\
    \frac{d[A]}{dt} &= -(k_1 + k_2)([A]-[A](\infty))\\
    \int_{[A](0)}^{[A]} \frac{d[A]}{[A]-[A](\infty)} &= -(k_1+k_2) \int_{t=0}^{t} \,dt\\
    \ln \frac{[A](0)-[A](\infty)}{[A]-[A](\infty)} &= (k_1+k_2)t\\
    [A] &= \left(e^{(k_1+k_2)t}\left([A](0)-[A](\infty)\right)\right) + [A](\infty)\\
    [A](1) &= 0.54\,\mathrm{\frac{mol}{l}}\\
    [A](10) &= 0.17\,\mathrm{\frac{mol}{l}}\\
    [B](1) &= 0.46\,\mathrm{\frac{mol}{l}}\\
    [B](10) &= 0.83\,\mathrm{\frac{mol}{l}}
\end{align*}
\subsection*{Aufgabe 2}
a)
\begin{align*}
    y = \frac{1}{x-2},\,y'= \frac{1}{(x-2)^2},\, y''=\frac{1}{(x-2)^3}\\
    2(x-2)\frac{1}{(x-2)^3}+(x+2)\frac{1}{(x-2)^2}+\frac{1}{x-2} &= 0\\
    \frac{4}{(x-2)^2}-\frac{x+2}{(x-2)^2}+\frac{x-2}{(x-2)^2} &= 0\\
    4+(-x-2)+(x-2) &= 0\\
    0 &= 0\\
\end{align*}
\subsection*{Aufgabe 3}
\begin{align*}
    y = \sum_{n=0} a_nx^n,\,y' = \sum_{n=0} na_nx^{n-1},\,y = \sum_{n=0} n(n-1)a_nx^{n-2}\\
    x \sum_{n=0} n(n-1)a_nx^{n-2} - \sum_{n=0} na_nx^{n-1} + x \sum_{n=0} na_nx^{n-1}+\sum_{n=0} a_nx^n &= 0\\
    \sum_{n=1} n(n-2)a_nx^{n-1}+\sum_{n=0} (1-n) a_nx^n &=0\\
    \sum_{n=0} ((n+1)(n-1)a_{n+1}-(n-1)a_n)x^n &=0\\
    a_0 = 1, a_1 = 1 a_2 = \frac{1}{2}\dots\\
    \text{Für }1+x:\\
    -(1-x)+(1+x)\\
    y=c_1e^x+c_2(1+2)
\end{align*}

\section*{Woche 8}
\subsection*{Aufgabe 1}
a)
\begin{align*}
    y'' - y' - 2y &= -4x^2-4x+10\\
    y&= ax^2 + bx + c\\
    y'&=2ax + b\\
    y'' &= 2a\\
    2a-(2ax+b)-2(ax^2+bx+c)&=-4x^2-4x+10\\
    2a-2ax-b-2ax^2-2bx-2c &= -4x^2-4x+10\\
    A = 2,\, B = 0,\,C=-3\\
\end{align*}
spezielle lösung:
\begin{equation*}
    y=2x^2-3
\end{equation*}
allgemeine Lösung:
\begin{equation*}
    y=c_1e^{-x}+c_2xe^{-x}+2x^2-3
\end{equation*}
b)

\subsection*{Aufgabe 2}
\begin{align*}
    a^2 &\neq b^2\\y''+a^2y&=2\cos(bx)\\
    \text{Homogen:}\\
    \lambda^2+a^2&=0\\
    \rightarrow\lambda_{1,2} &= \pm ia\\
    y_h &= c_1 e^{-iax}+c_2e^{iax}\\
    \text{Partikulär:}\\
    y_p &= c_3 \cos(bx)\\
    y_p' &= -c_3\sin(bx)\\
    y_p''&=-c_3b^2\cos(bx)\\
    -c_3b ^2\cos(bx)+a^2c_3\cos(bx)&=2\cos(bx)\\
    -c_3b^2+a^2c_3&=2\\
    c_3(a^2-b^2)&=2\\
\end{align*}
a)
\begin{align*}
    a^2 &\neq b^2\\
    c_3(a^2-b^2)&=2\\
    c_3 &= \frac{2}{(a^2-b^2)}\\
    y&=y_h+y_p\\
    &= c_1e^{-iax}+c_2e^{iax}+\frac{2}{(a^2-b^2)}\cdot \cos(bx)\\
\end{align*}
b)
\begin{align*}
    a^2 &= b^2\\
    c_3 &\neq 2\frac{2}{0}
\end{align*}
Durch 0 kann man nicht teilen... nicht lösbar.
\end{document}