\documentclass{article}

\usepackage{amsmath}
\usepackage{amssymb}

\begin{document}
    \section*{Woche 1}
    \subsection*{1. Rechenoperationen}
    Stellen Sie in der Form a + ib dar:
    \begin{equation*}
        \frac{1+i}{1-i}
    \end{equation*}
    \begin{equation*}
        \frac{1}{3i}\left(6-5i+\frac{1+5i}{1+i}\right)
    \end{equation*}
    \begin{equation*}
        \left(1-i\right)^{14}
    \end{equation*}
    \\
    a) \begin{equation*}
        \frac{1+i}{1-i} = \frac{\left(1+i\right)\left(1+i\right)}{\left(1-i\right)\left(1+i\right)}=\frac{1+2i+i^2}{1-i^2}=\frac{1+2i-1}{1-\left(-1\right)}=\frac{2i}{2}=i
    \end{equation*}
    b) \begin{equation*}
        \frac{1}{3i}\left(6-5i+\frac{1+5i}{1+i}\right)=\frac{1}{3i}\left(9-3i\right)=-1-3i
    \end{equation*}
    c) \begin{equation*}
        \left(1-i\right)^{14}=\left(1-i\right)^2\left(1-i\right)^2\left(1-i\right)^2\left(1-i\right)^2\left(1-i\right)^2\left(1-i\right)^2\left(1-i\right)^2
    \end{equation*}
    \begin{equation*}
        =2i\cdot 2i\cdot 2i\cdot 2i\cdot 2i\cdot 2i\cdot 2i=128i^7=128i^2\cdot i^2\cdot i^2\cdot i
    \end{equation*}
    \begin{equation*}
        -128\cdot i^2\cdot i^2\cdot i=128\cdot i^2\cdot i=-128i
    \end{equation*}

    \subsection*{2. Eulersche Formel}
    Stellen Sie in Polarkoordinaten $z = re^{i\varphi}$ dar:
    a) $1-i$ \begin{equation*}
        r=\sqrt{1^2+\left(-1\right)^2}=\sqrt{2}; \varphi=\arctan(\frac{1}{1})=45^{\circ}=\frac{\pi}{4}
    \end{equation*}
    \begin{equation*}
        \sqrt{2}e^{\frac{\pi i}{4}}
    \end{equation*}
    b) $-\sqrt{3}+3i$\begin{equation*}
        r=\sqrt{\left(-\sqrt{3}\right)^2+3^2}=\sqrt{12}; \varphi=\arctan(-\frac{3}{\sqrt{3}})=-\frac{\pi}{3}
    \end{equation*}
    \begin{equation*}
        \sqrt{12}e^{\frac{\pi i}{3}}
    \end{equation*}
    c) $\sqrt{2}i$\begin{equation*}
        r=\sqrt{2};\varphi=90^\circ=\frac{\pi}{2}
    \end{equation*}
    \begin{equation*}
        \sqrt{2}e^{\frac{\pi i}{2}}
    \end{equation*}

    \subsection*{3. Eulersche Formel}
    a) Welche Kurve in der komplexen Zahlenebene wird durch folgende Gleichung dargestellt?
    \begin{equation*}
        | z + 1 - i | = 2 
    \end{equation*}
    \begin{equation*}
        z = x+yi
    \end{equation*}
    \begin{equation*}
        |x+yi+1-i|=2
    \end{equation*}
    \begin{equation*}
        |i(y-1)+(x+1)|
    \end{equation*}
    wobei $i(y-1)$ den imaginären Teil dargestellt und $(x+1)$ den reellen Teil\\
    Es gilt:
    \begin{equation*}
        |z_0| = |x_0 + y_0i| = \sqrt{x_0^2 + y_0^2}
    \end{equation*}
    Somit:
    \begin{equation*}
        \sqrt{\left(y-1\right)^2+\left(x+1\right)^2} = 2
    \end{equation*}
    \begin{equation*}
        \left(y-1\right)^2+\left(x+1\right)^2=4
    \end{equation*}
    dies ist die Kreisfunktion mit einem Radius von $r=2$ und dem Mittelpunkt M(-1,  1).\\\\

    b)  Substituieren Sie z in der obigen Kurvengleichung durch die neue Variable
    \begin{equation*}
        w=\frac{1}{z+i+1}; z=\frac{1}{w}-i-1
    \end{equation*}
    \begin{equation*}
        |\frac{1}{w}-2i|=2
    \end{equation*}
    \begin{equation*}
        |\frac{1-2i(x+yi)}{x+yi}|=2
    \end{equation*}
     \begin{equation*}
        \sqrt{\left(1+2y\right)^2+{\left(-2x\right)^2}} =2\sqrt{x^2+y^2}
     \end{equation*}
     \begin{equation*}
        1+4y+4y^2+4x^2=4x^2+4y^2
     \end{equation*}
     \begin{equation*}
        1+4y=0
     \end{equation*}
     \begin{equation*}
        y=-\frac{1}{4}
     \end{equation*}

     \section*{Woche 2}

     \subsection*{4. Nullstellen}
     Berechnen und zeichnen Sie:\\
     a) $\sqrt{-i}$
     \begin{equation*}
        z_0 = \sqrt{r}\cdot e^{\left(\frac{i}{2}\left(\varphi + 2 \pi \cdot 0\right)\right)}
     \end{equation*}
     \begin{equation*}
        z_1 = \sqrt{r} \cdot e^{\left(\frac{i}{2}\left(\varphi + 2 \pi \cdot 1\right)\right)}
     \end{equation*}
     \begin{equation*}
        z_0 = \sqrt{1} \cdot e^{\left(-\frac{i}{2}\left(\frac{\pi}{2} + 2 \pi \cdot 0\right)\right)} = e^{-\frac{\pi}{4} \cdot i}
     \end{equation*}
     \begin{equation*}
        z_1 = \sqrt{1} \cdot e^{\left(\frac{i}{2}\left(-\frac{\pi}{2} + 2 \pi\right)\right)} = e^{\frac{3\pi}{4}\cdot i}
     \end{equation*}
     b) $\sqrt{1+i}$
     \begin{equation*}
        z = a + ib, r = \sqrt{a^2 + b^2}, \varphi=\arctan\left(\frac{a}{b}\right)
     \end{equation*}
     \begin{equation*}
        r = \sqrt{1^2 + 1^2} = \sqrt{2}
     \end{equation*}
     \begin{equation*}
        \varphi = \frac{\pi}{4}
     \end{equation*}
     \begin{equation*}
        \sqrt[n]{z} = \sqrt[n]{r}e^{i\frac{\varphi+2k\pi}{n}}
     \end{equation*}
     \begin{equation*}
        z_0 = \sqrt{2}e^{i\frac{\pi}{8}+0\pi} = \sqrt{2}e^{\frac{i\pi}{8}}
     \end{equation*}
    \begin{equation*}
        z_1 = \sqrt{2}e^{i \left( \frac{\pi}{8} + \pi\right)} = \sqrt{2}e^{\frac{9i\pi}{8}}
    \end{equation*}

    c) $\sqrt[3]{i}$
    \begin{equation*}
        r = \sqrt{1} = 1
    \end{equation*}
    \begin{equation*}
        \varphi = \frac{\pi}{2}
    \end{equation*}
    \begin{equation*}
        z_0 = \sqrt[3]{1}e^{i\left(\frac{\frac{\pi}{2} + 2\cdot0\cdot\pi}{3}\right)} = e^{\frac{i\pi}{6}}
    \end{equation*}
    \begin{equation*}
        z_1 = e^{i\left(\frac{\frac{\pi}{2} + 2\pi}{3}\right)} = e^{\frac{5i\pi}{6}}
    \end{equation*}
    \begin{equation*}
        z_2 = e^{i\left(\frac{\frac{\pi}{2} + 4\pi}{3}\right)} = e^{\frac{3i\pi}{2}}
    \end{equation*}

    \subsection*{5. Reihen}
    \begin{equation*}
        Z = \sum_{n}e^{\frac{-E_n}{kT}}
    \end{equation*}
    Die Geometrische Summenformel besagt:
    \begin{equation*}
        \sum_{k=0}^{n} q^k = \frac{1-q^{n+1}}{1-q}
    \end{equation*}
    in dem Fall sind
    \begin{equation*}
        n = \infty; q = e; k = \frac{-E_n}{kT}
    \end{equation*}
    Somit kann erhält man durch Einsetzen:
    \begin{equation*}
        \sum_{n}^{\infty} e^{\frac{-E_n}{kT}}
    \end{equation*}
    Da $E_n = (\frac{1}{2} + n)h\nu$:
    \begin{equation*}
        \sum_{n}^{\infty} e^{\frac{-(\frac{1}{2}+n)h\nu}{kT}}
    \end{equation*}
    Nun wird der Exponent $k = \frac{-(\frac{1}{2}+n)h\nu}{kT}$ wie folgt betrachtet:
    Es wird angenommen, dass $kT = \mathrm{const.}$ und $h\nu = \mathrm{const.}$
    \begin{equation*}
        \lim_{n \to \infty} \frac{-(\frac{1}{2}+\infty)h\nu}{kT} = \frac{-\infty}{kT} = -\infty
    \end{equation*}
    Somit kann in $\sum_{k=0}^{n} q^k = \frac{1-q^{n+1}}{1-q}$ das $n+1 = -\infty$ gesetzt werden, was zu folgender Betrachtung führt:
    \begin{equation*}
        e^x = \lim_{x \to -\infty} e^x = e^{-\infty} \to 0
    \end{equation*}
    Somit kann angenommen werden, dass $e^{-\infty} = 0$\\
    Mit dieser Betrachtung kann nun alles eingesetzt werden
    \begin{equation*}
        \sum_{n}^{\infty}e^{\frac{-E_n}{kT}} = \frac{1}{1-e} = (1-e)^{-1}
    \end{equation*}
    b) Es wird angenommen, dass das Ergebnis bei a) falsch ist, da $ln(\frac{1}{1-e})$ nicht existiert.\\
    somit wird mit
    \begin{equation*}
        z = \frac{e^{\frac{a}{2}}}{e^{a-1}}; a = \frac{h \nu}{kT}
    \end{equation*}
    weitergerechnet.
    \begin{equation*}
        C_V = \frac{\partial}{\partial T}\left(kT^2\frac{\partial}{\partial T}\ln (Z)\right)
    \end{equation*}
    \begin{equation*}
        \ln (Z) = \frac{a}{2} - a - 1 = -\frac{1}{2}a - 1 = -\frac{1}{2} \frac{h\nu}{kT} - 1
    \end{equation*}
    \begin{equation*}
        \frac{\partial}{\partial T} \left(-\frac{1}{2} \frac{h\nu}{kT} - 1\right)= \frac{1}{2}\frac{h\nu}{kT^2}
    \end{equation*}
    \begin{equation*}
        kT^2 \cdot \frac{1}{2} \frac{h\nu}{kT^2} = \frac{h \nu}{2}
    \end{equation*}
    \begin{equation*}
        \frac{\partial}{\partial T} \frac{h\nu}{2} = 1
    \end{equation*}
    Ja ne Cheffe, ich habe keinen Plan was ich hier mache lmao
    \subsection*{6. Reihen}
    Das Quotientenkriterium lautet:
    \begin{equation*}
        \lim_{k \to \infty} \frac{a_{k+1}}{a_k}
    \end{equation*}
    Wobei von $a$ der Summand der Reihe ist. $k$ ist eine frei wählbare Variable nach die Reihe summiert wird.\\
    a)
    \begin{equation*}
        \sum_{n=0}^{\infty} \frac{n^4}{e^{(nx)^2}}
    \end{equation*}
    Die Variablen sind:
    \begin{equation*}
        a = \frac{n^4}{e^{(nx)^2}}, k = n
    \end{equation*}
    Somit:
    \begin{equation*}
        \lim_{n \to \infty} \frac{\frac{(n+1)^4}{e^{((n+1)x)^2}}}{\frac{n^4}{e^{(nx)^2}}} = \lim_{x \to \infty} \frac{(n+1)^4}{e^{((n+1)x)^2}} \cdot \frac{e^{(nx)^2}}{n^4}
    \end{equation*}
    Der limes wird erstmal weggelassen...
    \begin{equation*}
        \frac{(n+1)^4}{n^4} \cdot \frac{e^{(nx)^2}}{e^{(xn+x)^2}} = \left(\frac{(n+1)}{n}\right)^4 \cdot \left(\frac{e^{(nx)}}{e^{(xn+x)}}\right)^2
    \end{equation*}
    Betrachtung des Termes mit $e$:
    \begin{equation*}
        \frac{e^{(nx)}}{e^{(xn+x)}} = \frac{e^{nx}}{e^{nx} \cdot e^x} = \frac{e^{nx}}{e^{nx}} e^{-x} = 1e^{-x} = e^{-x}
    \end{equation*}
    somit:
    \begin{equation*}
        \left(\frac{(n+1)}{n}\right)^4 \cdot \left(e^{-x}\right)^2
    \end{equation*}
    Betrachtung des anderen Termes:
    \begin{equation*}
        \lim_{n \to \infty} \frac{n+1}{n} = \frac{\infty + 1}{\infty} = 1
    \end{equation*}
    Somit:
    \begin{equation*}
        \left(e^{-x}\right)^2 = e^{-2x}; x \in \mathbb{R} 
    \end{equation*}
    \vspace*{1cm}
    b) $\sum_{n=0}^{\infty}\frac{(\sqrt{5}-x)^n}{n^2+x}$
    Die Variablen sind
    \begin{equation*}
        k = n, a = \frac{(\sqrt{5}-x)^n}{n^2+x}
    \end{equation*}
    Somit:
    \begin{equation*}
        \frac{(\sqrt{5}-x)^{n+1}}{(n+1)^2 + x} \cdot \frac{n^2 + x}{(\sqrt{5}-x)^n} = \frac{(\sqrt{5}-x)^{n+1}}{(\sqrt{5}-x)^n} \cdot \frac{n^2 + x}{(n+1)^2 + x}
    \end{equation*}
    \begin{equation*}
        (\sqrt{5}-x) \cdot \frac{n^2 + x}{(n+1)^2 + x}
    \end{equation*}
    Es wird der Term mit $n$ betrachtet:
    \begin{equation*}
        \lim_{n \to \infty} frac{n^2 + x}{(n+1)^2 + x} = \frac{\infty^2 x}{(\infty+1)^2 x} = \frac{\infty x}{\infty x} = 1
    \end{equation*}
    Somit:
    \begin{equation*}
        (\sqrt{5}-x)\cdot 1 = \sqrt{5}-x; x \in \mathbb{R}
    \end{equation*}
\end{document}