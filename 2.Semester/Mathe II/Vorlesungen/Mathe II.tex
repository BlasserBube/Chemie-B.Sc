\documentclass{article}
\usepackage{mhchem}
\usepackage{amsmath}
\usepackage{amssymb}

\begin{document}

\section{Lineare Algebra}
\subsection{Lineare Vektorräume}
Beispiel für Vektoren mit mehr als Komponenten\\
\hspace*{1cm} Quantenmechanik:\\
\hspace*{1cm} MO-LCAO: \underline{M}olecular \underline{O}rbitals by \underline{L}inear \underline{C}ombination of \underline{A}tomic \underline{O}rbitals\\
Das bindende $\sigma$-MO im \ce{H2} ist
\begin{equation*}
    |\sigma\rangle =c_1(|1s_A+|1s_B\rangle )+c_2(|2s_A\rangle +|2s_B\rangle )+c_3(|2p_{zA}\rangle +\dots)
\end{equation*}
MO-Koeffizientenvektor: $(c_1,c_2,c_3,c_4,\dots)$
Def.: Eine Menge V von Vektoren {${\vec{a},\vec{b},\dots}$} heißen linearer Vektorraum, wenn gilt:
\vspace*{1cm}
\begin{itemize}
    \item Für alle Elemente $\vec{a}$, $\vec{b}$ Element von V gibt es genau ein Element $\vec{a} + \vec{b}$ Element von V (Die Summe zweier Elemente von V gehört wieder zu V)
    \item Assoziativgesetz: $\vec{a}+(\vec{b}+\vec{c}) = (\vec{a}+\vec{b})+\vec{c}$ für $\vec{a}, \vec{b}, \vec{c}$ Element von V
    \item Kommutativgesetz: $\vec{a}+\vec{b} = \vec{b} + \vec{a}$
    \item Neutrales Element "Nullstelle $\vec{o}$ Element von V: $\vec{a} + \vec{o} = \vec{a}$
    \item Inverses Element $-\vec{a}$: $\vec{a} + (-\vec{a}) = 0$
\end{itemize}
Multiplikation mit reellen/komplexen Zahlen
\begin{itemize}
    \item Für alle Elemente $\vec{a}$ Element V und z Element C gibt es genau ein Element $z\cdot\vec{a}$ Element V
    \item Assoziativgesetz
    \item Distributivgesetz
\end{itemize}
\subsection{Dimension und Basis eines Vektorraums}
Def.: Die Vektoren {$\vec{a_1},\dots,\vec{a_n}$} heißen linear unabhängig, wenn $\sum_{k=1}^{n} z_k \vec{a_k}=\vec{o}$ nur für $\vec{z_k} = 0 (k=1,\dots,\dots)$ erfüllbar ist\\
Ist mindestens ein $z_k \neq 0$ wählbar, so heißen die Vektoren linear abhängig, d.h. ein Vektor lässt sich als Linearkombination der anderen schreiben:\\
$\vec{a_k}=-\frac{1}{z_k}\sum_{l=1, l\neq k}^{n}z_l \vec{a_l} = \sum_{l=1,l\neq k}^{a}{z_b}\vec{a_l}$ \hspace*{1cm} $z_b=-\frac{z_l}{z_k}$\\
Beispiel
\begin{equation*}
    \left[\left(\begin{array}{c}
        0\\ 1\\ 1
    \end{array}\right),\left(\begin{array}{c}
        1\\ 0\\ 1
    \end{array}\right),\left(\begin{array}{c}
        1\\ 1\\ 0
    \end{array}\right)\right]
\end{equation*}
\begin{equation*}
    z_1\left(\begin{array}{c}
        0\\ 1\\ 1
    \end{array}\right)+z_2\left(\begin{array}{c}
        1\\ 0\\ 1
    \end{array}\right)+z_3\left(\begin{array}{c}
        1\\ 1\\ 0
    \end{array}\right) = \vec{o}
\end{equation*}
\begin{equation*}
    z_1 0+z_2 0 + z_3 1 = 0
\end{equation*}
\begin{equation*}
    z_1 1+z_2 0 + z_3 1 = 0
\end{equation*}
\begin{equation*}
    z_1 1 +z_2 1 + z_3 0 = 0
\end{equation*}
Auflösen nach $z_1,z_2,z_3$ durch Addition bzw. Subtraktion von Vielfachen der Zeilen (Einzelgleichungen) bis Dreiecksform erreicht wurde, anschließend $z_1,z_2,z_3$ durch Rück-rechnung ermitteln, dies ist der Gaußsche Algorithmus\\
Hierbei gibt es 4 Möglichkeiten, dies ist eine Lösung:\\
\begin{equation*}
    0, 0, 2 | 0
\end{equation*}\
\begin{equation*}
    0, -1, 1 | 0
\end{equation*}
\begin{equation*}
    1, 1, 0 | 0
\end{equation*}
Durch Rückrechnung:
\begin{equation*}
    z_3 = 0, z_2 = 0, z_1 = 0
\end{equation*}
Die Vektoren sind linear unabhängig voneinandern d.h. (hier) sie lieben nicht in einer Ebene aber auf einer Geraden!\\
Das von den Vektoren aufgespannte Spat hat ein Volumen $\neq$ 0, d.h. die Detrminante ist ungleich 0, hier ergibt die Determinante 2.\\
Das Spatprodukt geht nur mit 3 3D-Vektoren, die allgemeine Determinantenmethode geht mit $n$ $n$D-Vektoren\\
Entstünde beim LGS eine Mullzeile, kann ein Parameter frei gewählt werden, hierbei kann man diesen z.B. durch $\lambda$ ersetzen.\
Da i.A. $\lambda \neq 0$ gewählt werden darf, sind die Vektoren linear abhängig, d.h. die liegen in einer Ebene,\\
$->$ Das von den Vektoren Aufgespannte Spat hat ein Volumen = 0, d.h. die Determinante würde hierbei 0 ergeben.\\
\\
Linear abhängig heißt, dass einige Vektoren durch andere desselben Satzes ausdrückbar sind.\\
\begin{equation*}
\left(\begin{array}{c}
    0 \\ i \\ -i
\end{array}\right)=-i\left[\left(\begin{array}{c}
    -1 \\ 0 \\ 1
\end{array}\right)+\left(\begin{array}{c}
    1 \\ -1 \\ 0
\end{array}\right)\right]
\end{equation*}
Def.: Die Maximalzahl linearunabhängiger Vektoren in der Menge V heißt Dimension des Vektorraums.\\
\\
Beispiel:\\
\begin{center}
    \begin{tabular}{c c} 
     \hline
     Vorstellung & Dimension \\ 
     \hline
     Punkt & 0\\ 
     Gerade & 1 \\ 
     Ebene & 2\\
     Raum & 3\\
     \hline
    \end{tabular}
    \end{center}
$->$ Die Vektoren ${\vec{o},\vec{a}}$ sind immer linear abhängig.\\
$->$ m $>$ n n-dimensionale Vektore sind immer linear abhängig.\\
Def.: Eine Menge B, die die Maximalzahl lineat unabhängiger Vektoren in der Menge V enthält, heißt Basis des Vekotrraumes V. Jeder Vektor $\vec{a_i}$ Element in V lässt sich als Linearkombination der Basisvektoren $\vec{b_j}$ Element in B c V. darstellen:\\
\begin{equation*}
    \vec{a_i}=\sum_{k=1}^{n}\alpha_{ki} \vec{b_k} = \left(\alpha_{1i},\alpha_{2i},\alpha_{3i},\dots,\alpha_{ni}\right)
\end{equation*}
Dies ist die Komponentendarstellung\\
\\
Anmerkung: Im allgemeinen gibt es mehr als nur eine mögliche Basis.\\Aber: Die Anzahl der linear unabhängigen Basisvektoren ist immer gleich der Dimensionalität.\\
\\
\\
Beispiel: 3D-Raum\\
Basis:\\
\begin{equation*}
    i = \left(\begin{array}{c}
        1 \\ 0 \\ 0
    \end{array}\right), j = \left(\begin{array}{c}
        0 \\ 1 \\ 0
    \end{array}\right), k = \left(\begin{array}{c}
        0 \\ 0 \\ 1
    \end{array}\right)
\end{equation*}
Beliebiger Vektor darstellbar durch
\begin{equation*}
    \vec{a_i}=\sum_{k=1}^{n}\alpha_{ki}\vec{b_k}
\end{equation*}
Zum Beispiel:
\begin{equation*}
    \left(\begin{array}{c}
        3 \\ 1 \\ -1
    \end{array}\right) = 3i + j - k
\end{equation*}
Diese abkürzende Schreibweise gibt nur die Koeffizienten $\alpha_{ki}$ an, nicht die Basisvektoren. Sie macht aber ohne Festlegung der Basisvektoren keinen Sinn, denn in anderer Basis sind die Koeffizienten im Allgemeinen anders.\\
\\
Neue Basis (ausgedrückt in alter Basis i, j, k)
\begin{equation*}
    \vec{l}=\left(\begin{array}{c}
        1 \\ 1 \\ 0
    \end{array}\right)_{ijk} = i + j 
\end{equation*}
\begin{equation*}
    \vec{m}=\left(\begin{array}{c}
        1 \\ -1 \\ 0
    \end{array}\right)_{ijk} = i - j
\end{equation*}
\begin{equation*}
    \vec{n}=\left(\begin{array}{c}
        0 \\ 0 \\ -1
    \end{array}\right)_{ijk} = -k
\end{equation*}
Selber Vektor von oben in neuer Basis:
\begin{equation*}
    \left(\begin{array}{c}
        3 \\ 1 \\ -1
    \end{array}\right)_{ijk} = 2 \left(\begin{array}{c}
        1 \\ 1 \\ 0
    \end{array}\right)_{ijk} + \left(\begin{array}{c}
        1 \\ -1 \\ 0
    \end{array}\right)_{ijk} + \left(\begin{array}{c}
        0 \\ 0 \\ -1
    \end{array}\right)_{ijk}
\end{equation*}
\begin{equation*}
    \left(\begin{array}{c}
        3 \\ 1 \\ -1
    \end{array}\right)_{ijk} = 2 \vec{l} + \vec{m} + \vec{n} = \left(\begin{array}{c}
        2 \\ 1 \\ 1
    \end{array}\right)_{lmn} 
\end{equation*}
Beispiel: $\left\{1,x,x^2,\dots\right\}$ bilden die Basis der unendlich dimensionalen Vektorraums der Polynome.\
\begin{equation*}
    P_5\left(x\right)=7+3x^2+4x^3-x^5
\end{equation*}
ist darstellbar als
\begin{equation*}
    \left(7,0,3,4,0,-1\right)
\end{equation*}
Dies sind die jeweiligen Koeffizienten.\\
Def.: Eine Teilmenge von Vektoren eines Vektorraums $\mathbb{V}$ bildet einen Untervektorraum $\mathbb{U}$, wenn alle Kriterien einer Vektorraumes erfüllt werden.\\
Trivial: $\left\{\vec{o}\right\}$ sowie $\mathbb{V}$ sind Untervektorräume von $\mathbb{V}$.\\
Beispiel:\\
Gerade ist eindimensionaler Unterraum einer Ebene.
Ebene ist zweidimensionaler Unterraum des 3D-Raums.
\subsection{Spezielle lineare Vektorräume}
\subsubsection{Euklidische Vektorräume}
a) Def.: Eine mathematische Operation $\varphi$, die zwei Vektoren eines Vektorraums $\mathbb{V}$ eine reelle Zahl zuordnet, heißt Skalarprodukt, wenn gilt:\\
für $\vec{a},\vec{b},\vec{c} \in \mathbb{V}$\\
\begin{itemize}
    \item[1)] Distributivgesetz
    \item[2)] Kommutativgesetz
    \item[3)] Homogenität $x \in \mathbb{R}$\\\\
    $\varphi\left(x\vec{a},\vec{b}\right)=\varphi\left(\vec{a},x\vec{b}\right)=x\varphi\left(\vec{a},\vec{b}\right)$
    \item positive Definitheit\\\\
    $\varphi\left(\vec{a},\vec{a}\right) > 0$ für $\vec{a} \neq \vec{0}$
\end{itemize}
Beispiel: Skalarprodukt zwischen
\begin{equation*}
    \vec{a}=\left(x_a,y_a\right),\vec{b}=\left(x_b,y_b\right)
\end{equation*}
\begin{equation*}
    => \vec{a}\vec{b}=x_ax_b+y_ay_b
\end{equation*}
Beispiel Skalarprodukt zwischen $|a\rangle =a(x), |b\rangle =b(x)$
\begin{equation*}
    => \varphi\left(a,b\right) = <a|b\rangle =\int a(x)b(x)dx
\end{equation*}
\\
b) Def.: Die Norm (Länge) eines Vektors $\vec{a}\in\mathbb{V}$ ist von der Definition her:
\begin{equation*}
    ||\vec{a}|| = \sqrt{\varphi\left(\vec{a},\vec{a}\right)}=\sqrt{\vec{a}\cdot\vec{a}}
\end{equation*}
Anmerkung: Das ist eine von unendlich vielen möglichen Norm-Definitionen\\
Einheitsvektor, normierter Vektor $\hat{a}: ||\hat{a}|| = 1$
\begin{equation*}
    \hat{a}=\frac{\vec{a}}{||\vec{a}||}
\end{equation*} 
für $\vec{a} \neq \vec{o}$\\
Winkel zwischen zwei Vektoren
\begin{equation*}
    \cos\gamma = \frac{\varphi \left(\vec{a},\vec{b}\right)}{||\vec{a}||\cdot||\vec{b}||}=\frac{\vec{a}\vec{b}}{||\vec{a}||\cdot||\vec{b}||}
\end{equation*}
Zwei Vektoren sind orthogonal, wenn $\varphi\left(\vec{a},\vec{b}\right)=\vec{a}\vec{b}=0$ gilt:\\
Spezialfall $\vec{o}\vec{a}=0$ für alle $\vec{a}$\\
Anmerkung: Bisher war der Betrag des Vektors $\vec{a}\left(x_a,y_a\right)$ gegeben als:\\\\
$|\vec{a}|=\sqrt{x_a^2+y_a^2+z_a^2}$\\\\
Warum als $||\vec{a}||$?\\\\
Verallgemeinerung auf Funktionen erfordert eine Unterscheidung.\\
Betragsbildung:
\begin{equation*}
    f(x)=-x^2, |f|=|-x^2|=x^2; x \in \mathbb{R}
\end{equation*}
Normbildung:
\begin{equation*}
    f(x)=-x^2, ||f||=\left[\int_{x_1}^{x_2} \left(-x^2\right)^2  \,dx \right]^\frac{1}{2}
\end{equation*}
\begin{equation*}
    =\left[\int_{x_1}^{x_2} x^4  \,dx \right]^\frac{1}{2} = \left[\frac{x^5}{5}|_{x_1}^{x_2}\right]^\frac{1}{2}=\mathrm{Zahl}
\end{equation*}
c) Def.: Ein Satz von Basisvektoren $\left\{\vec{e_i}\right\}$ heißt Orthonormalsystem (orthonomiert), wenn gilt:\\
\begin{equation*}
    \varphi\left(\hat{e_i}, \hat{e_j}\right) = \hat{e_i} \hat{e_j} = \delta_{ij} = 1 \left(i = j\right) \mathrm{bzw.} 0 \left(i \neq j\right)
\end{equation*}
Ist die Anzahl der Basisvektoren gleich der DImension des Vektorraums, so geiüt die Basis vollständig.\
(\underline{V}ollständig \underline{O}rtho\underline{N}ormal\underline{S}ystem, VONS)\\
Satz: Sind die $m \leq n$ Vektoren $\vec{e_1},\vec{e_2},\vec{e_3},\dots,\vec{e_m}$ in $\mathbb{V}^n$ paarweise zueinander orthogonal, so sind sie auch linear unabhängig. (Die Untersuchung des Satzes gilt nicht!)\\
Beispiel: Basisvektoren $\left\{\hat{i},\hat{j},\hat{k}\right\}$ des dreidimensionalen Raums: karthesisches Koordinatensystems\\
Jeder Vektor $\vec{a} \in \mathbb{V}$ lässt sich nach einem VONS entwickeln.
\begin{equation*}
    \vec{a}=\sum_{i=1}^{n}a_i\hat{e_i}=\left(a_1,a_2,\dots,a_n\right)
\end{equation*}
$\vec{a}=\left(\begin{array}{c}
    3 \\ 2 \\ 1
\end{array}\right)$ nach $\hat{i},\hat{j},\hat{k}$ entwickeln: $\vec{a}=3\hat{i}+2\hat{j}+\hat{k}$
Komponentendarstellung des Skalarprodukts (in einem Orthonormalsystem)\\
\begin{equation*}
    \vec{a}\cdot\vec{b}=\left(\sum_{i=1}^{n}a_i\hat{e_i}\right)\left(\sum_{j=1}^{n}b_j\hat{e_j}\right)=\sum_{i=1}^{n}\sum_{j=1}^{n}a_ib_j(\hat{e_i}\hat{e_j}) = \sum_{i=1}^{n}a_ib_i
\end{equation*}
Wobei $\hat{e_i}\hat{e_j} = \delta_{ij}$
c) Konsturktion eines ONS aus linear unabhängigen Vektoren\\
Das Gramschmidsche Orthogonalitätsverfahren.\\\\
gegeben: linear unabhängiger, jedoch nicht notwendigerweise normierter und nicht notwendigerweise orthogonale Vektoren $\left\{\vec{a_i}\right\}$.\\\\
gesucht:ONS im von $\left\{\vec{a_i}\right\}$ aufgespannten Vektorraum.\\\\
Vorgehen:\\
\begin{itemize}
    \item[1)] Normierung von $\vec{a_1}: \hat{a_1}=\frac{\vec{a_1}}{||\vec{a_1}||}$
    \item[2)] Setze $i=2$
    \item[3)] Orthogonalisierung von $\vec{a_i}$ auf alle $\vec{a_j}$ mit $j < i$\\\\
                $\vec{a_1}'=\vec{a_i}-\sum_{j=1}^{i-1}\left(\hat{a_j}\vec{a_i}\right)\hat{a_j}$\\\\wobei $\hat{a_j}\vec{a_i}$ die Komponente von $\vec{a_i}$ in Richtung von $\hat{a_j}$ ist
\begin{center}
    Notiz: \textbf{Das Skalarprodukt eines Vektors $\vec{\mathbf{r}}$ mit einem Einheitsvektor $\hat{\mathbf{e}}$ ergibt die Projektion von $\vec{\mathbf{r}}$ auf die Richtung von $\hat{\mathbf{e}}$}
\end{center}
    \item[4)] Normierung von $\vec{a_i}': \hat{a_i}=\frac{\vec{a_i}'}{||\vec{a_i}||}$
    \item[5)] Erhöhe $i$ um 1 für $i < n$ und gehe zurück zu 3), ansonsten fertig, wenn $i=n$ 
\end{itemize}
Beispiel:\\
VONS aus
\begin{equation*}
    \left\{
    \left(\begin{array}{c}
        1 \\ 1 \\ 0
    \end{array}\right) + \left(\begin{array}{c}
        1 \\ 0 \\ 1
    \end{array}\right) + \left(\begin{array}{c}
        0 \\ 1 \\ 1
    \end{array}\right)
    \right\} = \left\{\vec{a_1},\vec{a_2},\vec{a_3}\right\}
\end{equation*}
Lineare Unabhängigkeit, da das Bilden der Determinanten $\neq 0$ als Ergebnis liefert.\\
\begin{itemize}
    \item[1)] Normierung von $\vec{a_1}: \hat{a_1}=\frac{\vec{a_1}}{||\vec{a_1}||}=\frac{1}{\sqrt{2}}\left(\begin{array}{c}1 \\ 1 \\ 0\end{array}\right)$
    \item[2)] Orthogonalisierung von $\vec{a_2}$ auf $\hat{a_1}$:\\\\
    \begin{equation*}
        \left(\vec{a_2}\cdot\hat{a_1}\right) = \frac{1}{\sqrt{2}}\left(\begin{array}{c}1 \\ 1 \\ 0\end{array}\right)\left(\begin{array}{c}1 \\ 0 \\ 1\end{array}\right)=\frac{1}{}\sqrt{2}
    \end{equation*}
    \begin{equation*}
        \left(\vec{a_2}\cdot\hat{a_1}\right)\hat{a_1}=\frac{1}{\sqrt{2}}\cdot\frac{1}{\sqrt{2}}\left(\begin{array}{c}1 \\ 1 \\ 0\end{array}\right)=\frac{1}{2}\left(\begin{array}{c}1 \\ 1 \\ 0\end{array}\right)
    \end{equation*}
    \begin{equation*}
        \vec{a_2}'=\vec{a_2}-\left(\vec{a_2}\hat{a_1}\right)\hat{a_1}=\left(\begin{array}{c}1 \\ 0 \\ 1\end{array}\right)-\frac{1}{2}\left(\begin{array}{c}1 \\ 1 \\ 0\end{array}\right)=\left(\begin{array}{c}\frac{1}{2} \\ -\frac{1}{2} \\ 1\end{array}\right) \approx \left(\begin{array}{c}1 \\ -1 \\ 2\end{array}\right)
    \end{equation*}
    \item[3)] Normierung von $\vec{a_2}'$\\\\$\hat{a_2}=\frac{\vec{a_2}'}{||\vec{a_2}'||}=\frac{1}{\sqrt{6}}\left(\begin{array}{c}1 \\ -1 \\ 2\end{array}\right)$
    \item[4)] Orthogonalisierung von $\vec{a_3}$ auf $\hat{a_1}$ und $\vec{a_2}$\\
        \begin{equation*}
            \vec{a_3}'=\vec{a_3}-\left(\vec{a_3}\hat{a_1}\right)\hat{a_1}-\left(\vec{a_3}\hat{a_2}\right)\hat{a_2}
        \end{equation*}
        \begin{equation*}
            \left(\vec{a_3}\hat{a_1}\right)=\frac{1}{\sqrt{2}}\left(\begin{array}{c}1 \\ 1 \\ 0\end{array}\right)\left(\begin{array}{c}0 \\ 1 \\ 1\end{array}\right)=\frac{1}{\sqrt{2}}
        \end{equation*}
        \begin{equation*}
            \left(\vec{a_3}\hat{a_1}\right)\hat{a_1}=\frac{1}{\sqrt{2}}\frac{1}{\sqrt{2}}\left(\begin{array}{c}1 \\ 1 \\ 0\end{array}\right)=\frac{1}{2}\left(\begin{array}{c}1 \\ 1 \\ 0\end{array}\right)
        \end{equation*}
        \begin{equation*}
            \left(\vec{a_3}\hat{a_2}\right)=\frac{1}{\sqrt{6}}\left(\begin{array}{c}1 \\ -1 \\ 2\end{array}\right)\left(\begin{array}{c}0 \\ 1 \\ 1\end{array}\right)=\frac{1}{\sqrt{6}}\left(-1+2\right)=\frac{1}{\sqrt{6}}
        \end{equation*}
        \begin{equation*}
            \left(\vec{a_3}\hat{a_2}\right)\hat{a_2}=\frac{1}{\sqrt{6}}\frac{1}{\sqrt{6}}\left(\begin{array}{c}1 \\ -1 \\ 2\end{array}\right)=\frac{1}{6}\left(\begin{array}{c}1 \\ -1 \\ 2\end{array}\right)
        \end{equation*}
        \begin{equation*}
            \vec{a_3}'=\left(\begin{array}{c}0 \\ 1 \\ 1\end{array}\right)-\frac{1}{2}\left(\begin{array}{c}1 \\ 1 \\ 0\end{array}\right)-\frac{1}{6}\left(\begin{array}{c}1 \\ -1 \\ 2\end{array}\right)=\left(\begin{array}{c}-\frac{4}{6} \\ \frac{4}{6} \\ \frac{4}{6}\end{array}\right)=\left(\begin{array}{c}1 \\ -1 \\ -1\end{array}\right)
        \end{equation*}
    \item[5)] Normierung von $\vec{a_3}': \hat{a_3}=\frac{\vec{a_3}'}{||\vec{a_3}'||}=\frac{1}{\sqrt{3}}\left(\begin{array}{c}1 \\ -1 \\ -1\end{array}\right)$
    \item[Test:] \begin{equation*}
        \det\left(\hat{a_1},\hat{a_2},\hat{a_3}\right)=\frac{1}{\sqrt{2}}\frac{1}{\sqrt{6}}\frac{1}{\sqrt{3}}\begin{vmatrix} 

            1 & 1  & 1 \\ 
    
            1 & -1 & -1 \\ 
    
            0 &  2 & 1 \\  
    
        \end{vmatrix}=\frac{1}{6}\left(1+2+2+1\right)=1
    \end{equation*} 
    Das Ergebnis eines VONS-Rechtssystemes ist \textbf{\underline{IMMER}} 1!
\end{itemize}
\subsection{Lineare Operatoren}
\subsubsection{Grundlagen}
Def.: Ein Operator $A$ ist eine (Rechen-) Vorschrif, durch welche einem Vektor $|a\rangle $ (einer Funktion $a$) ein Vektor $|b\rangle $ (eine Funktion b) zugeorddnet werden kann.
\begin{equation*}
    A|a\rangle  = |b\rangle 
\end{equation*}
Für einen linearen Operator gilt:
\begin{equation*}
    A\left(|a\rangle  + |b\rangle \right)=A|a\rangle + A|b\rangle 
\end{equation*}
\begin{equation*}
    A\left(\alpha|a\rangle \right) = \alpha \left(A|a\rangle \right)
\end{equation*}
Dies gilt für alle $|a\rangle , |b\rangle  \in \mathbb{V}$ und $\alpha \in \mathbb{R}$\\
Satz: Ist $A, B$ linear, so ist auch $(A+B), (A\cdot B)$ linear\\
Beispiele: Translation, Rotation, Inversion, $\dots$
\end{document}