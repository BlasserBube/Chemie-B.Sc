\documentclass[a4paper, fleqn]{article}
\usepackage{chemfig}
\usepackage{mhchem}
\usepackage{xcolor}
\usepackage{amsmath}
\usepackage{tikz}
\usepackage{chemfig}
\usepackage{amssymb}
\usepackage{hyperref}
\usepackage[
  left=1cm,
  right=1cm,
  top=2cm,
  bottom=2cm,
]{geometry}

\begin{document}
\tableofcontents
\newpage
\section{Motivation und Abgrenzung zur Thermodynamik}
\begin{center}
    Chemische Reaktion\\
    \begin{tabular}{c c}
        Thermodynamik&Kinetik\\
        \hline
        Gibt es diese Reaktion? & Wie schnell findet diese statt?\\
        Lage des Gleichgewichts & Geschwindigkeit der Reaktion\\
        $\downarrow$ & und über welche Zweichenstufen wird das GGW erreicht?\\
        In welche Richtung läuft eine Reation ab? & Neue Variable Zeit $t$\\
        \multicolumn{2}{c}{Verknüpfung über das Massenwirkungsgesetz}\\
    \end{tabular}
\end{center}
Nutzen der Reaktionskinetik:
\begin{itemize}
    \item[I.] Kenntnis über die Dauer einer Reaktion
    \item[II.] Möglichkeit die Reaktionsgeschwindigkeit zu beeinflussen
    \item[III.] Aufklärung von Reaktionsmechanismen 
\end{itemize}

\section{Grundbegriffe}
\subsection{Definition der Reaktionsgeschwindigkeit}
Der Reaktionsfortgang einer allgemeinen chemischen Reaktion\\
\begin{equation*}
    \ce{\lvert\nu_A\rvert A + \lvert\nu_B\rvert B -> \lvert\nu_C\rvert C +\lvert\nu_D\rvert D}
\end{equation*}
Kann eindeutig über die \textcolor{red}{Reaktionslaufzahl} $\xi$ beschrieben werden, es gilt:
\begin{equation*}
    d\xi = \frac{dn_A}{\nu_A} = \frac{dn_B}{\nu_B} = \frac{dn_C}{\nu_C} = \frac{dn_D}{\nu_D}
\end{equation*}
Für Reaktanten (Edukte) negativ, da sie wegreagieren, bzw. deren Stoffmenge abnimmt und für die Produkte positiv.\\
Beziehungsweise mit $c=\frac{n}{V}$:
\begin{equation*}
    \frac{d\xi}{V} = \frac{d[A]}{\nu_A} =\frac{d[B]}{\nu_B}=\frac{d[C]}{\nu_C}=\frac{d[D]}{\nu_D}
\end{equation*}
$\xi$ ist mit der \textcolor{red}{Reaktionsvariablen} $x$ gemäß $x=\frac{\xi}{V}$ verknüpft.\\
Daher gilt:
\begin{equation*}
    dx = \frac{d[A]}{\nu_A}=\frac{d[B]}{\nu_B}=\frac{d[C]}{\nu_C}=\frac{d[D]}{\nu_D}
\end{equation*}
$\xi$ und $x$ ermöglichen es Änderungen von Stoffmengen bzw. Konzentrationen ohne Festlegung auf eine bestimmte Komponente zu formulieren.\\
Die \textcolor{red}{Reaktionsgeschwindigkeit} $v$ entspricht der zeitlichen Änderung der Reaktionslaufzahl:
\begin{equation*}
    v = \frac{1}{V} \frac{d\xi}{dt} = \frac{1}{\nu_A} \frac{d[A]}{dt} = \dots = \frac{1}{\nu_D} \frac{d[D]}{dt}
\end{equation*}
$v$ in [$\mathrm{\frac{mol}{l\cdot s}}$]
\subsection{Formuliereung eines allgemeinen Geschwindigkeitsgesetz}
\textcolor{red}{Elementarreaktionen} laufen in einem Schritt ohne Zwischenstufen ab.\\
Für diese lässt sich das \textcolor{red}{Geschwindigkeitsgesetz} mithilfe eines Produkt Ansatzes formulieren.
\begin{equation*}
    v = k(T)[A]^\alpha[B]^\beta\dots
\end{equation*}
Die Exponenten $\alpha,\beta,\dots$ nennen wir \textcolor{red}{Partialordnung} bezüglich der Reaktanten A,B,$\dots$.\\
Die Summe
\begin{equation*}
    n = \alpha + \beta + \dots
\end{equation*}
heißt \textcolor{red}{Gesamtordnung}\\
\underline{Achtung:} Die Reaktionsordnung und die Partialordnung sind experimentelle Größen. Sie haben in der Regel keinen Bezug zu den stöchiometrischen Koeffizienten. Nur für Elementarreaktionen kann $\alpha = \lvert \nu_A \rvert$, $\beta = \lvert \nu_B \rvert$ usw. angenommen werden.\\
Bsp.
\begin{center}
    \ce{N2O5_{(g)} -> 2NO2_{(g)} + 1/2 O2_{(g)}}\\
        $c = k [\ce{N2O5}],\, n=1$\\
    \ce{NO2_{(g)} -> NO_{(g)} + 1/2 O2_{(g)}}\\
        $v = k[\ce{NO2}]^2 ,\, n=2$
\end{center}
Es gibt auf Reaktionen für die der Begriff "Ordnung" nicht anwendbar ist.\\
Bsp.:
\begin{equation*}
    \ce{H2_{(g)} + Br2_{(g)} <=>[{$k$}][{$k'$}] 2HBr_{(g)}}
\end{equation*}
\begin{equation*}
    v = \frac{k[\ce{H2}][\ce{Br2}]^\frac{3}{2}}{[\ce{Br2}+k'[\ce{HBr}]]}
\end{equation*}
$k(T)$ ist die \textcolor{red}{(Reaktions-)Geschwindigkeitskonstante}\\
$k(T)$ ist
\begin{itemize}
    \item[I.] Temperaturabhängig
    \item[II.] unabhängig von der Konzentration 
    \item[III.] ihre Dimension (Einheit) hängt von der Reaktionsordnung ab.  
\end{itemize}

\subsection{Die Molekularität}
Chemische Reaktionen laufen über mehrere Einzelschritte, sogenannte Elementarreaktionen ab.\\
Die Zahl der Moleküle die an einem Einzelschritt beteiligt sind heißt \textcolor{red}{Molekularität}\\
Bsp.: Gesamtreaktion:\\
\begin{equation*}
    \ce{Hs + Br2 -> 2HBr}
\end{equation*}
Einzelschritte:\\
\begin{equation*}
    \ce{Br2 -> 2 Br}
\end{equation*}
\underline{Ein} Molekül zerfällt $\rightarrow$ unimolekulare Reaktion.\\
\begin{equation*}
    \ce{Br + H2 -> HBr + H}
\end{equation*}
\underline{Zwei} Moleküle stoßen zusammen $\rightarrow$ bimolekulare Reaktion\\
\underline{Achtung:} Molekularität und Reaktionsordnung sind im Allgmeinen nicht identisch. Nur bei Elementarreaktionen stimmen sie überein. Die Reaktionsordnung ist eine experimentelle Größe, die Molekularität eine theoretische Größe.\\
Die Wahrscheinlichkeit, dass mehrere Moleküle gleichzeitig zusammenstoßen nimmt mit deren Anzahl ab.\\
$\hookrightarrow$ Tri- und höhermolekulare Reaktionen äußerst selten.\\
\underline{Hinweis:} Bei Komplexeren Reaktionen wird häufig die Molekularität des geschwindigkeitsbestimmenden Schrittes als Molekularität der Reaktion bezeichnet.\\

\section{Einfache Geschwindigkeitsgesetze}
\subsection{Reaktionen 0.Ordnung}
Reaktionen die unabhängig von der Reaktionskonzentration sind.\\
\underline{Typisches Beispiel:}\\
Katalytische Reaktionen bei denen der Reaktant im Überschuss vorliegt, z.B. Zersetzung \ce{PH3} an einem heißen \ce{W}-Draht bei hohem Druck.\\
Reaktion: \ce{A ->[{k}] P}\\
Geschwindigkeitsgesetz:
\begin{equation*}
    v = \frac{1}{\nu_A} \frac{d[A]}{dt} = k
\end{equation*}
(hier:$\nu_A = -1$)\\
$v: \left[\mathrm{\frac{mol}{l\cdot s}}\right]$\\
$k: \left[\mathrm{\frac{mol}{l\cdot s}}\right]$\\
Zeitlicher Verlauf der Reaktantenkonzentration?\\
$\rightarrow$ Lösung der Differentialgleichung:\\
\begin{itemize}
    \item[1)] Separation der Variablen
    \item[2)] Integration
\end{itemize}
\begin{align*}
    \frac{d[A]}{dt} &= -k_t\\
    \int_{[A(t=0)]}^{[A]} d[A] &= -k \int_{0}^{t}\,dt\\
    \left[[A]\right]_{[A(t=0)]}^{[A]} &= -k[t]_0^t\\
    [A]-[A(t=0)] &= -k(t-0)\\
\end{align*}
\textcolor{red}{Integriertes Geschwindigkeitsgesetz:}\\
\begin{equation*}
    [A] = [A(t=0)]-kt
\end{equation*}\\
Die Halbwertszeit $t_\frac{1}{2}$ ist häufig eine nützliche Größe, sie entspricht der Zeit nach der gerade die Hälfte der Ausgangskonzentration umgesetzt wurde.\\
Für $t_\frac{1}{2}$ gilt: $[A]=\frac{[A(t=0)]}{2}$\\
Setzt man dies in das integrierte Geschwindigkeitsgesetz ein, so erhält man:\\
\begin{align*}
    \frac{[A(t=0)]}{2} &= [A(t=0)]-kt_\frac{1}{2}\\
    f_\frac{1}{2} &= \frac{[A(t=0)]}{2k}
\end{align*}
\subsection{Reaktionen 1.Ordnung}
Linearer Zusammenhang zwischen Reaktantenkonzentration und Reaktionsgeschwindigkeit.\\
\underline{Typisches Beispiel:} Radioaktiver Zerfall.\\
Reaktion: \ce{A ->[{k}] P}\\
\begin{align*}
    v &= -\frac{d[A]}{dt}\\
    &= k[A]\\
\end{align*}
Integration:
\begin{align*}
    \int_{[A(t=0)]}^{[A]}\frac{d[A]}{[A]} &= -k\int_{0}^{t}\,dt\\
    \left[\ln [A]\right]_{[A(t=0)]}^{[A]} &= -k[t]_0^t\\
    \ln [A] - \ln [A(t=0)] &= -k(t-0)\\
    \text{Integriertes geschwindigkeitsgesetz:}\\
    \ln \frac{[A]}{[A(t=0)]} &= -kt\\
    [A] &= [A(t=0)]e^{-kt}
\end{align*}
Halbwertszeit $t_\frac{1}{2}$ durch Einsetzen von $[A] = \frac{[A(t=0)]}{2}$ in das integrierte Geschwindigkeitsgesetz bei Reaktionen 1. Ordnung, erhält man:\\
\begin{align*}
    \frac{[A(t=0)]}{2} &= [A(t=0)]e^{-kt_\frac{1}{2}}\\
    \ln [A(t=0)] - \ln 2 &= \ln [A(t=0)] - kt_\frac{1}{2}\\
    t_\frac{1}{2} &= \frac{\ln 2}{k}
\end{align*}
Die Halbwertszeit von Reaktionen 1.Ordnung ist unabhängig von der Anfangskonzentration.\\
Wie verändert sich die Produkt konzentration mit der Zeit?\\
\ce{A ->[{k}] P}\\
Für $t=0,\, [P]=0,\, [A]=[A(t=0)]$\\
Für $t=t,\, [A] = [A(t=0)]-[P]$\\
\begin{equation*}
    v = \frac{1}{\nu_A}\frac{d[A]}{dt} = \frac{1}{\nu_p}\frac{d[P]}{dt} = k[A]
\end{equation*}
\begin{align*}
    \frac{d[P]}{dt} &= k([A(t=0)]-[P]) \text{ mit } \nu_P = 1\\
    \int_{0}^{[P]} \frac{d[P]}{[A(t=0)]-[P]} &= k \int_{0}^{t} dt\\
    \left[-\ln \left([A(t=0)]-[P]\right)\right]_0^{[P]} &= k[t]_0^t\\
    -\ln \left([A(t=0)]-[P]\right) + \ln ([A(t=0)]-[0]) &= k(t-0)\\
    \ln \frac{[A(t=0)]}{[A(t=0)]-[P]} &= kt\\
    \frac{[A(t=0)]}{[A(t=0)]-[P]} &= e^{kt}\\
    [A(t=0)]e^{-kt} &= [A(t=0)]-[P]\\
    [P] &= [A(t=0)]\left(1-e^{-kt}\right)
\end{align*}

\subsection{Reaktion 2.Ordnung}
\subsubsection{Variante 1}
\ce{2A ->[{k}] P}   $\nu_A = -2$\\
Geschwindigkeitsgesetz:\\
\begin{align*}
    -\frac{1}{2}\frac{d[A]}{dt} &= k[A]^2\\
    \frac{d[A]}{dt} &= -2 k [A]^2
\end{align*}
$k: [\mathrm{\frac{l}{mol\cdot s}}]$\\
Integration:
\begin{align*}
    \int_{[A(t=0)]}^{[A]} \frac{d[A]}{[A]^2} &= - 2k \int_{0}^{t} \, dt\\
    \left[-\frac{1}{[A]}\right]^{[A]}_{[A(t=0)]} &= -2k[t]_0^t\\
    \text{Integriertes Geschwindigkeitsgesetz:}\\
    \frac{1}{[A]} - \frac{1}{[A(t=0)]} &= 2kt\\
\end{align*}
Umformen nach $[A]$:\\
\begin{align*}
    \frac{1}{[A]} &= 2kt + \frac{1}{[A(t=0)]}\\
    &= \frac{2kt[A(t=0)]+1}{[A(t=0)]}\\
    [A]&=\frac{[A(t=0)]}{1+2kt[A(t=0)]}
\end{align*}
Nach der Halbwertszeit:\\
\begin{align*}
    \frac{1}{\frac{[A(t=0)]}{2}} - \frac{1}{[A(t=0)]} &= 2kt_{\frac{1}{2}}\\
    t_\frac{1}{2} &= \frac{1}{2k[A(t=0)]}
\end{align*}
Umgekehrt proportional zur Anfangskonzentration\\
Bsp.:\\
\ce{2I -> I2}\\
\ce{2NOBr -> 2NO + Br2}

\subsubsection*{Variante 2}
\ce{A+B -> P}\\
Geschwindigkeitsgesetz:\\
\begin{align*}
    \frac{d[A]}{dt} &= - k[A][B]
\end{align*}
\textbf{Variante 2a:}\\
\begin{align*}
    [A(t=0)] &= [B(t=0)]\\
\end{align*}
daraus folgt $[A]=[B]$ zu jedem Zeitpunkt $t$\\
\begin{align*}
    \Rightarrow \frac{d[A]}{dt} &= -k[A]^2
\end{align*}
integriertes Geschwindigkeitsgesetz:\\
\begin{align*}
    \frac{1}{[A]}-\frac{1}{[A(t=0)]} &= kt\\
    [A] &= \frac{[A(t=0)]}{1+[A(t=0)]kt}
\end{align*}
Halbwertszeit:\\
\begin{align*}
    t_\frac{1}{2} &= \frac{1}{k[A(t=0)]}
\end{align*}\\

\textbf{Merke:}\\
Für eine Reaktion \ce{A -> P} $n$-ter Ordnung gilt:\\
\begin{align*}
    t_\frac{1}{2} \tilde k^{-1}[A(t=0)]^{-(n-1)}
\end{align*}\\

\textbf{Variante 2b:}\\
\begin{equation*}
    [A(t=0)] \neq [B(t=0)]
\end{equation*}
Geschwindigkeitsgesetz:\\
\begin{align*}
    \frac{d[A]}{dt} &= -k[A][B]
\end{align*}
Unter zur Hilfenahme der Reaktionsvariable $x$\\
\begin{align*}
    [A] &= [A(t=0)]-x\\
    &\text{und}\\
    [B] &= [B(t=0)]-x\\
    &\text{Mit}\\
    \frac{dx}{dt} &= \frac{1}{\gamma_A}\frac{d[A]}{dt}\\
    \frac{dx}{dt} &= k([A(t=0)]-x)([B(t=0)]-x)\\
    \int_{0}^{x} ([A(t=0)]-x)([B(t=0)]-x)\\
    \frac{1}{[A(t=0)]-[B(t=0)]}\left(\int_{0}^{x}\frac{dx}{[B(t=0)-x]}-\int_{0}^{x} \frac{dx}{[A(t=0)]-x}\right)\\
    = \frac{1}{[A(t=0)]-[B(t=0)]}\left(-\ln ([B(t=0)]-x)+\ln[B(t=0)]+\ln([A(t=0)]-x)-\ln[A(t=0)]\right)\\
    \Rightarrow \frac{1}{[A(t=0)]-[B(t=0)]} \ln \left(\frac{[B(t=0)]([A(t=0)]-x)}{([B(t=0)]-x)[A(t=0)]}\right) &= kt
\end{align*}
Bsp.:\\
\ce{H2 + I2 -> 2HI}\\
\ce{H^{+} + OH^{-} -> H2O}\\
\underline{Sonderfall:} Reaktion pseudo erster Ordnung\\
\underline{Bsp:} Rohrzuckerinversion\\
\ce{\text{Rohrzucker} + H2O -> \text{Fructose} + \text{Glucose}}\\
Geschwindigkeitsgesetz:\\
\begin{align*}
    \frac{d[RZ]}{dt} &= -k[RZ][\ce{H2O}]
\end{align*}\\
Da \ce{H2O} al Lösungsmittel im großen Überschuss vorliegt, ist dessen Konzentration zeitlich nahezu konstant und kann in die Geschwindigkeitskonstante mit einbezogen werden\\
\begin{equation*}
    \Rightarrow \frac{d[RZ]}{dt} = - k'[RZ]
\end{equation*}

\subsection{Experimentelle Untersuchungsmethoden Prinzipielle Voraussetzungen}
0. Messung einer Größe $\lambda_i$ die direkt mit der Konzentration $[i]$ einer bestimmten Reaktionskomponente $i$ korreliert.\\
1. Zeitskala der Messung muss kurz ein im Vergleich zur Reaktionsgeschwindigkeit\\
Faustregel: $t_{Mess} < t_\frac{1}{2}$\\
2. Beginn der Reaktion muss klar definiert sein\\\\

\subsubsection{Unterscheidung in zwei grundlegende Messmethoden}
a) \underline{Diskontinuierliche Methode (ex-gita)}\\
Probe wird aus laufender Reaktion entnommen und shließend analysiert\\
\underline{Vorteil:} direkte Konzentrationsbestimmung\\
\underline{Nachteil:} Reaktion schreitet zwischen Probennahme und Analyse weiter fort $\rightarrow$ Methode ist nur für langsame Reaktionen geeignet.\\
\underline{Möglichkeiten um den Reaktionsfortschritt zu hemmen:}\\
\begin{itemize}
    \item Absenken der Temperatur ("Einfrieren")
    \item Beseitigung eines Reaktionspartners durch chem. Reaktion bzw. Inhibitor.
    \item Starkes verdünnen (Quenchen)
\end{itemize}
\underline{Bsp.:}\\
Chemische Methoden:\\
Gravimetrie\\
Titrimetrie\\
Instrumentelle Methoden:\\
Massenspektrometrie\\
Gaschromatographie\\

b)\underline{Kontinuierliche Methode (in-situ / in-operando)}\\
Konzentration wird unmittelbar am laufenden Reaktionssystem gemessen.\\
\underline{Vorteil:}\\
\begin{itemize}
    \item Keine Probenentnahme - auch schnellere Reaktionen können untersucht werden
\end{itemize}
\underline{Nachteil:}\\
\begin{itemize}
    \item indirekte Konzentrationsbestimmung
    \item Nebenprodukte können stören 
    \item[$\hookrightarrow$] Abhilfe durch Kombination mehrerer
\end{itemize}
\underline{Bsp:}\\
Klassische Methoden: Druckmessung, Volumenmessung, Polarimatrie, Refraktometrie, Kaloremetrie, Leitfähigkeit\\
Spektroskopische Methoden: IR/Raman-Spektroskopie, MMR-Spektroskopie, $\dots$\\
Praktisches Vorgehen für den Fall, dass die Konzentration nicht unmittelbar aus der gemessenen Größe $\lambda$ extrahiert werden kann, \underline{am Beispiel:}\\
\begin{equation*}
    \ce{A+B -> C+D} \text{ SIEHE FOLIE}
\end{equation*}\\
\underline{Initierung der Reaktion durch:}\\
\begin{itemize}
    \item[a)] Durchmischung $\rightarrow$ Strömungsmethode\\ Stopped-Flow-Methode (nur für vergleichsweise langsame Reaktionen geeignet)
    \item[b)] Relaxationsmethode $\rightarrow$ Temperatur- oder Drucksprung
    \item[c)] Photolyse $\rightarrow$ Blitzlichtphotolyse, Laser-Puls
    \item[d)] Puls-Radiolyse
    \item[] Details siehe Folie
\end{itemize}

\subsection{Bestimmung der Reaktionsordnung}
Kenntnis der Reaktionsordnung ermögicht Rückschlüsse auf den Reaktionsmechanismus
\subsubsection{Integrationsmethode}
Gemessene Reaktandenkonzentration wird in geeigeneter Weise ($[A], \ln [A], [A]^{-1}$) gemäß der integrierten Geschwindigkeitsgesetze über die Zeit $t$ aufgetragen.\\
0.Ordnung: $[A] = [A(t=0)]-kt$\\
1.Ordnung: $\ln \frac{[A]}{[(t=0)]} = -kt$\\
2. Ordnung: $\frac{1}{[A]} = \frac{1}{[A(t=0)]+kt}$\\

\subsubsection{Halbwertszeitmethode}
Messung der Halbwertszeit bei Variation der Anfangskonzentration $[A(t=0)]$\\
$\Rightarrow t_\frac{1}{2}$ verändert sich proportional zu $[A(t=0)]$\\
$\hookrightarrow$ Reaktion 0. Ordnung ($t_\frac{1}{2} = \frac{[A(t=0)]}{2k}$)\\
$t_\frac{1}{2}$ ist unabhängig von $[A(t=0)]$\\
$\hookrightarrow$ Reaktion 1. Ordnung ($t_\frac{1}{2}=\frac{\ln 2}{k}$)\\
$t_\frac{1}{2}$ verhält sich umgekehrt proportional zu $[A(t=0)]$\\
$\hookrightarrow$ Reaktion 2. ORdnung ($t_\frac{1}{2} = \frac{1}{k[A(t=0)]}$)\\
allgemein: für eine Reaktion $n$-ter Ordnung:\\
\begin{equation*}
    t_\frac{1}{2} ~ [A(t=0)^{-(n-1)}]
\end{equation*}

\subsubsection{Isolationsmethode}
Für Reaktionen mit mehreren Reaktanden, Beispiel:\\
\begin{equation*}
    \ce{A+B -> C+D}
\end{equation*}
\begin{align*}
    \frac{d[A]}{dt} &= -k[A]^\alpha[B]^\beta
\end{align*}
$\rightarrow [B]$ wird in großem Überschuss zugegeben, sodass sich dessen Konzentration während der Reaktion (zumindest zu Beginn) quasi nicht ändert.\\
Geschwindigkeitsgesetz vereinfacht sich zu:\\
\begin{equation*}
    \frac{d[A]}{dt} = -k'[A]^\alpha
\end{equation*}
(vgl. Reaktion pseudo-erster Ordnung)\\
$\rightarrow$ Bestimmung der Partialordnung $\alpha$ mit der Integrations- oder der Halbwertszeitmethode.\\
Anschließend das gleiche für $[B]$ zur Bestimmung von $\beta$.
\subsubsection{Methode der Anfangsgeschwindigkeit}
Reaktionsgeschwidigkeit wird nur zu Beginn der Reaktion gemessen wenn noch kaum Reaktand verbraucht wurde.\\
Beispielreaktion:\\
\begin{equation*}
    \ce{A+B -> C+D}, r = -k[A]^\alpha[B]^\beta
\end{equation*}
$[A(t=0)]$ wird variiert, Veränderung von B wird vernachlässgt\\
$\hookrightarrow\, \frac{d[A]}{t} = -k*[A]^\alpha$\\
\begin{equation*}
    \log \frac{d[A]}{dt} = -\log k' - a \log [A]
\end{equation*} 
\subsection{Temperaturabhängigkeit der Reaktionsgeschwindigkeitskonstante}
Arrhenius-Gleichung:
\begin{equation*}
    k(T) = A\cdot e^{-\frac{E_A}{RT}}
\end{equation*}
Die Gleochung wurde empirisch abgeleitet, kann inzwischen aber im Rahmen der Stoßtheorie und/oder der Theorie des Übergangszustandes erklärt werden ($\rightarrow$ Statistische Thermodynamik)\\
$A$ ist der präexponentieller Faktor / Frequenzfaktor:\\
Er hängt von der Stoßhäufigkeit und Orientierung der Reaktanten ab ($\rightarrow$ entropisch)\\
$E_A$ ist die Aktivierungsenergie\\
Mindestenergie die 1 mol Teilchen für die Reaktion benötigen\\
$e^{-\frac{E_A}{RT}}$ ist der Bruchteil der Teilchen, die bei dieser Temperatur die benötigte Aktivierungsenergie besitzen (Boltzmannverteilung)\\\\
\underline{Bestimmung der Arrheniusparameter ($E_A, A$)}\\
Bestimmung der Geschwindigkeitskonstante k bei verschiedenen Temperaturen $T$:\\
\begin{equation*}
    \ln k = \ln A - \frac{E_A}{RT}
\end{equation*}
Abweichung vom "Arrhenius-Verhalten" deutet auf kompliziertere Reaktionsmechanismen hin (siehe Folien)\\
\section{Kinetik Komplexer Reaktionen}
Komplexe Reaktionen sind aus mehreren Elementarreaktionen zusammengesetzt $\rightarrow$ Reaktionsmechanismen\\
\underline{Grundtypen:}
\begin{itemize}
    \item Reversible Reaktion
    \item Parallelreaktionen
    \item Folgereaktionen
\end{itemize}
\underline{Kombination}
\begin{itemize}
    \item Folgereaktion mit reversiblem Teilschritt\\ vorgelagertes GGW \ce{A <-> B -> C}\\nachgelagertes GGW \ce{A -> B <-> C}
    \item Geschlossene Folgereaktion\\\ce{A+B -> C+D}\\\ce{D+E -> A+C}\\Somit Bruttoreaktion: \ce{B+E -> 2 C}
    \item Folgereaktion mit Parallelreaktion
    \item und viele mehr
\end{itemize}
\subsection{Reversible Reaktionen}
Im Allgemeinen laufen chemische Reaktionen nicht vollständig ab, sondern es stellt sich ein Gleichgewicht ein:
\begin{equation*}
    \ce{A + B <->[{k}][{k.}] C + D}
\end{equation*}
Hinreaktion:
\begin{equation*}
    \ce{A + B ->[{k}] C + D}
\end{equation*}
\begin{align*}
    r &= \left(\frac{dx}{dt}\right)\\
    &= k[A][B]\\
    &= k([A(t=0)]+\gamma_Ax)([B(t=0)]\gamma_Bx)
\end{align*}
Rückreaktion:
\begin{equation*}
    \ce{C + D ->[{k.}] A + B}
\end{equation*}
\begin{align*}
    r &= k.[C][D]\\
    &= k.(\gamma_Cx)(\gamma_Dx)
\end{align*}
Gesamtreaktion:
\begin{align*}
    r &= k[A][B]-k.[C][D]\\
    r_{hin} &= r_{ruck}
\end{align*}
Prinzip der mikroskopischen Reversiabilität
\begin{align*}
    k[A_{GGW}][B_{GGW}]&= k.[C_{GGW}][D_{GGW}]\\
    \frac{[A_{GGW}][B_{GGW}]}{[C_{GGW}][D_{GGW}]} &= \frac{k}{k.} = K
\end{align*}
Achtung: kinetische Herleitung des Massenwirkungsgesetzes gilt nur Nährungsweise, da K über die Aktivität definiert ist.\\\\
\underline{Konhzentrationsverlauf einer reversiblen Reaktion}\\
\ce{A <->[{k}][{k.}] B}\\
Annahme: $[B]_0 = 0$\\
Alle Tielschritte sind 1. Ordnung.\\
\begin{align*}
    r &= \frac{-d[A]}{dt}\\
    &= k[A]-k.[B]\\
    [B] = [A]_0 - [A]\\
    &= k[A]-k.([A]_0-[A])\\
    &= (k1+k.)[A]-k.[A]_0
\end{align*}
Im Gleichgewicht: $t \rightarrow \infty$\\
\begin{align*}
    r &= 0\\
    [A_{GGW}]\\
    0 &= (k1+k.)[A]-k.[A]_0\\
    k[A&]_0=(k+k.)[A_{GGW}]\\
\end{align*}
\begin{align*}
    -\frac{d[A]}{dt} &= (k+k.)[A]-(k+k.)[A_{GGW}]\\
    - \int_{[A]_0}^{[A]}\,\frac{d[A]}{[A]-[A_{GGW}]} &= (k+k.) \int_{0}^{t}\,dt\\
    \ln \left(\frac{[A]_0 - [A_{GGW}]}{[A]-[A_{GGW}]}\right) &= (k+k.)t
\end{align*}
Falls $\mathbb{K}$ bekannt sind über $K = \frac{k}{k.}$ auch die einzelnen Geschwindigkeitskonstante zugängiglich, nicht nur deren Summe.\\
Konzentrationsverlauf:
\begin{align*}
    [A] &= [A_{GGW}] + \left([A]_0-[A_{GGW}]\right)e^{-(k+k.)t}\\
    [B] &= [B_{GGW}] + \left([A]_0-[A_{GGW}]\right)\left(1-e^{-(k+k.)t}\right)\\
\end{align*}

\subsection{Parallelreaktionen}
Beispiel: Spaltung von Formaldehyd\\
Annahme:
\begin{itemize}
    \item Alle Teilschritte 1. Ordnung
    \item Zu Beginn liegt nur A vor
\end{itemize}
Zerfall Stoff A:
\begin{align*}
    \frac{d[A]}{dt} &= -k[A]-k.[A]\\
    &= -(k+k.)[A]\\
    &= -k_{eff} [A]
\end{align*}
Wie Reaktion 1. Ordnung\\
integriertes Geschwindigkeitsgesetz:
\begin{equation*}
    [A] = [A]_0 e^{-(k1+k.)t}
\end{equation*}
Bildung von B:
\begin{equation*}
    \frac{d[B]}{dt} = k[A]
\end{equation*}
\begin{align*}
    \frac{d[B]}{dt} &= k[A]_0 e^{-(k+k.)t}\\
    \int_{0}^{[B]}\,d[B] &= k[A]_0 \int_{0}^{t} e^{-(k+k.)t}\,dt\\
\end{align*}
\end{document}