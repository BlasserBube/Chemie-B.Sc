\documentclass[a4paper, fleqn]{article}
\usepackage{chemfig}
\usepackage{mhchem}
\usepackage{xcolor}
\usepackage{amsmath}
\usepackage{tikz}
\usepackage{chemfig}
\usepackage{amssymb}
\usepackage{hyperref}
\usepackage[
  left=1cm,
  right=1cm,
  top=2cm,
  bottom=2cm,
]{geometry}

\begin{document}
\tableofcontents
\newpage
\section{Motivation und Abgrenzung zur Thermodynamik}
\begin{center}
    Chemische Reaktion\\
    \begin{tabular}{c c}
        Thermodynamik&Kinetik\\
        \hline
        Gibt es diese Reaktion? & Wie schnell findet diese statt?\\
        Lage des Gleichgewichts & Geschwindigkeit der Reaktion\\
        $\downarrow$ & und über welche Zweichenstufen wird das GGW erreicht?\\
        In welche Richtung läuft eine Reation ab? & Neue Variable Zeit $t$\\
        \multicolumn{2}{c}{Verknüpfung über das Massenwirkungsgesetz}\\
    \end{tabular}
\end{center}
Nutzen der Reaktionskinetik:
\begin{itemize}
    \item[I.] Kenntnis über die Dauer einer Reaktion
    \item[II.] Möglichkeit die Reaktionsgeschwindigkeit zu beeinflussen
    \item[III.] Aufklärung von Reaktionsmechanismen 
\end{itemize}

\section{Grundbegriffe}
\subsection{Definition der Reaktionsgeschwindigkeit}
Der Reaktionsfortgang einer allgemeinen chemischen Reaktion\\
\begin{equation*}
    \ce{\lvert\nu_A\rvert A + \lvert\nu_B\rvert B -> \lvert\nu_C\rvert C +\lvert\nu_D\rvert D}
\end{equation*}
Kann eindeutig über die \textcolor{red}{Reaktionslaufzahl} $\xi$ beschrieben werden, es gilt:
\begin{equation*}
    d\xi = \frac{dn_A}{\nu_A} = \frac{dn_B}{\nu_B} = \frac{dn_C}{\nu_C} = \frac{dn_D}{\nu_D}
\end{equation*}
Für Reaktanten (Edukte) negativ, da sie wegreagieren, bzw. deren Stoffmenge abnimmt und für die Produkte positiv.\\
Beziehungsweise mit $c=\frac{n}{V}$:
\begin{equation*}
    \frac{d\xi}{V} = \frac{d[A]}{\nu_A} =\frac{d[B]}{\nu_B}=\frac{d[C]}{\nu_C}=\frac{d[D]}{\nu_D}
\end{equation*}
$\xi$ ist mit der \textcolor{red}{Reaktionsvariablen} $x$ gemäß $x=\frac{\xi}{V}$ verknüpft.\\
Daher gilt:
\begin{equation*}
    dx = \frac{d[A]}{\nu_A}=\frac{d[B]}{\nu_B}=\frac{d[C]}{\nu_C}=\frac{d[D]}{\nu_D}
\end{equation*}
$\xi$ und $x$ ermöglichen es Änderungen von Stoffmengen bzw. Konzentrationen ohne Festlegung auf eine bestimmte Komponente zu formulieren.\\
Die \textcolor{red}{Reaktionsgeschwindigkeit} $v$ entspricht der zeitlichen Änderung der Reaktionslaufzahl:
\begin{equation*}
    v = \frac{1}{V} \frac{d\xi}{dt} = \frac{1}{\nu_A} \frac{d[A]}{dt} = \dots = \frac{1}{\nu_D} \frac{d[D]}{dt}
\end{equation*}
$v$ in [$\mathrm{\frac{mol}{l\cdot s}}$]
\subsection{Formuliereung eines allgemeinen Geschwindigkeitsgesetz}
\textcolor{red}{Elementarreaktionen} laufen in einem Schritt ohne Zwischenstufen ab.\\
Für diese lässt sich das \textcolor{red}{Geschwindigkeitsgesetz} mithilfe eines Produkt Ansatzes formulieren.
\begin{equation*}
    v = k(T)[A]^\alpha[B]^\beta\dots
\end{equation*}
Die Exponenten $\alpha,\beta,\dots$ nennen wir \textcolor{red}{Partialordnung} bezüglich der Reaktanten A,B,$\dots$.\\
Die Summe
\begin{equation*}
    n = \alpha + \beta + \dots
\end{equation*}
heißt \textcolor{red}{Gesamtordnung}\\
\underline{Achtung:} Die Reaktionsordnung und die Partialordnung sind experimentelle Größen. Sie haben in der Regel keinen Bezug zu den stöchiometrischen Koeffizienten. Nur für Elementarreaktionen kann $\alpha = \lvert \nu_A \rvert$, $\beta = \lvert \nu_B \rvert$ usw. angenommen werden.\\
Bsp.
\begin{center}
    \ce{N2O5_{(g)} -> 2NO2_{(g)} + 1/2 O2_{(g)}}\\
        $c = k [\ce{N2O5}],\, n=1$\\
    \ce{NO2_{(g)} -> NO_{(g)} + 1/2 O2_{(g)}}\\
        $v = k[\ce{NO2}]^2 ,\, n=2$
\end{center}
Es gibt auf Reaktionen für die der Begriff "Ordnung" nicht anwendbar ist.\\
Bsp.:
\begin{equation*}
    \ce{H2_{(g)} + Br2_{(g)} <=>[{$k$}][{$k'$}] 2HBr_{(g)}}
\end{equation*}
\begin{equation*}
    v = \frac{k[\ce{H2}][\ce{Br2}]^\frac{3}{2}}{[\ce{Br2}+k'[\ce{HBr}]]}
\end{equation*}
$k(T)$ ist die \textcolor{red}{(Reaktions-)Geschwindigkeitskonstante}\\
$k(T)$ ist
\begin{itemize}
    \item[I.] Temperaturabhängig
    \item[II.] unabhängig von der Konzentration 
    \item[III.] ihre Dimension (Einheit) hängt von der Reaktionsordnung ab.  
\end{itemize}

\subsection{Die Molekularität}
Chemische Reaktionen laufen über mehrere Einzelschritte, sogenannte Elementarreaktionen ab.\\
Die Zahl der Moleküle die an einem Einzelschritt beteiligt sind heißt \textcolor{red}{Molekularität}\\
Bsp.: Gesamtreaktion:\\
\begin{equation*}
    \ce{Hs + Br2 -> 2HBr}
\end{equation*}
Einzelschritte:\\
\begin{equation*}
    \ce{Br2 -> 2 Br}
\end{equation*}
\underline{Ein} Molekül zerfällt $\rightarrow$ unimolekulare Reaktion.\\
\begin{equation*}
    \ce{Br + H2 -> HBr + H}
\end{equation*}
\underline{Zwei} Moleküle stoßen zusammen $\rightarrow$ bimolekulare Reaktion\\
\underline{Achtung:} Molekularität und Reaktionsordnung sind im Allgmeinen nicht identisch. Nur bei Elementarreaktionen stimmen sie überein. Die Reaktionsordnung ist eine experimentelle Größe, die Molekularität eine theoretische Größe.\\
Die Wahrscheinlichkeit, dass mehrere Moleküle gleichzeitig zusammenstoßen nimmt mit deren Anzahl ab.\\
$\hookrightarrow$ Tri- und höhermolekulare Reaktionen äußerst selten.\\
\underline{Hinweis:} Bei Komplexeren Reaktionen wird häufig die Molekularität des geschwindigkeitsbestimmenden Schrittes als Molekularität der Reaktion bezeichnet.\\

\section{Einfache Geschwindigkeitsgesetze}
\subsection{Reaktionen 0.Ordnung}
Reaktionen die unabhängig von der Reaktionskonzentration sind.\\
\underline{Typisches Beispiel:}\\
Katalytische Reaktionen bei denen der Reaktant im Überschuss vorliegt, z.B. Zersetzung \ce{PH3} an einem heißen \ce{W}-Draht bei hohem Druck.\\
Reaktion: \ce{A ->[{k}] P}\\
Geschwindigkeitsgesetz:
\begin{equation*}
    v = \frac{1}{\nu_A} \frac{d[A]}{dt} = k
\end{equation*}
(hier:$\nu_A = -1$)\\
$v: \left[\mathrm{\frac{mol}{l\cdot s}}\right]$\\
$k: \left[\mathrm{\frac{mol}{l\cdot s}}\right]$\\
Zeitlicher Verlauf der Reaktantenkonzentration?\\
$\rightarrow$ Lösung der Differentialgleichung:\\
\begin{itemize}
    \item[1)] Separation der Variablen
    \item[2)] Integration
\end{itemize}
\begin{align*}
    \frac{d[A]}{dt} &= -k_t\\
    \int_{[A(t=0)]}^{[A]} d[A] &= -k \int_{0}^{t}\,dt\\
    \left[[A]\right]_{[A(t=0)]}^{[A]} &= -k[t]_0^t\\
    [A]-[A(t=0)] &= -k(t-0)\\
\end{align*}
\textcolor{red}{Integriertes Geschwindigkeitsgesetz:}\\
\begin{equation*}
    [A] = [A(t=0)]-kt
\end{equation*}\\
Die Halbwertszeit $t_\frac{1}{2}$ ist häufig eine nützliche Größe, sie entspricht der Zeit nach der gerade die Hälfte der Ausgangskonzentration umgesetzt wurde.\\
Für $t_\frac{1}{2}$ gilt: $[A]=\frac{[A(t=0)]}{2}$\\
Setzt man dies in das integrierte Geschwindigkeitsgesetz ein, so erhält man:\\
\begin{align*}
    \frac{[A(t=0)]}{2} &= [A(t=0)]-kt_\frac{1}{2}\\
    f_\frac{1}{2} &= \frac{[A(t=0)]}{2k}
\end{align*}
\subsection{Reaktionen 1.Ordnung}
Linearer Zusammenhang zwischen Reaktantenkonzentration und Reaktionsgeschwindigkeit.\\
\underline{Typisches Beispiel:} Radioaktiver Zerfall.\\
Reaktion: \ce{A ->[{k}] P}\\
\begin{align*}
    v &= -\frac{d[A]}{dt}\\
    &= k[A]\\
\end{align*}
Integration:
\begin{align*}
    \int_{[A(t=0)]}^{[A]}\frac{d[A]}{[A]} &= -k\int_{0}^{t}\,dt\\
    \left[\ln [A]\right]_{[A(t=0)]}^{[A]} &= -k[t]_0^t\\
    \ln [A] - \ln [A(t=0)] &= -k(t-0)\\
    \text{Integriertes geschwindigkeitsgesetz:}\\
    \ln \frac{[A]}{[A(t=0)]} &= -kt\\
    [A] &= [A(t=0)]e^{-kt}
\end{align*}
Halbwertszeit $t_\frac{1}{2}$ durch Einsetzen von $[A] = \frac{[A(t=0)]}{2}$ in das integrierte Geschwindigkeitsgesetz bei Reaktionen 1. Ordnung, erhält man:\\
\begin{align*}
    \frac{[A(t=0)]}{2} &= [A(t=0)]e^{-kt_\frac{1}{2}}\\
    \ln [A(t=0)] - \ln 2 &= \ln [A(t=0)] - kt_\frac{1}{2}\\
    t_\frac{1}{2} &= \frac{\ln 2}{k}
\end{align*}
Die Halbwertszeit von Reaktionen 1.Ordnung ist unabhängig von der Anfangskonzentration.\\
Wie verändert sich die Produkt konzentration mit der Zeit?\\
\ce{A ->[{k}] P}\\
Für $t=0,\, [P]=0,\, [A]=[A(t=0)]$\\
Für $t=t,\, [A] = [A(t=0)]-[P]$\\
\begin{equation*}
    v = \frac{1}{\nu_A}\frac{d[A]}{dt} = \frac{1}{\nu_p}\frac{d[P]}{dt} = k[A]
\end{equation*}
\begin{align*}
    \frac{d[P]}{dt} &= k([A(t=0)]-[P]) \text{ mit } \nu_P = 1\\
    \int_{0}^{[P]} \frac{d[P]}{[A(t=0)]-[P]} &= k \int_{0}^{t} dt\\
    \left[-\ln \left([A(t=0)]-[P]\right)\right]_0^{[P]} &= k[t]_0^t\\
    -\ln \left([A(t=0)]-[P]\right) + \ln ([A(t=0)]-[0]) &= k(t-0)\\
    \ln \frac{[A(t=0)]}{[A(t=0)]-[P]} &= kt\\
    \frac{[A(t=0)]}{[A(t=0)]-[P]} &= e^{kt}\\
    [A(t=0)]e^{-kt} &= [A(t=0)]-[P]\\
    [P] &= [A(t=0)]\left(1-e^{-kt}\right)
\end{align*}

\subsection{Reaktion 2.Ordnung}
\subsubsection{Variante 1}
\ce{2A ->[{k}] P}   $\nu_A = -2$\\
Geschwindigkeitsgesetz:\\
\begin{align*}
    -\frac{1}{2}\frac{d[A]}{dt} &= k[A]^2\\
    \frac{d[A]}{dt} &= -2 k [A]^2
\end{align*}
$k: [\mathrm{\frac{l}{mol\cdot s}}]$\\
Integration:
\begin{align*}
    \int_{[A(t=0)]}^{[A]} \frac{d[A]}{[A]^2} &= - 2k \int_{0}^{t} \, dt\\
    \left[-\frac{1}{[A]}\right]^{[A]}_{[A(t=0)]} &= -2k[t]_0^t\\
    \text{Integriertes Geschwindigkeitsgesetz:}\\
    \frac{1}{[A]} - \frac{1}{[A(t=0)]} &= 2kt\\
\end{align*}
\end{document}