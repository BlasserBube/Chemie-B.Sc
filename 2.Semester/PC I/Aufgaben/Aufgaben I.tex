\documentclass{article}

\usepackage{amsmath}
\usepackage{mhchem}
\usepackage{chemfig}
\usepackage{xcolor}
\usepackage{graphicx}
\usepackage[
  left=1cm,
  right=1cm,
  top=2cm,
  bottom=2cm,
]{geometry}

\begin{document}
\section*{Aufgaben - Woche 1}
\subsection*{Aufgabe 1.1}
\begin{equation*}
    p = \frac{nRT}{V}; n=\frac{m}{M}
\end{equation*}
\begin{equation*}
    p = \frac{mRT}{MV}
\end{equation*}
\begin{equation*}
    M = \frac{mRT}{pV} = \frac{2.55 \mathrm{g}\cdot 373.15 \mathrm{K}\cdot\ 8.314 \mathrm{\frac{J}{mol\cdot K}}}{101325 \mathrm{kPa} }= 78.08 \mathrm{\frac{g}{mol}}
\end{equation*}
Für den Stoff mit der Formel \ce{C6H6} ergibt die Molmasse\\$M=6\cdot M(\ce{C})+ 6\cdot M(\ce{H}) = 78.08 \mathrm{\frac{g}{mol}}$ 

\subsection*{Aufgabe 1.2}
Für den Druck gilt:
\begin{equation*}
    p_H=\frac{nRT}{V}=\frac{ 2 \mathrm{mol}\cdot 273.1 \mathrm{K} \cdot 8.135 \mathrm{\frac{J}{mol\cdot K}}}{ 0.0224 \mathrm{m^3}} = 99187 \mathrm{\frac{N}{m}}
\end{equation*}
\begin{equation*}
    p_N= 198375,2 \mathrm{\frac{N}{m}}
\end{equation*}
Die Reaktion läuft nach folgender Gleichung ab:\\
\ce{3H2 + N2 -> NH3}\\
Somit ergibt der Druck nach der vollständigen Umsetzung:
\begin{equation*}
    p_{Ges}=\frac{3}{4}p_H + \frac{1}{4}p_N = 198375,2 \frac{\mathrm{N}}{\mathrm{m}}
\end{equation*}
\\
Korrekte Lösung:
\begin{center}
    \begin{tabular}{c c c c}
        \hline
        - & $n_H$ & $n_N$ $n_{NH3}$\\
        \hline
        vor: & 2 mol & 1 mol & -\\
        Reakt: & 2 mol & $\frac{2}{3}$ mol & $\frac{4}{3}$ mol\\
        nach: & - & $\frac{1}{3}$ mol & $\frac{4}{3}$ mol\\ 
        \hline
    \end{tabular}
\end{center}
\begin{equation*}
    n_{ges}=\frac{1}{3} + \frac{4}{3} = \frac{5}{3} ( \mathrm{mol})
\end{equation*}
\begin{equation*}
    p_{ges} = 1,69 \cdot 10^5 Pa
\end{equation*}
\begin{equation*}
    x_i = \frac{n_i}{n_{ges}}
\end{equation*}
\begin{equation*}
    x_H = 0; x_N = \frac{\frac{1}{3}}{\frac{5}{3}}; x_{NH3}=0,8
\end{equation*}
\begin{equation*}
    p_i = x_i \cdot p_{ges}
\end{equation*}
Damit berechnen.

\subsection*{Aufgabe 1.3}
a)
\begin{equation*}
    dp = \left(\frac{\partial p}{\partial V}\right)_T dV + \left(\frac{\partial p}{\partial T}\right)_V dT
\end{equation*}
b)
\begin{equation*}
    d^2p=\frac{2nR}{V^3}d^2V-2\frac{nR}{V^2}dTdV
\end{equation*}
Schwartzschen Satz beweisen:
\begin{equation*}
    \frac{\partial^2 p}{\partial V \partial T} = \frac{\partial^2 p}{\partial T \partial V}
\end{equation*}
\begin{equation*}
    -\frac{nr}{V^2} = -\frac{nr}{V^2} 
\end{equation*}

\subsection*{Aufgabe 1.4}
a)
\begin{equation*}
    \alpha = \frac{1}{V}\left(\frac{\partial V}{\partial T}\right)_p = \frac{1}{\frac{nRT}{p}} \cdot \frac{nR}{p} = \frac{1}{T}
\end{equation*}
\begin{equation*}
    \beta = \frac{1}{p}\left(\frac{\partial p}{\partial T}\right)_V = \frac{1}{\frac{nRT}{V}} \cdot \frac{nR}{V} = \frac{1}{T}
\end{equation*}
\begin{equation*}
    K = -\frac{1}{V}\left(\frac{\partial V}{\partial p}\right)_V = -\frac{1}{\frac{nRT}{V}} \cdot -\left(\frac{nRT}{p^2}\right) = \frac{1}{p}
\end{equation*}
b)
\begin{equation*}
    \alpha = \beta K p
\end{equation*}
\begin{equation*}
    \frac{1}{V}\left(\frac{\partial V}{\partial T}\right)_p = \left(\frac{\partial p}{\partial T}\right)_V\cdot \left(-\frac{1}{V}\right)\left(\frac{\partial V}{\partial p}\right)_T
\end{equation*}
\begin{equation*}
    \left(\frac{\partial V}{\partial T}\right)_p = -\left(\frac{\partial p}{\partial T}\right)_V \left(\frac{\partial V}{\partial P}\right)_T   |\cdot \left(\frac{\partial T}{\partial V}\right)_P
\end{equation*}
\begin{equation*}
    \left(\frac{\partial V}{\partial T}\right)_p \left(\frac{\partial T}{\partial V}\right)_p = -\left(\frac{\partial p}{\partial T}\right)_V \left(\frac{\partial V}{\partial P}\right)_T \left(\frac{\partial T}{\partial V}\right)_P
\end{equation*}
Da gilt:
\begin{equation*}
    \left(\frac{\partial V}{\partial T}\right)_p\left(\frac{\partial p}{\partial T}\right)_V\left(\frac{\partial V}{\partial p}\right)_T = -1 ; \left(\frac{\partial V}{\partial T}\right)_p \left(\frac{\partial T}{\partial V}\right)_p = 1
\end{equation*}
Ergibt dies:
\begin{equation*}
    1 = 1
\end{equation*}

\subsection*{Aufgabe 1.5}
\begin{equation*}
    \lambda = \frac{\langle v \rangle}{z_1}=\frac{1}{\sqrt{2}\sigma \frac{N}{V}}=\frac{1}{\sqrt{2}\sigma \frac{p}{K_BT}}
\end{equation*}
Auf diese GLeichung kommt man mit folgenden Umformungen:
\begin{equation*}
     z_1 = \sqrt{2} \langle v \rangle \sigma \frac{N}{V}
\end{equation*}
\begin{equation*}
    pV = nRT | n = \frac{N}{N_A}
\end{equation*}
\begin{equation*}
    pV = \frac{N}{N_A} RT = NK_BT
\end{equation*}
\begin{equation*}
    p = \frac{N}{V}K_BT
\end{equation*}
\begin{equation*}
    \frac{N}{V} = \frac{p}{K_BT}
\end{equation*}
Somit:
\begin{equation*}
    \lambda_N = 6.76 \cdot 10^{-5} \mathrm{m}
\end{equation*}

\subsection*{Aufgabe 1.6}
\begin{equation*}
    T_1 = 273.15 \, \mathrm{K}, T_2 = 373.15 \, \mathrm{K}
\end{equation*}
\begin{equation*}
    p_1 = p_2
\end{equation*}
\begin{equation*}
    \frac{n_1 R T_1}{V} = \frac{n_2 R T_2}{V}
\end{equation*}
\begin{equation*}
    n_1 T_1 = n_2 T_2
\end{equation*}
\begin{equation*}
    n_1 + n_2 = n = 2 \, \mathrm{mol}
\end{equation*}
\begin{equation*}
    n_1T_1=(2-n_1)T_2
\end{equation*}
\begin{equation*}
    n_1T_1=2T_2-n_1T_2
\end{equation*}
\begin{equation*}
    n_1(T_1+T_2)=2T_2
\end{equation*}
\begin{equation*}
    n_1\frac{2T_2}{T_1 + T_2} = 0.845 \, \mathrm{mol}
\end{equation*}
\begin{equation*}
    n_2 = 2 - n_1 = 1.155 \, \mathrm{mol}
\end{equation*}
\begin{equation*}
    p = \frac{n_1RT_1}{V} = 1.072 \cdot 10^5 \, \mathrm{Pa}
\end{equation*}

\subsection*{Aufgabe 1.7}
\begin{equation*}
    E_{pot} = 4\varepsilon\left(\left(\frac{r_0}{r}\right)^{12} - \left(\frac{r_0}{r}\right)^6\right)
\end{equation*}
\begin{equation*}
    F = \frac{dE_{pot}}{dr}=\left(\left(-12\cdot 4\varepsilon r_0^{12} \cdot r^{-13}\right)-\left(-6\cdot 4\varepsilon r_0^6r^{-7}\right)\right)
\end{equation*}
\begin{equation*}
    \frac{48\varepsilon r_0^{12}}{r^13} - \frac{24\varepsilon r_0^6}{r^7} = 0
\end{equation*}
damit:
\begin{equation*}
    0 = \frac{2 r_0^12}{r^13}-\frac{r_0^6}{r^2}
\end{equation*}
Damit:
\begin{equation*}
    r_0^6r^6 = 2r_0^12
\end{equation*}
\begin{equation*}
    r = \sqrt[6]{2}r_0
\end{equation*}
\begin{equation*}
    r^6=\frac{2r_0^{12}}{r_0^6}=2r_0^6
\end{equation*}
\begin{equation*}
    r=\sqrt[6]{2}r_0
\end{equation*}
\begin{equation*}
    E_{pot}=-\varepsilon
\end{equation*}

\section*{Aufgaben - Woche 2}
\subsection*{Aufgabe 2.1}
Wird die Virialgleichung nach dem zweiten Glied abgebrochen lautet diese:
\begin{equation*}
    \frac{pV_m}{RT} = 1+B_pp
\end{equation*}
Mit den ersten Werten $p = 1.013$ bar und $pV_m = 22.693\,\mathrm{\frac{bar}{mol}}$, ergibt sich:
\begin{equation*}
    \frac{22.693\,\mathrm{\frac{bar}{mol}}}{8.3145\,\mathrm{\frac{J}{molK}}\,\cdot273\,\mathrm{K}} = 1 + B_p \cdot 1.013 \,\mathrm{bar} \rightarrow B_p = -0.977\,\mathrm{J}
\end{equation*}
\begin{center}
    \begin{tabular}{c c c}
        \hline
        $p$ [bar] & $pV_m$ [$\mathrm{\frac{bar}{mol}}$] & $B_p$ [J]\\
        \hline
        1.013&22.693&-0.9772\\
        3.039&22.673&-0.3256\\
        5.065&22.652&-0.1955\\
        \hline
    \end{tabular}
\end{center}
Mit der Gleichung
\begin{equation*}
    T_B \approx \frac{a}{bR}
\end{equation*}
mit den Van-der-Waals-Koeffizienten $a(\ce{N2}) = 140.8\cdot 10^{-3}\,\mathrm{\frac{Jm^3}{mol^2}}$ und $b(\ce{N2})=39.1\cdot 10^{-6}\,\mathrm{\frac{m^3}{mol}}$
\begin{equation*}
    T_B = 433.10\,\mathrm{K} < 273\,\mathrm{K}
\end{equation*}
Somit liegt die Messtemperatur unter der Boyletemperatur $T_B$.

\subsection*{Aufgabe 2.2}
\begin{equation*}
    p = \frac{nRT}{V-nb} - a \left(\frac{n}{V}\right)^2 = \frac{1\cdot 8.3145 \cdot 200}{0.005 - 1 \cdot 39.13 \cdot 10^{-6}}-140,8\cdot 10^{-3}\left(\frac{1}{0.005}\right)^2 \,\mathrm{bar} = 329.5740\,\mathrm{kbar}
\end{equation*}
\begin{equation*}
    p = \frac{nRT}{V} \rightarrow p = \frac{1\cdot 8.3145 \cdot 200}{0.005}\,\mathrm{bar} = 332.5800\,\mathrm{kbar}
\end{equation*}
Da der Druck nach der VdW Gleichung kleiner ist als nach der idealen Gasgleichung ist davon auszugehen, dass die anziehenden Kräfte zwischen den Molekülen überwiegt, dafür spricht auch das typische Verhalten bei kleinem $T$ und $B_p < 0$

\subsection*{Aufgabe 2.3}
a)
\begin{equation*}
    p = \frac{RT}{V_m-b}-\frac{a}{V_m^2}
\end{equation*}
\begin{equation*}
    \frac{\partial p}{\partial V_m} = 0, \, \frac{\partial^2 p}{partial V_m^2}=0
\end{equation*}
\begin{equation*}
    \frac{\partial p}{\partial V_m} = \frac{RT}{(V_m-b)^2}+\frac{2a}{V_m^3}=0 \Rightarrow \frac{2}{(V_m-b)}\frac{2a}{V_m^3} = \frac{6a}{V_m^4}
\end{equation*}
\begin{equation*}
    \frac{\partial^2 p}{\partial V_m^2} = \frac{2RT}{(V_m-b)^3}-\frac{6a}{Vm^4}=0 \Rightarrow \frac{2}{V_m-b}=\frac{3}{V_m}
\end{equation*}
\begin{equation*}
    V_{m,krit} = 3b
\end{equation*}
Daraus folgt:
\begin{equation*}
    T_{krit} = \frac{2a(V_m-b)^2}{V_m^3\cdot R}=\frac{8a}{27Rb}
\end{equation*}
\begin{equation*}
    p_{krit} = \frac{\frac{8a}{b}}{3b-b}-\frac{a}{9b^2} = \frac{a}{27b^2}
\end{equation*}
b)
\begin{equation*}
    T_{krit} = 304,01 \,\mathrm{K}
\end{equation*}
c)
\begin{equation*}
    \left(p_r\frac{3}{V_r^2}\right) = \frac{\frac{8T_r}{3}}{\left(V_r-\frac{1}{3}\right)}
\end{equation*}
\begin{equation*}
    \left(p_r\frac{3}{V_r^2}\right)\left(V_r-\frac{1}{3}\right) = \frac{8}{3}T_r
\end{equation*}

\subsection*{Aufgabe 2.4}
\begin{equation*}
    p = \frac{RT}{Vm}-\frac{B}{V_m^2}+\frac{C}{V_m^3}
\end{equation*}
\begin{eqnarray*}
    p' = -RTV_m^{-2}+2BV_m^{-3}-3CV_m^{-4} = 0\\
    p'' = 2RTV_m^{-3}-6BV_m^{-4}+12CV_m^{-5} = 0
\end{eqnarray*}
Die beiden miteinander verrechnet ergibt:
\begin{eqnarray*}
    (4B-6B)V_m+12C-6C = 0\\
    -2BV_m+6C=0\\
    V_{m,krit} = \frac{3C}{B}
\end{eqnarray*}

\subsection*{Aufgabe 2.5}
\begin{eqnarray*}
    \frac{pV_m}{RT}=1+B_pp+C_pp^2\\
    \frac{pV_m}{RT}=1+\frac{B_V}{V_m}+\frac{C_V}{V_m^2}
\end{eqnarray*}
Die zweite Gleichung nach p umgestellt ergibt:
\begin{equation*}
    p = \frac{RT}{V_m}+\frac{B_VRT}{V_m}+\frac{C_VRT}{V_m^3}
\end{equation*}
\begin{equation*}
    \frac{pV_m}{RT}=1+\frac{B_pRT}{V_m}+\frac{B_pB_VRT}{V_m^2}+\frac{B_pC_VRT}{V_m^3}+C_p\left(\frac{RT}{V_m}+\frac{B_VRT}{V_m^2}+\frac{C_VRT}{V_m^3}\right)\cdot\left(\frac{RT}{V_m}+\frac{B_VRT}{V_m^2}+\frac{C_VRT}{V_m^3}\right)
\end{equation*}
\begin{eqnarray*}
    \frac{pV_m}{RT}\approx 1 + \frac{B_pRT}{V_m}+\frac{B_pB_vRT}{V_m^2}+\frac{C_p(RT)^2}{V_m^2}\\
    =1+\frac{B_pRT}{V_m}+\frac{RTB_pB_V+C_p(RT)^2}{V_m^2}
\end{eqnarray*}
Somit ist $\frac{B_pRT}{V_m} = B_V$ und $\frac{RTB_pB_V+C_p(RT)^2}{V_m^2} = C_V$
\begin{eqnarray*}
    = B_V^2+C_p(RT)^2\\
    C_V - B_V^2 = C_p(RT)^2\\
    C_p = \frac{C_V-B_V^2}{(RT)^2}
\end{eqnarray*}

\subsection*{Aufgabe 2.6}
a)
\begin{equation*}
    w = -\int_{V_A}^{V_E} p\,dV = -nRT\ln\left(\frac{V_E}{V_A}\right) = -1\cdot 8.3145 \cdot 273 \cdot \ln\left(\frac{0.0448}{0.0224}\right) \,\mathrm{J}= -1573,3460 \,\mathrm{J}
\end{equation*}
\\b)
\begin{equation*}
    -p_{ex}\Delta V = -\frac{RT}{V} \cdot 0.0224\,\mathrm{m^3}=-1135.5528\,\mathrm{J}
\end{equation*}
\\c)
\begin{equation*}
    w = 0
\end{equation*}

\subsection*{Aufgabe 2.7}
\begin{eqnarray*}
    \partial w = -p\,dV\\
    dV = \left(\frac{\partial V}{\partial T}\right)_p\,dT + \left(\frac{\partial V}{\partial p}\right)_T\,dp,\,V=\frac{nRT}{p}\\
    \left(\frac{\partial V}{\partial T}\right)_p = \frac{nR}{p}\\
    \left(\frac{\partial V}{\partial p}\right)_T = \frac{-nRT}{p^2}\\
    \partial w = -p\frac{nR}{p}\,dT + p \frac{nRT}{p^2}dp = -nR\,dT + \frac{nRT}{p}\,dp\\
    \left(\frac{\partial w}{\partial T}\right)_p = -nR,\, \left(\frac{\partial w}{\partial p}\right)_T = \frac{nRT}{p}\\
    \left(\frac{\partial^2 w}{\partial T \partial p}\right) = \frac{\partial}{\partial p}\left(\frac{nRT}{p}\right)=\frac{nR}{p}\\
    \frac{\partial^2 w}{\partial p \partial T} = \frac{\partial}{\partial p}(-nR) = 0\\
    \frac{\partial^2 w}{\partial T \partial p} = \frac{\partial^2 w}{\partial p \partial T} \Rightarrow \partial w
\end{eqnarray*}
$w$ ist keine Zustandsgröße

\section*{Woche 3}
\subsection*{Augabe 3.1}
a)\begin{eqnarray*}
    V=\mathrm{const.},\,w=0\\
    q_v=nc_{m,V}\Delta T\\
    c_p=c_V+nR \rightarrow c_{v,m} = c_{p,m} - R\\
    q_V=(c_{p,m}-R)n\Delta T = 124.76\,\mathrm{J}
\end{eqnarray*}
Es wird keine Arbeit verrichtet\\
b)\begin{eqnarray*}
    p = \mathrm{const.}\\
    q_p=nc_{p,m}\Delta T = 207.9\,\mathrm{J}\\
\end{eqnarray*}
\begin{equation*}
    w=-p\Delta V \ce{->[{pV=nRT}]} -p(\frac{nRT_E}{p}-\frac{nRT_A}{p})=-nr\Delta T = -83.14\,\mathrm{J}
\end{equation*}

\subsection*{Aufgabe 3.2}
\begin{eqnarray*}
    q_K=C\Delta T,\,V=\mathrm{const.}\\
    q_V=\Delta U=-C\Delta T = n \Delta U_m = \frac{m}{M} \Delta U_m\\
    \Delta U_m = -\frac{C\Delta TM}{m}=-3261.96\,\mathrm{J}
\end{eqnarray*}
\begin{eqnarray*}
    H=U+pV\\
    \Delta H=\Delta U + \Delta (pV) = \Delta U + p \Delta V \rightarrow[{p\Delta V = \Delta n_gRT}]\rightarrow \Delta U + \Delta n_g RT\\
    \Delta n_g = (6-7.5)\,\mathrm{mol} = -1.5\,\mathrm{mol}\\
    \Delta H_m = -3265.7\cdot 10^3 \,\mathrm{\frac{J}{mol}}
\end{eqnarray*}

\subsection*{Aufgabe 3.3}
\begin{eqnarray*}
    \mu = \left(\frac{\partial T}{\partial p}\right)_H
    d H_m = \left(\frac{\partial H_m}{\partial T}\right)_p\,dT+\left(\frac{\partial H_m}{\partial p}\right)_T\,dp=C_{p,m}\,dT+\left(\frac{\partial H_m}{\partial p}\right)_T\,dp\\
    \text{mit: } \left(\frac{\partial H_m}{\partial p}\right)_T = -T\left(\frac{\partial V_m}{\partial T}\right)_p + V_m\\
    c_{p,m}\,dT+\left[-T\left(\frac{\partial V_m}{\partial T}\right)_p+V_m\right]\,dp=0\\
    \mu = \left(\frac{\partial T}{\partial p}\right)_H = \frac{1}{C_{p,m}}\left[T\left(\frac{\partial V_m}{\partial T}\right)_p - V_m\right]
\end{eqnarray*}
a)\begin{eqnarray*}
    pV_m-pb=RT\\
    V_m = \frac{RT}{p}+b\\
    \left(\frac{\partial V_m}{\partial T}\right)_p = \frac{R}{p}\\
    \mu = \frac{1}{c_{p,m}}\left[T\frac{R}{p}-\left(\frac{RT}{p}+b\right)\right]=-\frac{b}{c_{p,m}}<0\\
\end{eqnarray*}
b)\begin{eqnarray*}
    pV_m=RT+(b-\frac{a}{RT})_p\\
    V_m = \frac{RT}{p}+b-\frac{a}{RT}\\
    \left(\frac{\partial V_m}{\partial T}\right)_p = \frac{R}{p}+\frac{a}{RT^2}\\
    \mu = \frac{1}{c_{p,m}}\left[-\left(\frac{R}{p}+\frac{a}{RT^2}\right)-\left(\frac{RT}{p}+b-\frac{a}{RT}\right)\right]\\
    = \frac{1}{c_{p,m}}\left(\frac{2a}{RT}-b\right)
\end{eqnarray*}
\subsection*{Aufgabe 3.4}
\begin{center}
    I. \ce{(Cyclohexan) + 18 O2 -> 6CO2 + 6H2O}\\
    II. \ce{(Cyclohexen) + 16 O2 -> 6CO2 + 5H2O}\\
    III. \ce{(Cyclohexa-1,3-dien) + 14 O2 -> 6CO2 + 4H2O}\\
    IV. \ce{(Benzol) + 12 O2 -> 6CO2 + 3H2O}
\end{center}
I.
\begin{eqnarray*}
    6\cdot\Delta_BH^0_m(\ce{CO2})+6\cdot\Delta_BH^0_m(\ce{H2O})-\Delta_BH^0_m(\mathrm{Cyclohexan})\\
    6\cdot -395.5 + 6 \cdot -285.9 + 156.2\,\mathrm{\frac{kJ}{mol}} = -3932.2\,\mathrm{\frac{kJ}{mol}}\\
\end{eqnarray*}
Analog:\\
II. gegeben: $-3739.0\,\mathrm{\frac{kJ}{mol}}$\\\\
III $-3623.6\,\mathrm{\frac{kJ}{mol}}$\\\\
IV $-3279.74\,\mathrm{\frac{kJ}{mol}}$\\\\
Hydrierungsenthalpien:\\
II:
\begin{equation*}
    -3739 + 3932.2 \,\mathrm{\frac{kJ}{mol}}= 193.2\,\mathrm{\frac{kJ}{mol}}
\end{equation*}
III:
\begin{equation*}
    -3623.6 + 3739 \,\mathrm{\frac{kJ}{mol}}= 116\,\mathrm{\frac{kJ}{mol}}
\end{equation*}
IV:
\begin{equation*}
    -3279.74 + 3623.6 \,\mathrm{\frac{kJ}{mol}}= 343.86\,\mathrm{\frac{kJ}{mol}}
\end{equation*}
\subsection*{Aufgabe 3.5}
\begin{eqnarray*}
    \Delta_RH^0(600)=\Delta_RH^0(298.15)+(0.001(6\cdot 36 + 4\cdot 29 - 5 \cdot 29 - 4\cdot 42))\Delta T \,\mathrm{\frac{kJ}{mol}}= -904.6 \cdot 0.019 \cdot 301.85 \,\mathrm{\frac{kJ}{mol}} = -5188.0167\,\mathrm{\frac{kJ}{mol}}
\end{eqnarray*}

\section*{Woche 4}
\subsection*{Aufgabe 4.1}
a)
\begin{center}
    \includegraphics[width=500pt]{img/What.png}
\end{center}
b)\begin{eqnarray*}
    V_A = 24.5\,\mathrm{l}\\
    V_B = 49\,\mathrm{l}\\
    T_A = 298\,\mathrm{K}\\
    T_B = 298\,\mathrm{K}\\
    p_AV_A=p_BV_B (T=\mathrm{const.})\\
    p_B=\frac{p_AV_A}{V_B}=\frac{p_A}{2} = \frac{2\,\mathrm{bar}}{2} = 1\,\mathrm{bar}\\\\
    B\rightarrow C\\
    p = \mathrm{const} p_C = 1\,\mathrm{bar}\\\\
    C\rightarrow A (q = 0)\\
    p_AV_A^\gamma = p_CV_C^\gamma\\
    \gamma = \frac{C_{p,m}}{C_{V,m}} = \frac{C_{V,m}+R}{C_{v,m}} = \frac{\frac{5R}{2}+R}{\frac{5R}{2}} = \frac{7}{5}= 1,4
    V_C = V_A \left(\frac{p_A}{p_C}\right)^\frac{1}{\gamma} = 24.5\,\mathrm{l}\cdot 2^\frac{1}{1.4} = 40.2\,\mathrm{l}\\\\
    B\rightarrow C\\
    \frac{V_C}{V_B} = \frac{T_C}{T_B}\\
    T_C = \frac{T_BV_C}{V_B} = 244.5\,\mathrm{K}\\
    T_V = \frac{p_CV_C}{nR}\\
    n=\frac{p_AV_A}{RT_A} = 1.978\,\mathrm{mol}
\end{eqnarray*}
c)\begin{eqnarray*}
    A\rightarrow B\\
    T =\mathrm{const},\,\Delta U_1 = 0 = q_1+w_1 \Rightarrow w_1 = -q_1\\
    w_1 = -nRT_A\ln\frac{V_B}{V_A} = -p_AV_A\ln 2 = -2\cdot 10^5 \,\mathrm{Pa} \cdot 24.5\,cdot 10^{-3}\,\mathrm{m^3}\ln 2 = -3396,42\,\mathrm{J}\\
    q_1=3396,42\,\mathrm{J}\\\\
    B\rightarrow C\\
    p = \mathrm{const.}\\
    w_2 = -p_B(V_C-V_B)=-10^5\,\mathrm{Pa}(40.2-49)\cdot 10^{-3}\,\mathrm{m^3} = 880.4\,\mathrm{J}\\
    q_2 = nC_{p,m}(T_C-T_B) = \frac{7}{2}\cdot 8.314\,\mathrm{\frac{J}{mol\cdot K}}\cdot 1.978\,\mathrm{mol}(244.5-298)\,\mathrm{K} = -3081.6\,\mathrm{J}\\
    C_{p,m}=C_{V,m}+R = \frac{7R}{2}\\\\
    C\rightarrow A\\
    q_3 = 0,\, \Delta U_3 = w_3 = nC_{V,m}(T_A-T_C)=2201.18\,\mathrm{J}
\end{eqnarray*}
d)\begin{eqnarray*}
    \eta = \frac{|w_{Ges|}}{q_1}\\
    w_{Ges} = 2_1 + w_2 + w_3 = -314.8\,\mathrm{J}\\
    \eta = \frac{314.8\,\mathrm{J}}{3396.42\,\mathrm{J}} = 0.093 = 9.3\,\%
\end{eqnarray*}

\subsubsection*{Aufgabe 4.2}
\begin{eqnarray*}
    dS = \frac{\delta q_{rev}}{T},\, \delta q =\frac{nRT}{V}dV\\
    dS = \frac{1}{T}\left(\frac{nRT}{V}dV\right)\\
    \Delta S = nR \int_{V_A}^{V_E}\frac{1}{V}dV\\
    = nR \ln \frac{V_E}{V_A}\\
    \Delta S = \Delta S_A + \Delta S_B\\
    =n_A R\ln \left(\frac{V_{ges}}{V_A}\right)+n_BR\ln\left(\frac{V_{ges}}{V_B}\right),\, p_V = nRT \Rightarrow V = \frac{nRT}{p}\\
    =n_AR\ln\left(\frac{n_{ges}}{n_A}\right)+n_BR\ln\left(\frac{n_{ges}}{n_A}\right)\\\\
    B:
    \Delta S = 4.16\,\mathrm{\frac{J}{K}}\\
    \text{Wenn nicht gleich 0, irreversibel}\\
\end{eqnarray*}

\subsection*{Aufgabe 4.3}
\begin{eqnarray*}
    T_W =\frac{T_1+T_2}{2}\\
    \Delta S_{Sys} = \Delta S_1 + \Delta S_2\\
    = \int_{T_1}^{T_E}C_p\,dT + \int_{T_1}^{T_E}C_p\,dT\\
    C_P\ln\frac{T_2}{T_1}+C_p\ln\frac{T_E}{T_2} \text{ für} T_E \text{ in vorheriger Gleichung einsetzen}\\
    C_p\ln\frac{T_E^2}{T_1T_2}=C_p\ln\frac{\left(T_1+T_2\right)^2}{4T_1T_2}\\
    \Delta S_{Umg} = 0\\
    \frac{\ln\left(T_1 + T_2\right)^2}{4T_1T_2}\\\text{Der obere Teil im ln muss größer 1 sein, damit der gesamte obere Teil des Bruches größer als 1 wird}\\
    (T_1 + T_2)^2 > 4T_1T_2\\
    T_1^2+2T_1T_2+T_2^2 > 4T_1T_2\\
    (T_1 - T_2)^2 > 0\\
    T_1 \neq T_2
\end{eqnarray*}

\subsubsection*{Aufgabe 4.4}
a)\begin{eqnarray*}
    \Delta S_m = -\Delta_{S,m} H_m^{273}=\frac{-6000\,\mathrm{\frac{J}{mol}}}{273.15\,\mathrm{K}}=-21.97\,\mathrm{\frac{J}{mol\cdot K}}\\
    \Delta_{einfrieren}H = -\Delta_{s,m}H\\
    \Delta S_{Umg} = 21.97\,\mathrm{\frac{J}{mol\cdot K}}\\
    \Delta S_{Ges} = 0
\end{eqnarray*}
b)\begin{align*}
    \Delta S &=C_p\ln\frac{T_E}{T_A}\\
    \Delta S &= \Delta S_{1,m} + \Delta S_{2,m} + \Delta S_{3,m} = C_{p,m}\ln\frac{T_2}{T_1}-\frac{\Delta_{s,m}H^{273}}{T_2}+C_{p,m}\ln\frac{T_1}{T_2}\\
    &=75\,\mathrm{\frac{J}{mol\cdot K}}\ln\frac{273.15}{263.15}-21.97\,\,mathrm{\frac{J}{mol\cdot K}}+38\,\mathrm{\frac{J}{mol\cdot K}}\ln\frac{263.15}{273.15}=-20.59\,\mathrm{\frac{J}{mol\cdot K}}\\
    \mathrm{Umgebung}:\\
    \Delta S_{Umg} &= \frac{\Delta_{s,m} H ^{263}}{T_1} = \frac{5630\,\mathrm{\frac{J}{mol\cdot K}}}{263.15\,\mathrm{K}}=21.395\,\mathrm{\frac{J}{mol\cdot K}}\\
    \Delta S_{Ges} &= 0.8\,\mathrm{\frac{J}{mol\cdot K}}
\end{align*}

\section*{Woche 5}
\subsection*{Aufgabe 5.1}
1) Eis -10 °C auf 0 °C
\begin{eqnarray*}
    q_1 = nC_{p,m}\Delta T\\
    n = \frac{m}{M} = \frac{1000}{18} = 55.556\,\mathrm{mol}\\
    q_1 = 55.556\,\mathrm{mol}\cdot 38\,\mathrm{\frac{J}{mol\cdot K}}\cdot 10\,\mathrm{K}=21.1\,\mathrm{kJ}
\end{eqnarray*}
2)
\begin{eqnarray*}
    q_2 = n\Delta H_{schmelz,s}^{Eis} = 333,3\,\mathrm{kJ}
\end{eqnarray*}
3) Wasser 0 °C auf 100 °C
\begin{eqnarray*}
    q_3 = nC_{p,m}^{Wasser}\Delta T =419.4\,\mathrm{kJ}
\end{eqnarray*}
4) Verdampfen
\begin{eqnarray*}
    q_4 = n \Delta H_{verd,m}1{Wasser} = 2277.8\,\mathrm{kJ}
\end{eqnarray*}
5) Dampf 100 °C auf 150 °C
\begin{eqnarray*}
    q_5 = nC_{p,m}^{Dampf} \Delta T = 
\end{eqnarray*}
\begin{equation*}
    q_{\sum} = q_1 + q_2 + q_3 + q_4 + q_5 = 3146\,\mathrm{kJ}
\end{equation*}
\begin{align*}
    \Delta S &= n C_{p,m} \ln\frac{T_E}{T_A}\text{ Erwärmen}\\
    \Delta S &= \frac{n\Delta H}{T}=\frac{q}{T}\text{ Phasenübergang}\\
    \Delta S_1 &= 78.8\,\mathrm{\frac{J}{K}}\\
    \Delta S_2 &= 1221\,\mathrm{\frac{J}{K}}\\
    \Delta S_3 &= 1309.1\,\mathrm{\frac{J}{K}}\\
    \Delta S_4 &= 6106.6\,\mathrm{\frac{J}{K}}\\
    \Delta S_5 &= 237.6\,\mathrm{\frac{J}{K}}\\
    \Delta S_{\sum} &= 8953.1\,\mathrm{\frac{J}{K}}
\end{align*}
\subsection*{Aufgabe 5.2}
\begin{align*}
    C &= a+bT\\
    T_1 &= 373\,\mathrm{K}\\
    T_2 = 573\,\mathrm{K}\\
\end{align*}
a)
\begin{align*}
    \Delta H &= a(T_2-T_1)+\frac{b}{2}(T_2^2-T_1^2) = 22.996\,\mathrm{\frac{kJ}{mol}}
\end{align*}
b)
\begin{align*}
    \Delta S = a \ln\frac{T_2}{T_1} + b(T_2-T_1) = 48.734\,\mathrm{\frac{J}{mol\cdot K}}
\end{align*}

\subsection*{Aufgabe 5.3}
\begin{align*}
    \Delta S &= \int_{T_1}^{T_2} \frac{\delta q}{T} = \int_{T_1}^{T_2} \frac{C_p}{T}\,dT\\
    \Delta S &= \frac{\Delta H}{T}\\
    \Delta S &= S_{T_S} - S_{T=0\,\mathrm{K}} = S_{T_S}\\
    &= \int_{0}^{T_t} \frac{c_p\,dT}{T}+\frac{\Delta H_{trans}}{T_t} + \int_{T_t}^{T_t}\frac{c_p\,dT}{T}+\frac{\Delta H_{Schm}}{T_f} + \int_{T_f}^{T_S} \frac{c_p\,dT}{T}+\frac{\Delta H{Verd.}}{T_S} = 27.2\,\mathrm{\frac{J}{mol\cdot K}}\\
    &= 27.2\,\mathrm{\frac{J}{mol\cdot K}}+\frac{229\,\mathrm{\frac{J}{mol}}}{35.61\,\mathrm{K}}+23.4\,\mathrm{\frac{J}{mol\cdot K}}+\frac{721\,\mathrm{\frac{J}{mol}}}{63.14\,\mathrm{K}}+11.4\,\mathrm{\frac{J}{mol\cdot K}}+\frac{5580\,\mathrm{\frac{J}{mol}}}{77.32\,\mathrm{K}}\\
    &= 152\,\mathrm{\frac{J}{mol\cdot K}}
\end{align*}

\subsection*{Aufgabe 5.4}
a)
\begin{align*}
    -\left(\frac{\partial S}{\partial p}\right)_T &= \left(\frac{\partial V}{\partial T}\right)_p\\
    dG &= -S\,dT+V\,dp\\
    dG &= \left(\frac{\partial G}{\partial T}\right)_p \,dT + \left(\frac{\partial G}{\partial p}\right)_T\,dp\\
    \Rightarrow \left(\frac{\partial G}{\partial T}\right)_p &= -S\\
    \Rightarrow \left(\frac{\partial G}{\partial p}\right)_T &= V\\
    \frac{\partial}{\partial p}\left(\frac{\partial G}{\partial T}\right) &= \frac{\partial}{\partial T}\left(\frac{\partial G}{\partial p}\right)\\
    -\left(\frac{\partial S}{\partial p}\right)_T &= \left(\frac{\partial V}{\partial T}\right)_p\\
\end{align*}
b)
\begin{align*}
    dH &= T\,dS + V\,dp\\
    \frac{\partial H}{\partial p} &= T\frac{\partial S}{\partial p}+V\\
    \left(\frac{\partial H}{\partial p}\right)_T &= -T \left(\frac{\partial V}{\partial T}\right)_p + V
\end{align*}

\subsection*{Aufgabe 5.5}
\begin{equation*}
    G(p)=ap+b\ln p + c
\end{equation*}
a)
\begin{align*}
    V(p) &= \left(\frac{\partial G}{\partial p}\right)_T\\ &= \frac{\partial }{\partial p} (ap+b\ln p + c)\\ &= a+\frac{b}{p}
\end{align*}
b)
\begin{align*}
    K &= -\frac{1}{V}\left(\frac{\partial V}{\partial p}\right)_T\\
    \left(\frac{\partial V}{\partial p}\right)_T &= \frac{\partial}{\partial p}\left(a+\frac{b}{p}\right)\\
    &= -\frac{b}{p^2}\\
    K &= -\frac{1}{a+\frac{b}{p}}\left(-\frac{b}{p^2}\right)\\
    &= \frac{b}{ap^2 + bp}
\end{align*}
c)
\begin{align*}
    A(p)\\
    A &= U-TS\\
    H &= U+pV\\
    U &= H - pV\\
    A &= H-pV-TS\\
    G &= H-TS\\
    A &= G-pV\\
    &= ap + b\ln p + c - p (a+\frac{b}{p})\\
    &= b\ln p + c - b
\end{align*}

\section*{Woche 6}
\subsection*{6.1}
a) 3 Phasen - fest, flüssig, gasförmig. 1 Komponente - Natriumsulfat.
\begin{align*}
    F&=K-P+2\\
    F&=1-3+2\\
    &= 0
\end{align*}
b) 2 Phasen - gasförmig, flüssig. 1 Komponente\\
\begin{align*}
    F&=1-2+2\\
    &=1
\end{align*}

\subsection*{6.2}
----
\begin{align*}
    \frac{V}{\mathrm{cm^3}}&=1001.21+34.69\left(\frac{b}{1,\mathrm{\frac{mol}{kg}}}-0.07\right)^2\\
    &=1001.21+34.69\left(\frac{0.050\,\mathrm{\frac{mol}{kg}}}{1,\mathrm{\frac{mol}{kg}}}-0.07\right)^2\\
    &= 1001.1961\,\mathrm{cm^3}
\end{align*}
0.050 mol \ce{MgSO4} in 1kg Wasser haben ein Volumen von 1001.1961 cm$^3$\\
\begin{align*}
    0.050 &\hat{=} 1001.1961\\
    1 &\hat{=} 20023.922
\end{align*}
----

\subsection*{6.3}
\begin{align*}
    \frac{dp}{dT} &= \frac{\Delta_V H_m}{T_V \Delta_V V_m}\\
    \frac{0.68\,\mathrm{bar}}{30\,\mathrm{K}} &= \frac{\Delta_V H_m}{373.15\,\mathrm{K} \cdot \frac{R\cdot 373.15\,\mathrm{K}}{1.0\,\mathrm{bar}}}\\
    \Delta_V H_m &= \frac{0.68\,\mathrm{bar}}{30\,\mathrm{K}} \cdot 373.15\,\mathrm{K} \cdot \frac{8.3145\,\mathrm{\frac{J}{K\cdot mol}}\cdot 373.15\,\mathrm{K}}{1.0\,\mathrm{bar}}\\
    \Delta_V H_m &= 26241.6227\,\mathrm{\frac{J}{mol}}
\end{align*}


\subsection*{6.4}
----
\begin{align*}
    p_2 &= p_1 + \frac{\Delta_{sm}H_m}{\Delta_{sm}V_m}\left(\frac{T_2-T_1}{T_1}\right)\\
    \left(p_2 - p_1\right)\cdot \frac{\Delta_{sm}V_m}{\Delta_{sm}H_m} &= \frac{T_2-T_1}{T_1}\\
    T_2 &= \left(p_2 - p_1\right)\cdot \frac{T_1\Delta_{sm}V_m}{\Delta_{sm}H_m} + T_1\\
    T_2 &= \left(1000\,\mathrm{bar} - 1\,\mathrm{bar}\right)\cdot \frac{273.15\,\mathrm{K} \cdot 0.083\,\mathrm{\frac{kg}{dm^3}}}{6.01\,\mathrm{\frac{kJ}{mol}}} + 273.15\,\mathrm{K}\\
\end{align*}
----
\section*{Woche 7}
\subsection*{7.1}
a)
\begin{align*}
    \pi &= CRT = \frac{n}{V_l}RT\\
    n_{9\ce{NaCl}} &= 0.0017\,\mathrm{mol}\\
    n &= 2n_{\ce{NaCl}}\\
    \pi &= 8.48\cdot 10^4 \,\mathrm{Pa}
\end{align*}
b)
Da sich bei der Gefrierpunktserniedrigung um eine Kolligative Eigenschaft handelt, also die Anzahl an zugesetzten Teilchen eine Rolle spielt, nicht die Art dieser und die 0.1$\,\mathrm{\frac{mol}{l}}$ an \ce{CaCl2} und \ce{HNO3} in Wasser gelöst werden bzw. deprotonieren, ist der Effect hier stärker als bei der Essigsäure, da bei \ce{CaCl2} 3 Ionen pro in Wasser gelöstes Molekül vorliegen, fällt hier der Effekt der Gefrierpunktserniedrigung stärker aus, als bei der simplen Deprotonierung der Salpetersäure.\\
\subsection*{7.2}
a)
\begin{align*}
    \Delta T &= K_{Kr} \dot b_{\text{Saccharose}}\\
    n(\text{Saccharose}) &= \frac{m}{M} = \frac{7.5}{342.30}\,\mathrm{mol} = 0.02191\,\mathrm{mol}\\
    m(\text{Wasser}) &= \rho(\text{Wasser}) \cdot V = 0.993\,\mathrm{\frac{g}{cm^3}} \cdot 250\,\mathrm{cm^3} = 249.5\,\mathrm{g}\\
    b_{\text{Saccharose}} &= \frac{n(\text{Saccharose})}{m(Gesamt)} = \frac{0.02191\,\mathrm{mol}}{(249.5 + 7.5)\,\mathrm{g}} = 0.0000852529\,\mathrm{\frac{mol}{g}}\\
    \Delta T &= 1860\,\mathrm{\frac{K \cdot g}{mol}} \cdot 0.0000852529 \,\mathrm{\frac{mol}{g}} = 0.1586\,\mathrm{K}\\
    T_{\text{Gefrierpunkt}} &= 273.15\,\mathrm{K} - \Delta T = 272.99\,\mathrm{K}\\s
\end{align*}
b)
\begin{align*}
    \ce{C6H5COOH &<-> C6H5Coo^{-} + H^{+}}\\
    \Delta T &= bK_{Kr} = \frac{nK_{Kr}}{m_w}\\
    n &= \frac{\Delta T m_w}{K_{Kr}}\\
    n_0 &= \frac{m_{BS}}{M_{BS}}\\
    \alpha &= \frac{n_{Diss}}{n_0} =  \frac{n-n_0}{n_0}\\
    \ce{HA &<-> A^- + H^+}\\
    n_{\ce{H^+}} &= n_{\ce{A^-}} = \alpha n_0\\
    n_{\ce{HA}} &= n_0 - \alpha n_0 = (1-\alpha)n_0\\
    n &= n_{\ce{HA}} + n_{\ce{H^+}} + n_{\ce{A^-}} = (1-\alpha) n_0 + 2 \alpha n_0 = (1 + \alpha) n_0\\
    1 + \alpha &= \frac{n}{n_0}\\
    alpha &= \frac{n}{n_0}-1 = \frac{n-n_0}{n_0}\\
    \alpha &= \frac{n}{n_0} - 1 = \frac{\Delta T m_w}{K_{Kr} \frac{m_{BS}}{M_{BS}}} - 1 = 0.093 = 9.3 \%
\end{align*}
\subsection*{7.3}
\begin{align*}
    \mu^*_A(s) &= \mu_A(l) = \mu^*_A(l) + RT \ln x_A\\
    \mu^*_A(s) &= \mu^*_A(l) +RT\ln x_A\\
    \ln x_A &= \frac{\mu^*_A(s) - \mu^*_A(l)}{RT} = \frac{\Delta_{\text{gefrieren}}G}{RT} = \frac{-\Delta_{\text{sm}} G }{RT}\\
    \frac{d}{dT}\ln x_A &= -\frac{1}{R}\frac{d}{dt}\left(\frac{\Delta_{\text{sm}} G}{T}\right)\\
    \text{Mit Gibbs-Helmholtz-Gleichung:}\\
    \frac{\partial}{\partial T} \left(\frac{G}{T}\right) &= -\frac{H}{T^2}\\
    \frac{d}{dT}\ln x_A &= -\frac{1}{R} \frac{-\Delta_{\text{sm}H}}{T^2}\\
    \frac{d}{dT}\ln x_A &= \frac{\Delta_{\text{sm}}H}{R} \frac{1}{T^2}\\
    \int_{x_A = 1}^{x_A}\,d\ln x_A &= \frac{\Delta_{\text{sm}}H}{R}\int_{T^*}^{T}\frac{dT}{T^2}\\
    \ln x_A - \ln (1) &= \frac{-\Delta_{sm}H}{R} \left(\frac{1}{T}-\frac{1}{T^\alpha}\right) = \frac{-\Delta_{\text{sm}}H}{R}\left(\frac{T^*-T}{TT^*}\right) = \frac{-\Delta_{\text{sm}}H\cdot \Delta T}{RT^2}\\
    \ln x_a &= \ln (1-x_b) \approx -x_B\\
    x_B &= \frac{\Delta_{\text{sm}}H\cdot \Delta T}{RT^2}\\
    x_B &= \frac{n_B}{n_B+n_A} \approx \frac{n_B}{n_A} = \frac{n_B}{\frac{m_A}{M_A}} = \frac{n_B M_A}{m_A}\\
    \frac{n_B M_A}{m_A} &= \frac{\Delta_{\text{sm}}H\cdot \Delta T}{RT^2}\\
    \Delta T &= \frac{RT^2 n_B M_A}{\Delta_{\text{sm}}H m_A} = K_{Kr}b\\
    K_{Kr} &= \frac{RT^2 M_A}{\Delta_{\text{sm}}H}
\end{align*}
\subsection*{7.4}
%\begin{align*}
%    x_j &= 0.2\\
%    x_k &= 0.3\\
%    x_i &= 0.5\\
%    \text{Rein:}\\
%    \mu^_i &= \mu^*_i + RT\ln \frac{p^*_i}{p^0}\\
%    \text{Gemisch:}\\
%    \mu_i &= \mu_i^0 + RT \ln \frac{p_i}{p^0}\\
%    \text{Gesamt:}\\
%    \mu_i &= \mu^*_i + RT\ln \frac{p_i}{p^*_i}\\
%    \mu_i &= \mu^*_i + RT\ln \frac{x_ip^*_i}{p^*_i}\\
%    \mu_i &= \mu^*_i - 0.6931 \cdot 8.3145\,\mathrm{\frac{J}{mol \cdot K}} \cdot 298\,\mathrm{K}\\
%    \mu_i &= \mu^*_i - 1717.35\,\mathrm{\frac{J}{mol}}\\
%    \Delta \mu_i &= \mu^*_i - \mu_i = \mu^*_i - (\mu^*_i - 1717.35\,\mathrm{\frac{J}{mol}}) = 1717.35\,\mathrm{\frac{J}{mol}}
%\end{align*}

\subsection*{7.5}
\begin{align*}
    n&=\frac{pV}{RT}=\frac{100000\,\mathrm{Pa}\cdot 0.0025\,\mathrm{m^3}}{8.3145\,\mathrm{\frac{J}{K\cdot mol}} \cdot 298.15\,\mathrm{K}} = 0.1\,\mathrm{mol}\\
    \Delta G &= n_{ges} \cdot RT(x_A\ln x_A + x_B \ln x_B) = -343.64\,\mathrm{J}\\
    \Delta S &= -nR(x_A\ln x_A + x_B \ln x_B) = 1.15\,\mathrm{\frac{J}{K}}\\
    \Delta_KS + \Delta_{mix}S = 0\\
    w &= -nRT\ln\frac{V_E}{V_A}\\
    T &= const.\\
    \Delta U &= q+w = 0\\
    q &= -w = nRT\ln\frac{V_E}{V_A}\\
    \Delta_K S &= \frac{q_{rev}}{T} = nR\ln\frac{V_E}{V_A} = -\Delta_{mix}S\\
    \ln \frac{V_E}{V_A} &= -\frac{\Delta_{mix S}}{nR}\\
    V_E &= V_A e^{-\frac{\Delta_{mix}S}{nR}} \approx 2.5\cdot 10^{-3}\,\mathrm{m^3}
\end{align*}

\section*{Woche 8}
\subsection*{8.1}
Für eine Reaktion 0. Ordnung müsste der Umsatz an Reaktanten zu Produkten unabhängig zu deren Konzentration sein.\\
Anhand der Betrachtung der ersten drei Werte ist dies nicht gegeben.\\
Verlust an Konzentration von $t = 0$ s zu $t = 2000$ s beträgt 0.24 $\mathrm{\frac{mol}{l}}$, wobei sie von $t = 2000$ zu $t = 4000$ auch denselben Konzentrationsanteil verlieren sollte, jedoch werden nur 0.19 $\mathrm{\frac{mol}{l}}$ verloren.\\
Um zu überprüfen ob es sich um einen linearen Zusammenhang handelt wird das Verhältnis von Konzentrationsänderung zur Ratenänderung betrachtet wird.\\
Konzentrationsänderung 1. Schritt: 0.24 $\mathrm{\frac{mol}{l}}$\\
Konzentrationsänderung 2. Schritt: 0.19 $\mathrm{\frac{mol}{l}}$\\
Rate Schritt 1: 0.00012 $\mathrm{\frac{mol}{l \dot s}}$\\
Rate Schritt 2: 0.000095 $\mathrm{\frac{mol}{l \cdot s}}$\\
Verhältnis $c = \frac{c_2}{c_1} = \frac{0.19}{0.24} = 0.7917$\\
Verhältnis Ratenänderung $\frac{r_2}{r_1} = \frac{0.000095}{0.00012} = 0.7917$\\
Da beide Verhältnisse dasselbe Ergebnis liefern ist davon auszugehen, dass es sich hierbei um eine Reaktion 1. Ordnung handelt.\\
Für die Bestimmung der Reaktionskonstante gilt somit:\\
\begin{align*}
    \ln \frac{[A]}{[A(t=0)]} &= -kt\\
    - \frac{ln \frac{[A]}{[A(t=0)]}}{t} &= k\\
    - \frac{ln \frac{0.86\,\mathrm{\frac{mol}{l}}}{1.10\,\mathrm{\frac{mol}{l}}}}{2000\,\mathrm{s}} &= k\\
    0.00012306653\,\mathrm{\frac{1}{s}} &= k
\end{align*}

\subsection*{8.2}
Geschwindigkeitskonstante:
\begin{align*}
    k &= - \frac{ln \frac{[A]}{[A(t=0)]}}{t}\\
    k &= - \frac{ln \frac{0.8\,\mathrm{\frac{mol}{l}}}{1\,\mathrm{\frac{mol}{l}}}}{10\,\mathrm{s}}\\
    k &= 0.0223\,\mathrm{\frac{1}{s}}
\end{align*}
Verbrauchte Konzentration zu $t= 40$ s:
\begin{align*}
    [A] &= [A(t=0)]\cdot e^{-kt}\\
    [A] &= 1\,\mathrm{\frac{mol}{l}} \cdot e^{40\,\mathrm{s} \cdot 0.0223\,\mathrm{\frac{1}{s}}}\\
    [A] &= 2.44\,\mathrm{\frac{mol}{l}}
\end{align*}
Lösung macht keinen Sinn, es kann nicht mehr Konzentration verbraucht werden als zu Beginn vorhanden ist...\\

\section*{Woche 9}
\subsection*{Aufgabe 1}
a)
\begin{align*}
    p_A &= x_Ap^*_A\\
    x_B^g &= \frac{p_B}{p}\\
    p_B &= x_B^lp^*_B\\
\end{align*}
\begin{center}
    \begin{tabular}{l l l l l l l l}
        \hline
        $x_{\ce{N2}^l}$ & 0.0 & 0.1 & 0.2 & 0.4 & 0.6 & 0.8 & 1.0 \\
        \hline
        $p_{\ce{N2}}$ & 0.0 & 0.36 & 0.72 & 1.44 & 2.16 & 2.88 & 3.6 \\
        $x_{\ce{O2}^l}$ & 1.0 & 0.9 & 0.8 & 0.6 & 0.4 & 0.2 & 0.0 \\
        $p_{\ce{O2}}$ & 1.0 & 0.9 & 0.8 & 0.6 & 0.4 & 0.2 & 0.0 \\
        $p$ & 1.0 & 1.26 & 1.52 & 2.04 & 2.56 & 3.08 & 3.6 \\ 
    \end{tabular}
\end{center}
b)
\begin{align*}
    p_A &= x^l_Ap^*_A = x^g_Ap\\
    x^l_Ap^*_A &= x^g_Ap\\
    x^l_Ap^*_A &= x^g_A(p_A+p_B)\\
    3.6x^l_{\ce{N2}} &= x^g_{\ce{N2}}(x^l_{\ce{N2}}+(1-x^l_{\ce{N2}})p^*_{\ce{O2}})\\
    3.6x^l_{\ce{N2}} &= x^g_{\ce{N2}}x^l_{\ce{N2}}+x^g_{\ce{N2}}-x^g_{\ce{N2}}x^l_{\ce{N2}}\\
    3.6x^l_{\ce{N2}} &= x^g_{\ce{N2}}
\end{align*}
c) Nicht verstanden

\subsection*{Aufgabe 2}
%\includegraphics{yep.png}

\subsection*{Aufgabe 3}
a)
\begin{align*}
    x_{Hex} &= 0.33\\
    x_{Hep} &= 0.67\\
    \Delta_{Misch}S_m &= -R(0.33\ln 0.33 + 0.67 \ln 0.67) = 5.27\,\mathrm{\frac{J}{mol \cdot K}}\\
    \Delta_{Misch}G_m &= -298\,\mathrm{K}\cdot 5.27\,\mathrm{\frac{J}{mol \cdot K}} = -1570.46\,\mathrm{\frac{J}{mol}}\\ 
\end{align*}
b)
\begin{align*}
    \frac{\partial \Delta_{Misch}S_m}{\partial x_{Hex}} &= -R\cdot (\ln x_{Hex}-\ln (1-x_{Hex})) = 0\\
    x_{Hep} &= 0.5
\end{align*}
Mischentropie maximal wenn in gleichen Teilen vermischt.\\
\begin{align*}
    x_{Hep} &= \frac{\frac{m_{Hep}}{M_{Hep}}}{\frac{m_{Hex}}{M_{Hex}}+\frac{m_{Hep}}{M_{Hep}}}\\
    m_{Hex} = 42.92\,\mathrm{g}
\end{align*}
Die Mischentropie wird somit maximal bei 50 g Heptan und 42.92 g Hexan.

\subsection*{Aufgabe 5}
%\includegraphics{yepp.png}

\section*{Woche 10}
\subsection*{Aufgabe 1}
a)
\begin{align*}
    \Delta_R G &= -RT\ln K\\
    -\frac{\Delta_R G}{RT} &= \ln K\\
    e^{-\frac{\Delta_R G}{RT}} &= K\\
    e^{-\frac{-2.5\,\mathrm{\frac{kJ}{mol}}}{0.0083145\,\mathrm{\frac{kJ}{mol\cdot K}}\cdot 523.15\,\mathrm{K}}} &= 0.5628
\end{align*}
b)
Anhand von Le Chatelier, dem Prinzip des geringsten Zwanges, wird die Reaktion in Richtung der geringeren Anzahl an mol Gas gehen bzw. der Seite auf der weniger Teilchen liegen, die zur Entstehung eines Druckes führen.

\subsection*{Aufgabe 2}
\begin{align*}
    \Delta G(\ce{NH3}) &= \Delta H - T \Delta S\\
    \Delta G(\ce{NH3}) &= -46.19\,\mathrm{\frac{kJ}{mol}} - 298.15\,\mathrm{K}\cdot 0.19251\,\mathrm{\frac{J}{mol\cdot K}}\\
    \Delta G(\ce{NH3}) &= -57.44\,\mathrm{\frac{kJ}{mol}}\\
    \Delta G(\ce{N2}) &= -57.10\,\mathrm{\frac{kJ}{mol}}\\
    \Delta G(\ce{H2}) &= -38.93\,\mathrm{\frac{kJ}{mol}}\\
    \alpha &= \frac{n_0 - n_{GGW}}{n_0}\\
    K &= \frac{\alpha^{\frac{3}{2}}p^{\frac{1}{2}}}{(1-\alpha)(2-\alpha)^\frac{1}{2} (p^0)^\frac{1}{2}}
\end{align*}

\subsection*{Aufgabe 4}
\begin{align*}
    \ln k_1 &= \ln A -\frac{E_{A}}{RT}\\
    (\ln A - \ln k_1) RT &= E_{A}\\
    I. (\ln A + 6.0281) 2271.11\,\mathrm{\frac{J}{mol}} &= E_{A}\\
    II. (\ln A + 3.0989) 2437.40\,\mathrm{\frac{J}{mol}} &= E_{A}\\
    III. (\ln A + 0.5604) 2603.77\,\mathrm{\frac{J}{mol}} &= E_{A}\\
    I. 2271.11\,\mathrm{\frac{J}{mol}}\ln A &= E_A - 13690.4782 \\
    II. 2437.40\,\mathrm{\frac{J}{mol}}\ln A &= E_A - 7553.2589\\
    II. 2603.77\,\mathrm{\frac{J}{mol}}\ln A &= E_A -  1459.1527
\end{align*}

\subsection*{Aufgabe 6}
\begin{align*}
    r_1 &= k[\ce{CH3CH3}]\\
    \frac{d[\ce{CH3}]\cdot}{dt} &= k[\ce{CH3}][\ce{CH3CH3}]\\
    r_3 &= k[\ce{CH3CH2}\cdot]\\
    \frac{d[\ce{H}\cdot]}{dt} &= k[\ce{H}\cdot][\ce{CH3CH3}]\\
    \frac{d[\ce{CH3CH2}\cdot]}{dt} &= -2k[\ce{C4H10}]^2
\end{align*}
\end{document}