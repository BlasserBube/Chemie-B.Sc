\documentclass{article}

\usepackage{amsmath}
\usepackage{mhchem}
\usepackage{chemfig}
\usepackage{xcolor}

\begin{document}
\section*{Aufgaben - Woche 1}
\subsection*{Aufgabe 1.1}
\begin{equation*}
    p = \frac{nRT}{V}; n=\frac{m}{M}
\end{equation*}
\begin{equation*}
    p = \frac{mRT}{MV}
\end{equation*}
\begin{equation*}
    M = \frac{mRT}{pV} = \frac{2.55 \mathrm{g}\cdot 373.15 \mathrm{K}\cdot\ 8.314 \mathrm{\frac{J}{mol\cdot K}}}{101325 \mathrm{kPa} }= 78.08 \mathrm{\frac{g}{mol}}
\end{equation*}
Für den Stoff mit der Formel \ce{C6H6} ergibt die Molmasse\\$M=6\cdot M(\ce{C})+ 6\cdot M(\ce{H}) = 78.08 \mathrm{\frac{g}{mol}}$ 

\subsection*{Aufgabe 1.2}
\textcolor{red}{ACHTUNG, dieser Teil der Aufgabe ist falsch.}\\
Für den Druck gilt:
\begin{equation*}
    p_H=\frac{nRT}{V}=\frac{ 2 \mathrm{mol}\cdot 273.1 \mathrm{K} \cdot 8.135 \mathrm{\frac{J}{mol\cdot K}}}{ 0.0224 \mathrm{m^3}} = 99187 \mathrm{\frac{N}{m}}
\end{equation*}
\begin{equation*}
    p_N= 198375,2 \mathrm{\frac{N}{m}}
\end{equation*}
Die Reaktion läuft nach folgender Gleichung ab:\\
\ce{3H2 + N2 -> NH3}\\
Somit ergibt der Druck nach der vollständigen Umsetzung:
\begin{equation*}
    p_{Ges}=\frac{3}{4}p_H + \frac{1}{4}p_N = 198375,2 \frac{\mathrm{N}}{\mathrm{m}}
\end{equation*}
\\
Korrekte Lösung:
\begin{center}
    \begin{tabular}{c c c c}
        \hline
        - & $n_H$ & $n_N$ $n_{NH3}$\\
        \hline
        vor: & 2 mol & 1 mol & -\\
        Reakt: & 2 mol & $\frac{2}{3}$ mol & $\frac{4}{3}$ mol\\
        nach: & - & $\frac{1}{3}$ mol & $\frac{4}{3}$ mol\\ 
        \hline
    \end{tabular}
\end{center}
\begin{equation*}
    n_{ges}=\frac{1}{3} + \frac{4}{3} = \frac{5}{3} ( \mathrm{mol})
\end{equation*}
\begin{equation*}
    p_{ges} = 1,69 \cdot 10^5 Pa
\end{equation*}
\begin{equation*}
    x_i = \frac{n_i}{n_{ges}}
\end{equation*}
\begin{equation*}
    x_H = 0; x_N = \frac{\frac{1}{3}}{\frac{5}{3}}; x_{NH3}=0,8
\end{equation*}
\begin{equation*}
    p_i = x_i \cdot p_{ges}
\end{equation*}
Damit berechnen.

\subsection*{Aufgabe 1.3}
a)
\begin{equation*}
    dp = \left(\frac{\partial p}{\partial V}\right)_T dV + \left(\frac{\partial p}{\partial T}\right)_V dT
\end{equation*}
b)
\begin{equation*}
    d^2p=\frac{2nR}{V^3}d^2V-2\frac{nR}{V^2}dTdV
\end{equation*}
Schwartzschen Satz beweisen:
\begin{equation*}
    \frac{\partial^2 p}{\partial V \partial T} = \frac{\partial^2 p}{\partial T \partial V}
\end{equation*}
\begin{equation*}
    -\frac{nr}{V^2} = -\frac{nr}{V^2} 
\end{equation*}

\subsection*{Aufgabe 1.4}
a)
\begin{equation*}
    \alpha = \frac{1}{V}\left(\frac{\partial V}{\partial T}\right)_p = \frac{1}{\frac{nRT}{p}} \cdot \frac{nR}{p} = \frac{1}{T}
\end{equation*}
\begin{equation*}
    \beta = \frac{1}{p}\left(\frac{\partial p}{\partial T}\right)_V = \frac{1}{\frac{nRT}{V}} \cdot \frac{nR}{V} = \frac{1}{T}
\end{equation*}
\begin{equation*}
    K = -\frac{1}{V}\left(\frac{\partial V}{\partial p}\right)_V = -\frac{1}{\frac{nRT}{V}} \cdot -\left(\frac{nRT}{p^2}\right) = \frac{1}{p}
\end{equation*}
b)
\begin{equation*}
    \alpha = \beta K p
\end{equation*}
\begin{equation*}
    \frac{1}{V}\left(\frac{\partial V}{\partial T}\right)_p = \left(\frac{\partial p}{\partial T}\right)_V\cdot \left(-\frac{1}{V}\right)\left(\frac{\partial V}{\partial p}\right)_T
\end{equation*}
\begin{equation*}
    \left(\frac{\partial V}{\partial T}\right)_p = -\left(\frac{\partial p}{\partial T}\right)_V \left(\frac{\partial V}{\partial P}\right)_T   |\cdot \left(\frac{\partial T}{\partial V}\right)_P
\end{equation*}
\begin{equation*}
    \left(\frac{\partial V}{\partial T}\right)_p \left(\frac{\partial T}{\partial V}\right)_p = -\left(\frac{\partial p}{\partial T}\right)_V \left(\frac{\partial V}{\partial P}\right)_T \left(\frac{\partial T}{\partial V}\right)_P
\end{equation*}
Da gilt:
\begin{equation*}
    \left(\frac{\partial V}{\partial T}\right)_p\left(\frac{\partial p}{\partial T}\right)_V\left(\frac{\partial V}{\partial p}\right)_T = -1 ; \left(\frac{\partial V}{\partial T}\right)_p \left(\frac{\partial T}{\partial V}\right)_p = 1
\end{equation*}
Ergibt dies:
\begin{equation*}
    1 = 1
\end{equation*}

\subsection*{Aufgabe 1.5}
\begin{equation*}
    \lambda = \frac{\langle v \rangle}{z_1}=\frac{1}{\sqrt{2}\sigma \frac{N}{V}}=\frac{1}{\sqrt{2}\sigma \frac{p}{K_BT}}
\end{equation*}
Auf diese GLeichung kommt man mit folgenden Umformungen:
\begin{equation*}
     z_1 = \sqrt{2} \langle v \rangle \sigma \frac{N}{V}
\end{equation*}
\begin{equation*}
    pV = nRT | n = \frac{N}{N_A}
\end{equation*}
\begin{equation*}
    pV = \frac{N}{N_A} RT = NK_BT
\end{equation*}
\begin{equation*}
    p = \frac{N}{V}K_BT
\end{equation*}
\begin{equation*}
    \frac{N}{V} = \frac{p}{K_BT}
\end{equation*}
Somit:
\begin{equation*}
    \lambda_N = 6.76 \cdot 10^{-5} \mathrm{m}
\end{equation*}

\subsection*{Aufgabe 1.6}
\begin{equation*}
    T_1 = 273.15 \, \mathrm{K}, T_2 = 373.15 \, \mathrm{K}
\end{equation*}
\begin{equation*}
    p_1 = p_2
\end{equation*}
\begin{equation*}
    \frac{n_1 R T_1}{V} = \frac{n_2 R T_2}{V}
\end{equation*}
\begin{equation*}
    n_1 T_1 = n_2 T_2
\end{equation*}
\begin{equation*}
    n_1 + n_2 = n = 2 \, \mathrm{mol}
\end{equation*}
\begin{equation*}
    n_1T_1=(2-n_1)T_2
\end{equation*}
\begin{equation*}
    n_1T_1=2T_2-n_1T_2
\end{equation*}
\begin{equation*}
    n_1(T_1+T_2)=2T_2
\end{equation*}
\begin{equation*}
    n_1\frac{2T_2}{T_1 + T_2} = 0.845 \, \mathrm{mol}
\end{equation*}
\begin{equation*}
    n_2 = 2 - n_1 = 1.155 \, \mathrm{mol}
\end{equation*}
\begin{equation*}
    p = \frac{n_1RT_1}{V} = 1.072 \cdot 10^5 \, \mathrm{Pa}
\end{equation*}

\subsection*{Aufgabe 1.7}
\begin{equation*}
    E_{pot} = 4\varepsilon\left(\left(\frac{r_0}{r}\right)^{12} - \left(\frac{r_0}{r}\right)^6\right)
\end{equation*}
\begin{equation*}
    F = \frac{dE_{pot}}{dr}=\left(\left(-12\cdot 4\varepsilon r_0^{12} \cdot r^{-13}\right)-\left(-6\cdot 4\varepsilon r_0^6r^{-7}\right)\right)
\end{equation*}
\begin{equation*}
    \frac{48\varepsilon r_0^{12}}{r^13} - \frac{24\varepsilon r_0^6}{r^7} = 0
\end{equation*}
damit:
\begin{equation*}
    0 = \frac{2 r_0^12}{r^13}-\frac{r_0^6}{r^2}
\end{equation*}
Damit:
\begin{equation*}
    r_0^6r^6 = 2r_0^12
\end{equation*}
\begin{equation*}
    r = \sqrt[6]{2}r_0
\end{equation*}
\begin{equation*}
    r^6=\frac{2r_0^{12}}{r_0^6}=2r_0^6
\end{equation*}
\begin{equation*}
    r=\sqrt[6]{2}r_0
\end{equation*}
\begin{equation*}
    E_{pot}=-\varepsilon
\end{equation*}
\end{document}