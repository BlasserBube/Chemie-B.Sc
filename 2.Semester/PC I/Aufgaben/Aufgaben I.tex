\documentclass{article}

\usepackage{amsmath}
\usepackage{mhchem}
\usepackage{chemfig}
\usepackage{xcolor}
\usepackage[
  left=1cm,
  right=1cm,
  top=2cm,
  bottom=2cm,
]{geometry}

\begin{document}
\section*{Aufgaben - Woche 1}
\subsection*{Aufgabe 1.1}
\begin{equation*}
    p = \frac{nRT}{V}; n=\frac{m}{M}
\end{equation*}
\begin{equation*}
    p = \frac{mRT}{MV}
\end{equation*}
\begin{equation*}
    M = \frac{mRT}{pV} = \frac{2.55 \mathrm{g}\cdot 373.15 \mathrm{K}\cdot\ 8.314 \mathrm{\frac{J}{mol\cdot K}}}{101325 \mathrm{kPa} }= 78.08 \mathrm{\frac{g}{mol}}
\end{equation*}
Für den Stoff mit der Formel \ce{C6H6} ergibt die Molmasse\\$M=6\cdot M(\ce{C})+ 6\cdot M(\ce{H}) = 78.08 \mathrm{\frac{g}{mol}}$ 

\subsection*{Aufgabe 1.2}
\textcolor{red}{ACHTUNG, dieser Teil der Aufgabe ist falsch.}\\
Für den Druck gilt:
\begin{equation*}
    p_H=\frac{nRT}{V}=\frac{ 2 \mathrm{mol}\cdot 273.1 \mathrm{K} \cdot 8.135 \mathrm{\frac{J}{mol\cdot K}}}{ 0.0224 \mathrm{m^3}} = 99187 \mathrm{\frac{N}{m}}
\end{equation*}
\begin{equation*}
    p_N= 198375,2 \mathrm{\frac{N}{m}}
\end{equation*}
Die Reaktion läuft nach folgender Gleichung ab:\\
\ce{3H2 + N2 -> NH3}\\
Somit ergibt der Druck nach der vollständigen Umsetzung:
\begin{equation*}
    p_{Ges}=\frac{3}{4}p_H + \frac{1}{4}p_N = 198375,2 \frac{\mathrm{N}}{\mathrm{m}}
\end{equation*}
\\
Korrekte Lösung:
\begin{center}
    \begin{tabular}{c c c c}
        \hline
        - & $n_H$ & $n_N$ $n_{NH3}$\\
        \hline
        vor: & 2 mol & 1 mol & -\\
        Reakt: & 2 mol & $\frac{2}{3}$ mol & $\frac{4}{3}$ mol\\
        nach: & - & $\frac{1}{3}$ mol & $\frac{4}{3}$ mol\\ 
        \hline
    \end{tabular}
\end{center}
\begin{equation*}
    n_{ges}=\frac{1}{3} + \frac{4}{3} = \frac{5}{3} ( \mathrm{mol})
\end{equation*}
\begin{equation*}
    p_{ges} = 1,69 \cdot 10^5 Pa
\end{equation*}
\begin{equation*}
    x_i = \frac{n_i}{n_{ges}}
\end{equation*}
\begin{equation*}
    x_H = 0; x_N = \frac{\frac{1}{3}}{\frac{5}{3}}; x_{NH3}=0,8
\end{equation*}
\begin{equation*}
    p_i = x_i \cdot p_{ges}
\end{equation*}
Damit berechnen.

\subsection*{Aufgabe 1.3}
a)
\begin{equation*}
    dp = \left(\frac{\partial p}{\partial V}\right)_T dV + \left(\frac{\partial p}{\partial T}\right)_V dT
\end{equation*}
b)
\begin{equation*}
    d^2p=\frac{2nR}{V^3}d^2V-2\frac{nR}{V^2}dTdV
\end{equation*}
Schwartzschen Satz beweisen:
\begin{equation*}
    \frac{\partial^2 p}{\partial V \partial T} = \frac{\partial^2 p}{\partial T \partial V}
\end{equation*}
\begin{equation*}
    -\frac{nr}{V^2} = -\frac{nr}{V^2} 
\end{equation*}

\subsection*{Aufgabe 1.4}
a)
\begin{equation*}
    \alpha = \frac{1}{V}\left(\frac{\partial V}{\partial T}\right)_p = \frac{1}{\frac{nRT}{p}} \cdot \frac{nR}{p} = \frac{1}{T}
\end{equation*}
\begin{equation*}
    \beta = \frac{1}{p}\left(\frac{\partial p}{\partial T}\right)_V = \frac{1}{\frac{nRT}{V}} \cdot \frac{nR}{V} = \frac{1}{T}
\end{equation*}
\begin{equation*}
    K = -\frac{1}{V}\left(\frac{\partial V}{\partial p}\right)_V = -\frac{1}{\frac{nRT}{V}} \cdot -\left(\frac{nRT}{p^2}\right) = \frac{1}{p}
\end{equation*}
b)
\begin{equation*}
    \alpha = \beta K p
\end{equation*}
\begin{equation*}
    \frac{1}{V}\left(\frac{\partial V}{\partial T}\right)_p = \left(\frac{\partial p}{\partial T}\right)_V\cdot \left(-\frac{1}{V}\right)\left(\frac{\partial V}{\partial p}\right)_T
\end{equation*}
\begin{equation*}
    \left(\frac{\partial V}{\partial T}\right)_p = -\left(\frac{\partial p}{\partial T}\right)_V \left(\frac{\partial V}{\partial P}\right)_T   |\cdot \left(\frac{\partial T}{\partial V}\right)_P
\end{equation*}
\begin{equation*}
    \left(\frac{\partial V}{\partial T}\right)_p \left(\frac{\partial T}{\partial V}\right)_p = -\left(\frac{\partial p}{\partial T}\right)_V \left(\frac{\partial V}{\partial P}\right)_T \left(\frac{\partial T}{\partial V}\right)_P
\end{equation*}
Da gilt:
\begin{equation*}
    \left(\frac{\partial V}{\partial T}\right)_p\left(\frac{\partial p}{\partial T}\right)_V\left(\frac{\partial V}{\partial p}\right)_T = -1 ; \left(\frac{\partial V}{\partial T}\right)_p \left(\frac{\partial T}{\partial V}\right)_p = 1
\end{equation*}
Ergibt dies:
\begin{equation*}
    1 = 1
\end{equation*}

\subsection*{Aufgabe 1.5}
\begin{equation*}
    \lambda = \frac{\langle v \rangle}{z_1}=\frac{1}{\sqrt{2}\sigma \frac{N}{V}}=\frac{1}{\sqrt{2}\sigma \frac{p}{K_BT}}
\end{equation*}
Auf diese GLeichung kommt man mit folgenden Umformungen:
\begin{equation*}
     z_1 = \sqrt{2} \langle v \rangle \sigma \frac{N}{V}
\end{equation*}
\begin{equation*}
    pV = nRT | n = \frac{N}{N_A}
\end{equation*}
\begin{equation*}
    pV = \frac{N}{N_A} RT = NK_BT
\end{equation*}
\begin{equation*}
    p = \frac{N}{V}K_BT
\end{equation*}
\begin{equation*}
    \frac{N}{V} = \frac{p}{K_BT}
\end{equation*}
Somit:
\begin{equation*}
    \lambda_N = 6.76 \cdot 10^{-5} \mathrm{m}
\end{equation*}

\subsection*{Aufgabe 1.6}
\begin{equation*}
    T_1 = 273.15 \, \mathrm{K}, T_2 = 373.15 \, \mathrm{K}
\end{equation*}
\begin{equation*}
    p_1 = p_2
\end{equation*}
\begin{equation*}
    \frac{n_1 R T_1}{V} = \frac{n_2 R T_2}{V}
\end{equation*}
\begin{equation*}
    n_1 T_1 = n_2 T_2
\end{equation*}
\begin{equation*}
    n_1 + n_2 = n = 2 \, \mathrm{mol}
\end{equation*}
\begin{equation*}
    n_1T_1=(2-n_1)T_2
\end{equation*}
\begin{equation*}
    n_1T_1=2T_2-n_1T_2
\end{equation*}
\begin{equation*}
    n_1(T_1+T_2)=2T_2
\end{equation*}
\begin{equation*}
    n_1\frac{2T_2}{T_1 + T_2} = 0.845 \, \mathrm{mol}
\end{equation*}
\begin{equation*}
    n_2 = 2 - n_1 = 1.155 \, \mathrm{mol}
\end{equation*}
\begin{equation*}
    p = \frac{n_1RT_1}{V} = 1.072 \cdot 10^5 \, \mathrm{Pa}
\end{equation*}

\subsection*{Aufgabe 1.7}
\begin{equation*}
    E_{pot} = 4\varepsilon\left(\left(\frac{r_0}{r}\right)^{12} - \left(\frac{r_0}{r}\right)^6\right)
\end{equation*}
\begin{equation*}
    F = \frac{dE_{pot}}{dr}=\left(\left(-12\cdot 4\varepsilon r_0^{12} \cdot r^{-13}\right)-\left(-6\cdot 4\varepsilon r_0^6r^{-7}\right)\right)
\end{equation*}
\begin{equation*}
    \frac{48\varepsilon r_0^{12}}{r^13} - \frac{24\varepsilon r_0^6}{r^7} = 0
\end{equation*}
damit:
\begin{equation*}
    0 = \frac{2 r_0^12}{r^13}-\frac{r_0^6}{r^2}
\end{equation*}
Damit:
\begin{equation*}
    r_0^6r^6 = 2r_0^12
\end{equation*}
\begin{equation*}
    r = \sqrt[6]{2}r_0
\end{equation*}
\begin{equation*}
    r^6=\frac{2r_0^{12}}{r_0^6}=2r_0^6
\end{equation*}
\begin{equation*}
    r=\sqrt[6]{2}r_0
\end{equation*}
\begin{equation*}
    E_{pot}=-\varepsilon
\end{equation*}

\section*{Aufgaben - Woche 2}
\subsection*{Aufgabe 2.1}
Wird die Virialgleichung nach dem zweiten Glied abgebrochen lautet diese:
\begin{equation*}
    \frac{pV_m}{RT} = 1+B_pp
\end{equation*}
Mit den ersten Werten $p = 1.013$ bar und $pV_m = 22.693\,\mathrm{\frac{bar}{mol}}$, ergibt sich:
\begin{equation*}
    \frac{22.693\,\mathrm{\frac{bar}{mol}}}{8.3145\,\mathrm{\frac{J}{molK}}\,\cdot273\,\mathrm{K}} = 1 + B_p \cdot 1.013 \,\mathrm{bar} \rightarrow B_p = -0.977\,\mathrm{J}
\end{equation*}
\begin{center}
    \begin{tabular}{c c c}
        \hline
        $p$ [bar] & $pV_m$ [$\mathrm{\frac{bar}{mol}}$] & $B_p$ [J]\\
        \hline
        1.013&22.693&-0.9772\\
        3.039&22.673&-0.3256\\
        5.065&22.652&-0.1955\\
        \hline
    \end{tabular}
\end{center}
Mit der Gleichung
\begin{equation*}
    T_B \approx \frac{a}{bR}
\end{equation*}
mit den Van-der-Waals-Koeffizienten $a(\ce{N2}) = 140.8\cdot 10^{-3}\,\mathrm{\frac{Jm^3}{mol^2}}$ und $b(\ce{N2})=39.1\cdot 10^{-6}\,\mathrm{\frac{m^3}{mol}}$
\begin{equation*}
    T_B = 433.10\,\mathrm{K} < 273\,\mathrm{K}
\end{equation*}
Somit liegt die Messtemperatur unter der Boyletemperatur $T_B$.

\subsection*{Aufgabe 2.2}
\begin{equation*}
    p = \frac{nRT}{V-nb} - a \left(\frac{n}{V}\right)^2 = \frac{1\cdot 8.3145 \cdot 200}{0.005 - 1 \cdot 39.13 \cdot 10^{-6}}-140,8\cdot 10^{-3}\left(\frac{1}{0.005}\right)^2 \,\mathrm{bar} = 329.5740\,\mathrm{kbar}
\end{equation*}
\begin{equation*}
    p = \frac{nRT}{V} \rightarrow p = \frac{1\cdot 8.3145 \cdot 200}{0.005}\,\mathrm{bar} = 332.5800\,\mathrm{kbar}
\end{equation*}
Da der Druck nach der VdW Gleichung kleiner ist als nach der idealen Gasgleichung ist davon auszugehen, dass die anziehenden Kräfte zwischen den Molekülen überwiegt, dafür spricht auch das typische Verhalten bei kleinem $T$ und $B_p < 0$

\subsection*{Aufgabe 2.3}
a)
\begin{equation*}
    p = \frac{RT}{V_m-b}-\frac{a}{V_m^2}
\end{equation*}
\begin{equation*}
    \frac{\partial p}{\partial V_m} = 0, \, \frac{\partial^2 p}{partial V_m^2}=0
\end{equation*}
\begin{equation*}
    \frac{\partial p}{\partial V_m} = \frac{RT}{(V_m-b)^2}+\frac{2a}{V_m^3}=0 \Rightarrow \frac{2}{(V_m-b)}\frac{2a}{V_m^3} = \frac{6a}{V_m^4}
\end{equation*}
\begin{equation*}
    \frac{\partial^2 p}{\partial V_m^2} = \frac{2RT}{(V_m-b)^3}-\frac{6a}{Vm^4}=0 \Rightarrow \frac{2}{V_m-b}=\frac{3}{V_m}
\end{equation*}
\begin{equation*}
    V_{m,krit} = 3b
\end{equation*}
Daraus folgt:
\begin{equation*}
    T_{krit} = \frac{2a(V_m-b)^2}{V_m^3\cdot R}=\frac{8a}{27Rb}
\end{equation*}
\begin{equation*}
    p_{krit} = \frac{\frac{8a}{b}}{3b-b}-\frac{a}{9b^2} = \frac{a}{27b^2}
\end{equation*}
b)
\begin{equation*}
    T_{krit} = 304,01 \,\mathrm{K}
\end{equation*}
c)
\begin{equation*}
    \left(p_r\frac{3}{V_r^2}\right) = \frac{\frac{8T_r}{3}}{\left(V_r-\frac{1}{3}\right)}
\end{equation*}
\begin{equation*}
    \left(p_r\frac{3}{V_r^2}\right)\left(V_r-\frac{1}{3}\right) = \frac{8}{3}T_r
\end{equation*}

\subsection*{Aufgabe 2.4}
\begin{equation*}
    p = \frac{RT}{Vm}-\frac{B}{V_m^2}+\frac{C}{V_m^3}
\end{equation*}
\begin{eqnarray*}
    p' = -RTV_m^{-2}+2BV_m^{-3}-3CV_m^{-4} = 0\\
    p'' = 2RTV_m^{-3}-6BV_m^{-4}+12CV_m^{-5} = 0
\end{eqnarray*}
Die beiden miteinander verrechnet ergibt:
\begin{eqnarray*}
    (4B-6B)V_m+12C-6C = 0\\
    -2BV_m+6C=0\\
    V_{m,krit} = \frac{3C}{B}
\end{eqnarray*}

\subsection*{Aufgabe 2.5}
\begin{eqnarray*}
    \frac{pV_m}{RT}=1+B_pp+C_pp^2\\
    \frac{pV_m}{RT}=1+\frac{B_V}{V_m}+\frac{C_V}{V_m^2}
\end{eqnarray*}
Die zweite Gleichung nach p umgestellt ergibt:
\begin{equation*}
    p = \frac{RT}{V_m}+\frac{B_VRT}{V_m}+\frac{C_VRT}{V_m^3}
\end{equation*}
\begin{equation*}
    \frac{pV_m}{RT}=1+\frac{B_pRT}{V_m}+\frac{B_pB_VRT}{V_m^2}+\frac{B_pC_VRT}{V_m^3}+C_p\left(\frac{RT}{V_m}+\frac{B_VRT}{V_m^2}+\frac{C_VRT}{V_m^3}\right)\cdot\left(\frac{RT}{V_m}+\frac{B_VRT}{V_m^2}+\frac{C_VRT}{V_m^3}\right)
\end{equation*}
\begin{eqnarray*}
    \frac{pV_m}{RT}\approx 1 + \frac{B_pRT}{V_m}+\frac{B_pB_vRT}{V_m^2}+\frac{C_p(RT)^2}{V_m^2}\\
    =1+\frac{B_pRT}{V_m}+\frac{RTB_pB_V+C_p(RT)^2}{V_m^2}
\end{eqnarray*}
Somit ist $\frac{B_pRT}{V_m} = B_V$ und $\frac{RTB_pB_V+C_p(RT)^2}{V_m^2} = C_V$
\begin{eqnarray*}
    = B_V^2+C_p(RT)^2\\
    C_V - B_V^2 = C_p(RT)^2\\
    C_p = \frac{C_V-B_V^2}{(RT)^2}
\end{eqnarray*}

\subsection*{Aufgabe 2.6}
a)
\begin{equation*}
    w = -\int_{V_A}^{V_E} p\,dV = -nRT\ln\left(\frac{V_E}{V_A}\right) = -1\cdot 8.3145 \cdot 273 \cdot \ln\left(\frac{0.0448}{0.0224}\right) \,\mathrm{J}= -1573,3460 \,\mathrm{J}
\end{equation*}
\\b)
\begin{equation*}
    -p_{ex}\Delta V = -\frac{RT}{V} \cdot 0.0224\,\mathrm{m^3}=-1135.5528\,\mathrm{J}
\end{equation*}
\\c)
\begin{equation*}
    w = 0
\end{equation*}

\subsection*{Aufgabe 2.7}
\begin{eqnarray*}
    \partial w = -p\,dV\\
    dV = \left(\frac{\partial V}{\partial T}\right)_p\,dT + \left(\frac{\partial V}{\partial p}\right)_T\,dp,\,V=\frac{nRT}{p}\\
    \left(\frac{\partial V}{\partial T}\right)_p = \frac{nR}{p}\\
    \left(\frac{\partial V}{\partial p}\right)_T = \frac{-nRT}{p^2}\\
    \partial w = -p\frac{nR}{p}\,dT + p \frac{nRT}{p^2}dp = -nR\,dT + \frac{nRT}{p}\,dp\\
    \left(\frac{\partial w}{\partial T}\right)_p = -nR,\, \left(\frac{\partial w}{\partial p}\right)_T = \frac{nRT}{p}\\
    \left(\frac{\partial^2 w}{\partial T \partial p}\right) = \frac{\partial}{\partial p}\left(\frac{nRT}{p}\right)=\frac{nR}{p}\\
    \frac{\partial^2 w}{\partial p \partial T} = \frac{\partial}{\partial p}(-nR) = 0\\
    \frac{\partial^2 w}{\partial T \partial p} = \frac{\partial^2 w}{\partial p \partial T} \Rightarrow \partial w
\end{eqnarray*}
$w$ ist keine Zustandsgröße

\section*{Woche 3}
\subsection*{Augabe 3.1}
a)\begin{eqnarray*}
    V=\mathrm{const.},\,w=0\\
    q_v=nc_{m,V}\Delta T\\
    c_p=c_V+nR \rightarrow c_{v,m} = c_{p,m} - R\\
    q_V=(c_{p,m}-R)n\Delta T = 124.76\,\mathrm{J}
\end{eqnarray*}
Es wird keine Arbeit verrichtet\\
b)\begin{eqnarray*}
    p = \mathrm{const.}\\
    q_p=nc_{p,m}\Delta T = 207.9\,\mathrm{J}\\
\end{eqnarray*}
\begin{equation*}
    w=-p\Delta V \ce{->[{pV=nRT}]} -p(\frac{nRT_E}{p}-\frac{nRT_A}{p})=-nr\Delta T = -83.14\,\mathrm{J}
\end{equation*}

\subsection*{Aufgabe 3.2}
\begin{eqnarray*}
    q_K=C\Delta T,\,V=\mathrm{const.}\\
    q_V=\Delta U=-C\Delta T = n \Delta U_m = \frac{m}{M} \Delta U_m\\
    \Delta U_m = -\frac{C\Delta TM}{m}=-3261.96\,\mathrm{J}
\end{eqnarray*}
\begin{eqnarray*}
    H=U+pV\\
    \Delta H=\Delta U + \Delta (pV) = \Delta U + p \Delta V \rightarrow[{p\Delta V = \Delta n_gRT}]\rightarrow \Delta U + \Delta n_g RT\\
    \Delta n_g = (6-7.5)\,\mathrm{mol} = -1.5\,\mathrm{mol}\\
    \Delta H_m = -3265.7\cdot 10^3 \,\mathrm{\frac{J}{mol}}
\end{eqnarray*}

\subsection*{Aufgabe 3.3}
\begin{eqnarray*}
    \mu = \left(\frac{\partial T}{\partial p}\right)_H
    d H_m = \left(\frac{\partial H_m}{\partial T}\right)_p\,dT+\left(\frac{\partial H_m}{\partial p}\right)_T\,dp=C_{p,m}\,dT+\left(\frac{\partial H_m}{\partial p}\right)_T\,dp\\
    \text{mit: } \left(\frac{\partial H_m}{\partial p}\right)_T = -T\left(\frac{\partial V_m}{\partial T}\right)_p + V_m\\
    c_{p,m}\,dT+\left[-T\left(\frac{\partial V_m}{\partial T}\right)_p+V_m\right]\,dp=0\\
    \mu = \left(\frac{\partial T}{\partial p}\right)_H = \frac{1}{C_{p,m}}\left[T\left(\frac{\partial V_m}{\partial T}\right)_p - V_m\right]
\end{eqnarray*}
a)\begin{eqnarray*}
    pV_m-pb=RT\\
    V_m = \frac{RT}{p}+b\\
    \left(\frac{\partial V_m}{\partial T}\right)_p = \frac{R}{p}\\
    \mu = \frac{1}{c_{p,m}}\left[T\frac{R}{p}-\left(\frac{RT}{p}+b\right)\right]=-\frac{b}{c_{p,m}}<0\\
\end{eqnarray*}
b)\begin{eqnarray*}
    pV_m=RT+(b-\frac{a}{RT})_p\\
    V_m = \frac{RT}{p}+b-\frac{a}{RT}\\
    \left(\frac{\partial V_m}{\partial T}\right)_p = \frac{R}{p}+\frac{a}{RT^2}\\
    \mu = \frac{1}{c_{p,m}}\left[-\left(\frac{R}{p}+\frac{a}{RT^2}\right)-\left(\frac{RT}{p}+b-\frac{a}{RT}\right)\right]\\
    = \frac{1}{c_{p,m}}\left(\frac{2a}{RT}-b\right)
\end{eqnarray*}
\subsection*{Aufgabe 3.4}
\begin{center}
    I. \ce{(Cyclohexan) + 18 O2 -> 6CO2 + 6H2O}\\
    II. \ce{(Cyclohexen) + 16 O2 -> 6CO2 + 5H2O}\\
    III. \ce{(Cyclohexa-1,3-dien) + 14 O2 -> 6CO2 + 4H2O}\\
    IV. \ce{(Benzol) + 12 O2 -> 6CO2 + 3H2O}
\end{center}
I.
\begin{eqnarray*}
    6\cdot\Delta_BH^0_m(\ce{CO2})+6\cdot\Delta_BH^0_m(\ce{H2O})-\Delta_BH^0_m(\mathrm{Cyclohexan})\\
    6\cdot -395.5 + 6 \cdot -285.9 + 156.2\,\mathrm{\frac{kJ}{mol}} = -3932.2\,\mathrm{\frac{kJ}{mol}}\\
\end{eqnarray*}
Analog:\\
II. gegeben: $-3739.0\,\mathrm{\frac{kJ}{mol}}$\\\\
III $-3623.6\,\mathrm{\frac{kJ}{mol}}$\\\\
IV $-3279.74\,\mathrm{\frac{kJ}{mol}}$\\\\
Hydrierungsenthalpien:\\
II:
\begin{equation*}
    -3739 + 3932.2 \,\mathrm{\frac{kJ}{mol}}= 193.2\,\mathrm{\frac{kJ}{mol}}
\end{equation*}
III:
\begin{equation*}
    -3623.6 + 3739 \,\mathrm{\frac{kJ}{mol}}= 116\,\mathrm{\frac{kJ}{mol}}
\end{equation*}
IV:
\begin{equation*}
    -3279.74 + 3623.6 \,\mathrm{\frac{kJ}{mol}}= 343.86\,\mathrm{\frac{kJ}{mol}}
\end{equation*}
\subsection*{Aufgabe 3.5}
\begin{eqnarray*}
    \Delta_RH^0(600)=\Delta_RH^0(298.15)+(0.001(6\cdot 36 + 4\cdot 29 - 5 \cdot 29 - 4\cdot 42))\Delta T \,\mathrm{\frac{kJ}{mol}}= -904.6 \cdot 0.019 \cdot 301.85 \,\mathrm{\frac{kJ}{mol}} = -5188.0167\,\mathrm{\frac{kJ}{mol}}
\end{eqnarray*}
\end{document}