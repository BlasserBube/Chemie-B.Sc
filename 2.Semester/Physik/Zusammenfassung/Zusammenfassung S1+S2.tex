\documentclass{article}
\usepackage{amsmath}
\usepackage{amssymb}
\usepackage{hyperref}
\usepackage{graphicx}
\usepackage{xcolor}
\usepackage[
  left=1cm,
  right=1cm,
  top=2cm,
  bottom=2cm,
]{geometry}

\begin{document}

\section*{Physik Zusammenfassung}
\section{Basiswissen}
\subsection{Konstanten}
\begin{center}
  \begin{tabular}{c c} 
   \hline
   Konstantenname & Wert\& Einheit\\ 
   \hline
   Lichtgeschwindigkeit i Vakuum & $c=299792458\,\mathrm{\frac{m}{s}}$\\
   Planck-Konstante & $h=6,62607015\cdot 10^{-34}\,\mathrm{Js}$\\
   Elementarladung & $e=1,602176634\cdot 10^{-19}\,\mathrm{As}$\\
   Boltzmann-Konstante & $k_\mathrm{B}=1,380649\cdot 10^{-23}\,\mathrm{\frac{kg\cdot m^2}{s^2\cdot K}}$\\
   Avogrado-Konstante & $N_\mathrm{A}=6,02214076\cdot 10^{23}\,\mathrm{\frac{l}{mol}}$\\
   \hline
  \end{tabular}
\end{center}

\subsection{SI-Vorsätze}
\begin{center}
  \begin{tabular}{c c c c} 
   \hline
   Name & Zeichen & &\\ 
   \hline
   
   \hline
  \end{tabular}
\end{center}

\end{document}
