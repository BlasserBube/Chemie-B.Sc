\documentclass[a4paper]{article}
\usepackage{chemfig}
\usepackage{mhchem}
\usepackage{xcolor}
\usepackage{amsmath}
\usepackage{tikz}
\usepackage{chemfig}
\usepackage{amssymb}
\usepackage{hyperref}
\usepackage[
  left=1cm,
  right=1cm,
  top=2cm,
  bottom=2cm,
]{geometry}

\begin{document}
\section*{Aufgaben Sindlinger}
\subsection*{Aufgabe 1}
Elektronegativität ist ein Maß für die Fähigkeit eines Atoms in einer Verbindung Elektronen aus Bindungen zu sich zu ziehen.\\
Dieser Wert kann nicht gemessen werden aber es gibt unterschiedliche Zugänge zu EN die auf einigen experimentell bestimmten Größen basieren.\\
\subsection*{Aufgabe 2}
Ein Stoff mit größerer EN liegt in der Energieskala eines MO-Schemas weiter unten.
\subsection*{Aufgabe 3}
In einer Periode nehmen die Radien ab, in einer Gruppe zu.
Die Ionisierungsenergie nimmt in einer Periode zu, in einer Gruppe hingegen ab.
\subsection*{Aufgabe 4}
Ein \ce{H^-} ist größer als ein \ce{Br^-}, da das Verhältnis von Elektronen zu Protonen viel geringer ist. Wobei eher davon auszugehen ist, dass as \ce{Br^-} immernoch größer ist, da ein \ce{H^-} eigentlich nicht so groß sein sollte.
\subsection*{Aufgabe 5}
Finale Reaktionsgleichungen:\\
\ce{H2O -> H2 + 1/2 O2}\\
und\\
\ce{H^+_{(aq)} + Cl^-_{(aq)} -> 1/2 H2 + 1/2 Cl2}
\subsection*{Aufgabe 6}
\textbf{Steam-Reforming}\\
$\rightarrow$ fossile Wasserstoffverbindungen, Methan + höhere Aliphaten\\
\begin{center}
\ce{CH4 + H2= ->[{Ni-Katalysator}][{700-900 $^\circ$C}] 3H2 -> CO ->[{H2O}][{Cr/Fe-Katalysator}] 4H2 + CO2$\uparrow$}\\
\end{center}
Den ersten Schritt nennt man das Steam-Reforming, den zweiten die Wassergas-Shift-Reaktion.
\section*{Aufgaben Hartenbach}
\subsection*{Aufgabe 1}
Man weist den einzelnen Stoffen je eine Ecke zu \ce{CaF2} ionisch, \ce{Ca} metallisch \ce{F2} kovalent. \ce{SiO2} sollte dann zwischen ionisch und kovalent, aber eher auf der kovalenten Seite liegen.
\subsection*{Aufgabe 2}
\ce{Si} hat Halbleitereigenschaften, somit sind Valenz und Leitungsband nur etwas voneinander entfernt.\\
\ce{Mg} ist ein Metall und somit überlappen oder grenzen die beiden Bänder direkt aneinander.\\
\ce{Cl} ist ein Nichtleiter beziehungsweise Isolator, somit liegt ein großer Abstand zwischen den Bändern.
\subsection*{Aufgabe 3}
Bei Beryllium: 2s und 2p\\
Bei Calicum: 4s und 3d\\
\subsection*{Aufgabe 4}
Die Eckatome sind immer an 8 Elementarzellen beteiligt, somit immer nur zu $\frac{1}{8}$ an einer einzelnen beteiligt.\\
An den Flächen nur $\frac{1}{2}$.\\
Somit kommt man Aufgabe
\begin{equation*}
    \frac{1}{8} \cdot 8 + \frac{1}{2} \cdot 6 = 1 + 3 = 4
\end{equation*}
4 Atome pro Elementarzelle.\\
Der kürzeste Abstand liegt zwischen einem Atom einer Ecke und der Flächenmitte.
\begin{equation*}
    d = \frac{1}{2}\sqrt{2}a = \frac{\sqrt{2}a}{2} = 2r
\end{equation*}
\begin{equation*}
    r = \frac{\sqrt{2}a}{4} \approx 128.15 \mathrm{pm}
\end{equation*}
\subsection*{Aufgabe 5}
a)\\
Abfolge: A c B c A c B $\dots$\\\\
b)\\
Abfolge: A $\beta$ $\alpha$ B $\gamma$ $\beta$ C $\alpha$ $\gamma$ A $\dots$
\section{Aufgaben Blaschi}
\subsection*{Aufgabe 1}
Korrektur: $M(F) = 18.998$ $\mathrm{\frac{g}{mol}}$\\\\
a)\\$n = 3.7314$ mol\\$m = 544.98$ g\\
b)\\$n=0.4639$ mol\\$m = 67.75$ g\\$c=0.9278$ $\mathrm{\frac{g}{mol}}$\\
c)\\$p = 17.29$ bar
\subsection*{Aufgabe 2}
a)\\$M(\ce{K2Cr2O7})=294.181 \mathrm{\frac{g}{mol}}$\\$\omega(\ce{Cr})=0.3535$\\$M(\ce{(NH4)2Fe(SO4)2})=392.135 \mathrm{\frac{g}{mol}}$\\$\omega(\ce{Fe})=0.1424$\\$\omega(\ce{H2O})=0.2756$\\\\
b)\\$n(\ce{K2Cr2O7}) = 16.67$ mmol\\$m(\ce{K2Cr2O7}) = 4.9030$ g\\\\
c)\\\ce{Cr2O7^{2-} + 6 Fe^{3+} + 14 H^+ -> 2 Cr^{3+} + 6Fe^{3+} + 7H2O}
\newpage
\subsection*{Aufgabe 3}
a)\\$c=10.18 \mathrm{\frac{mol}{l}}$\\\\
b)\\Muss auf 100 ml aufgefüllt werden\\\\
c)\\$m = 31.25$ g
\subsection*{Aufgabe 4}

\section*{Aufgabenblatt 3}
\subsection*{Ingo}
\subsubsection*{1 Komplexnomeklatur}
a)\\
Pentacyanidonickelat(II)\\
Triaquadichloridofluoridoeisen (III)\\
Diammintetrahydroxidochromat(III)\\\\
b)\\
\ce{[Co(H2O)3(NH3)3]^{2+}}\\
\ce{[AuF6]^{-}}\\
\ce{[Cu(C2O4)2]^{2-}}

\subsubsection*{2 Kemplexisomerie}
Trigonalbipyramidal:\\
Oben und Unten - trans\\
Gleiche Liganen bilden ein "Face", eine Fläche - fac\\
Gleiche Liganden bilden einen Tielkreis - mer\\\\
Hexagonal:\\
3 mal - cis cis trans\\
1 mal - trans trans trans\\
2 mal - 4cis cis cis\\
6 Varianten insgesamt.

\subsubsection*{3 Anwendung der Kristallfeldtheorie - Spinell}
\begin{center}
    \begin{tabular}{l l l l}
        \hline
        \multicolumn{2}{l}{Normaler Spinell} & \multicolumn{2}{l}{Inverser Spinell}\\
        \hline
        \ce{Mn^{2+}} in TL & $2\cdot (-6) + 3 \cdot 4 = 0$ & \ce{Mn^{3+}} in TL & = $-1.\bar{7}$\\
        \ce{Mn^{3+}} in OL & $-6$ & \ce{Mn^{3+}} in OL & $-6$\\
        \ce{Mn^{3+}} in OL & $-6$ & \ce{Mn^{2+}} in OL & $0$\\
        Gesamt: & $-12$ & & $-7.\bar{7}$\\
    \end{tabular}
    $\rightarrow$ Spinell (verzerrt, da Orbitale ungleich besetzt, können nochmal aufgespalten werden)\\
\end{center}

\subsubsection*{Darstellungsverfahren}
Carbochlorierung:\\
\ce{TiO2 + 2C + 2 Cl2 -> TiCl4 + 2CO}\\
Kroll-Prozess:\\
\ce{TiCl4 + 2Mg -> Ti + 2MgCl2}\\
von Arhd-deBoer Prozess:\\
\ce{Ti + 2I2 <->[{T1}][{T2}] TiI4}\\
$T_2 > T_1$

\subsubsection*{Metall-Metall-Mehrfachbindung}
\end{document}