\documentclass[12pt]{scrartcl} %Dokumentklasse, bestimmt Teile des Layouts
\usepackage[utf8]{inputenc} %Ermöglicht Unicode, absolut notwendig
\usepackage[ngerman]{babel} %Ermöglicht deutsche Sonderzeichen, z. B. Umlaute
\usepackage[T1]{fontenc} %Ermöglicht Unicode, absolut notwendig
\usepackage{amsmath} %Wichtige Mathe-Möglichkeiten
\usepackage{amssymb} %Wichtige Mathe-Möglichkeiten
\usepackage{epstopdf} %Um .eps-Bilder in Latex verwenden zu können
\usepackage[onehalfspacing]{setspace} %Bestimmt den Zeilenabstand, hier 1,5
\usepackage{graphicx} %Bessere Bilder
\usepackage{array} %Bessere Formatierung / Umgebung für mehrzeilige Formeln
\usepackage{upgreek} %Für nicht kursive griechische Buchstaben, z. B. \upmu statt \mu
\usepackage{float} %Großes H zum fixieren von Abbildungen/Tabellen
%\usepackage{csquotes}% Anführungszeichen etc.
\usepackage{mhchem}% für Chemische Formeln und Reaktionsgleichungen
\usepackage{tikz} %Vektorgrafziken im LaTeX-eigenen Format, z. B. zum Export aus QtiPlot
\usepackage{graphicx}
\usepackage[font=small]{caption} %Bestimmt die Schriftgröße von Tabellenüberschriften und Bildunterschriften, optional
\usepackage[backend=biber, style=chem-angew]{biblatex} %Literaturverzeichnis
\usepackage{tabularx, booktabs, multirow} %Für bessere Tabellen und einfachen Excel to Latex import
\usepackage[
  left=3cm,
  right=2cm
]{geometry}
\usepackage{pgfplots}
\newcommand{\celsius}{^{\circ}\mathrm{C}} %Ermöglicht es, \celsius für die Einheit °C zu verwenden
\geometry{bottom=100pt} \geometry{top=100pt} %Definiert den oberen und unteren Rand, optional
\parindent0pt %Kein Einzug am Anfang von Absätzen
\sloppy %Besserer Blocksatz
\renewcaptionname{ngerman}{\figurename}{Abb.} %Umbenennung Abbildungen, optional
\renewcaptionname{ngerman}{\tablename}{Tab.} %Umbenennung Tabellen, optional
\begin{document}
\begin{titlepage}
\begin{center}
\vspace*{2cm}
\begin{LARGE}
\vspace*{1cm}
\textbf{\textsf{Quantitative Analyse - Aufgabe 3\\}}
\end{LARGE}
\vspace*{1cm}
\textbf{\textsf{Praktikum zur analystischen Chemie}}\\
\vspace*{1.5cm}
\begin{table}[H]
\sffamily
\hspace*{3cm}\begin{tabular}{>{\bfseries}l>{\bfseries}l}
Verfasser: Maxim Gilsendegen\\
Matrikelnummer: 3650677\\
E-Mail-Adresse: 182513@stud.uni-stuttgart.de\\
Assistent: Robert Stelzer\\
Abgabedatum: 19.07.2023\\
\end{tabular}
\end{table}
\end{center}
\end{titlepage}
\renewcommand{\thepage}{\Roman{page}}\setcounter{page}{1}
\tableofcontents %Generiert ein Inhaltsverzeichnis, optional
\newpage
\renewcommand{\thepage}{\arabic{page}}\setcounter{page}{1}

\section{Aufgabe}
Bestimmung der Stoffmenge von \ce{I^{-}} durch Fällung mit \ce{AgNO3}.\\
Die zu bestimmende Stoffmenge des Anions soll durch Reaktion zu einer schwerlöslichen Verbindung \ce{AgI} und folgender Wiegung rechnerisch bestimmt werden.

\section{Durchführung}
Für diese Titration ist die Lichtempfindlichkeit der \ce{AgNO3}-Lösung zu beachten, da durch Lichteinstrahlung der Titerfaktor vor und während der Titration verändert werden kann.\\
Es wurden fünf Aliquote Titriert, vier mal 10 ml und ein mal 25 ml, diese wurden jeweils in einen 250 ml Erlenmeyerkolben überführt und mit 50 ml demineralisiertem Wasser verdünnt. Nach Zugabe von zehn Tropfen einer 1\%-igen Eosin-Lösung als Farbindikator, wird zu einem Umschlag von orange zu pink titriert.

\section{Auswertug}
\ce{I^{-}} wird mit \ce{AgNO3} nach Reaktionsgleichung 1 gefällt.
\begin{equation}
    \ce{I^{-}_{(aq)} + Ag^{+}_{(aq)} -> AgI_{(s)}}
\end{equation}
In Tabelle 1 können die Volumina der Maßlösung abgelesen werde, die bis zum Umschlagspunkt in die \ce{I^{-}}-Lösung titriert wurden.
Aus diesen Volumina kann die Stoffmenge $n$ von \ce{I^{-}} berechnet werden, wobei für $c(\ce{AgNO3}) = 0.1004\,\mathrm{\frac{mol}{l}}$ gilt und $i$ die Zahl des Aliquoten angibt.
\begin{align*}
    n(\ce{I^{-}})_i &= V(\ce{AgNO3}) \cdot c(\ce{AgNO3}) \cdot \frac{100}{V_{\mathrm{Aliquot}(\text{in ml})}}\\
    &= 0.0031\,\mathrm{l} \cdot 0.1004\,\mathrm{\frac{mol}{l}} \cdot 10\\
    &= 0.0031124\,\mathrm{mol}\\
    &= 3.1124\,\mathrm{mmol}
\end{align*}
Analog dazu werden auch die anderen Aliquoten berechnet, dessen Stoffmengen in Tabelle 1 aufgeführt sind.\\
\newpage
\begin{center}
    Tab.1: Verbrauchte Volumina nach Aliquoten und deren berechneten Stoffmengen.\\
  \begin{tabular}{l l l l}
      \hline
      Aliquot & $V_{\mathrm{Aliquot}}$ [ml] & $\Delta V_{\text{Maßlösung}}$ [ml] & $n_{\ce{I^{-}}}$ [mmol]\\
      \hline
      1 & 10 & 3.1 & 3.1124\\
      2 & 10 & 3.05 & 3.0622\\
      3 & 10 & 3.05 & 3.0622\\
      4 & 10 & 3.05 & 3.0622\\
      5 & 25 & 7.65 & 3.07224\\
    \hline
  \end{tabular}
\end{center}

Der Mittelwert ergibt sich durch folgende Rechnung.
\begin{align*}
  n_{\mathrm{M}} &= \frac{\sum_{i = 1}^{5} n(\ce{I^{-}})_i}{5}\\
  &= \frac{3.1124\,\mathrm{mmol}+3.0622\,\mathrm{mmol}+3.0622\,\mathrm{mmol}+3.0622\,\mathrm{mmol}+3.07224\,\mathrm{mmol}}{5}\\
  &= 3.074248\,\mathrm{mmol}
\end{align*}

Somit wurde experimentell eine Stoffmenge von $n_{\mathrm{M}} = 3.074248\,\mathrm{mmol}$ bestimmt.

\section{Literatur}
[1] Skript zum Praktikum im Modul AC I: 19.07.2023
\end{document}
