\documentclass[12pt]{scrartcl} %Dokumentklasse, bestimmt Teile des Layouts
\usepackage[utf8]{inputenc} %Ermöglicht Unicode, absolut notwendig
\usepackage[ngerman]{babel} %Ermöglicht deutsche Sonderzeichen, z. B. Umlaute
\usepackage[T1]{fontenc} %Ermöglicht Unicode, absolut notwendig
\usepackage{amsmath} %Wichtige Mathe-Möglichkeiten
\usepackage{amssymb} %Wichtige Mathe-Möglichkeiten
\usepackage{epstopdf} %Um .eps-Bilder in Latex verwenden zu können
\usepackage[onehalfspacing]{setspace} %Bestimmt den Zeilenabstand, hier 1,5
\usepackage{graphicx} %Bessere Bilder
\usepackage{array} %Bessere Formatierung / Umgebung für mehrzeilige Formeln
\usepackage{upgreek} %Für nicht kursive griechische Buchstaben, z. B. \upmu statt \mu
\usepackage{float} %Großes H zum fixieren von Abbildungen/Tabellen
%\usepackage{csquotes}% Anführungszeichen etc.
\usepackage{mhchem}% für Chemische Formeln und Reaktionsgleichungen
\usepackage{tikz} %Vektorgrafziken im LaTeX-eigenen Format, z. B. zum Export aus QtiPlot
\usepackage{graphicx}
\usepackage[font=small]{caption} %Bestimmt die Schriftgröße von Tabellenüberschriften und Bildunterschriften, optional
\usepackage[backend=biber, style=chem-angew]{biblatex} %Literaturverzeichnis
\usepackage{tabularx, booktabs, multirow} %Für bessere Tabellen und einfachen Excel to Latex import
\usepackage[
  left=3cm,
  right=2cm
]{geometry}
\usepackage{pgfplots}
\newcommand{\celsius}{^{\circ}\mathrm{C}} %Ermöglicht es, \celsius für die Einheit °C zu verwenden
\geometry{bottom=100pt} \geometry{top=100pt} %Definiert den oberen und unteren Rand, optional
\parindent0pt %Kein Einzug am Anfang von Absätzen
\sloppy %Besserer Blocksatz
\renewcaptionname{ngerman}{\figurename}{Abb.} %Umbenennung Abbildungen, optional
\renewcaptionname{ngerman}{\tablename}{Tab.} %Umbenennung Tabellen, optional
\begin{document}
\begin{titlepage}
\begin{center}
\vspace*{2cm}
\begin{LARGE}
\vspace*{1cm}
\textbf{\textsf{Quantitative Analyse - Aufgabe 6\\}}
\end{LARGE}
\vspace*{1cm}
\textbf{\textsf{Praktikum zur analystischen Chemie}}\\
\vspace*{1.5cm}
\begin{table}[H]
\sffamily
\hspace*{3cm}\begin{tabular}{>{\bfseries}l>{\bfseries}l}
Verfasser: Maxim Gilsendegen\\
Matrikelnummer: 3650677\\
E-Mail-Adresse: 182513@stud.uni-stuttgart.de\\
Assistent: Robert Stelzer\\
Abgabedatum: 19.07.2023\\
\end{tabular}
\end{table}
\end{center}
\end{titlepage}
\renewcommand{\thepage}{\Roman{page}}\setcounter{page}{1}
\tableofcontents %Generiert ein Inhaltsverzeichnis, optional
\newpage
\renewcommand{\thepage}{\arabic{page}}\setcounter{page}{1}

\section{Aufgabe}
Bestimmung der Stoffmenge von \ce{Sb^{3+}} durch Titrieren mit \ce{KBrO3}.\\

\section{Durchführung}
Bevor der Analysekolben mit demineralisiertem Wasser auf 100 ml aufgefüllt wurde, wurden 25 ml halbkonzentrierte \ce{HCl}-Lösung hinzugegeben.\\
Es wurde mit einem 25 ml Aliquoten und zwei 10 ml Aliquoten titriert, diese wurden auf 100 ml verdünnt, mit 10 ml 7 M \ce{HCl} versetzt und auf 50$\celsius$ erhitzt.\\
Als Farbindikator wurden zwei Tropfen Methylorange hingegeben und mit 0.01682 M \ce{KBrO3}-Lösung titriert, bis die Lösung komplett klar wurde.


\section{Auswertug}
\ce{KBrO3} reagiert in saurer Lösung zu:
\begin{equation}
    \ce{BrO3^{-}_{(aq)} + 6 H^{+}_{(aq)} + 3 Sb^{3+}_{(aq)} -> 3 Sb^{5+}_{(aq)} + Br^{-}_{(aq)} + 3 H2O_{(l)}}
\end{equation}

Ist kein \ce{Sb^{3+}} mehr vorhanden um Reaktionsgleichung 1 zu ermöglichen, reagiert \ce{KBrO3} nach Reaktionsgleichung 2.
\begin{equation}
    \ce{BrO3^{-}_{(aq)} + 5 Br^{-}_{(aq)} + 6 H^{+}_{(aq)} -> 3 Br2_{(aq)} + 3 H2O_{(l)}}
\end{equation}
Durch das entstehende Brom würde sich die Lösung nach dem Äquivalenzpunkt leicht gelblich färben.
Die Volumina bis zum Äquivalenzpunkt sind in Tabelle 1 festgehalten.\\
Die Stoffmenge kann nach der folgenden Formel mit den Werten aus Tabelle 1 berechnet werden.
\begin{align*}
    n(\ce{KBrO3}) &= c(\ce{KBrO3}) \cdot \Delta V\\
    &= 0.01682\,\mathrm{\frac{mol}{l}} \cdot 0.01035\,\mathrm{l}\\
    &= 0.000174087\,\mathrm{mol}
\end{align*}
\newpage
Anhand von den Koeffizienten der Reaktanten aus Reaktionsgleichung 1, wird die Stoffmenge von \ce{Sb^{3+}} wie folgt bestimmt.
\begin{align*}
    n(\ce{Sb^{3+}}) &= 3 \cdot n(\ce{KBrO3}) \cdot \frac{100\,\mathrm{ml}}{V_{\mathrm{Aliquot}}}\\
    &= 3 \cdot 0.000174087\,\mathrm{mol} \cdot 4\\
    &= 0.0020890\,\mathrm{mol}\\
    &= 2.0890\,\mathrm{mmol}
\end{align*} 

Diese Berechnung wird analog für den Aliquoten 2 und 3 durchgeführt

\begin{center}
    Tab.1: Volumina an Maßlösung für die einzelnen Titrationen.
    \begin{tabular}{l l l l}
        \hline
        Aliquot & $V_{\mathrm{Aliquot}}$ [ml] & $\Delta V$ [ml] & $n$ [mmol]\\
        \hline
        1 & 25 & 10.35 & 2.0890\\
        2 & 10 & 4.3 & 2.1698\\
        3 & 10 & 4.65 & 2.3464\\
        \hline
    \end{tabular}
\end{center}

Ein Mittelwert für die Stoffmenge wird durch folgende Gleichung bestimmt, wobei $i$ für die Nummer des jeweiligen Aliquoten steht.
\begin{align*}
    n &= \frac{\sum_{i=1}^{3} n_i}{3}\\
    &= \frac{2.0890\,\mathrm{mmol} + 2.1698\,\mathrm{mmol} + 2.3464\,\mathrm{mmol}}{3}\\
    &= 2.2017\,\mathrm{mmol}
\end{align*}

Damit wurde eine Stoffmenge von $n = 2.2017\,\mathrm{mmol}$ experimentell bestimmt.

\section{Literatur}
[1] Skript zum Praktikum im Modul AC I: 19.07.2023
\end{document}
