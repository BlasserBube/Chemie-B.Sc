\documentclass[12pt]{scrartcl} %Dokumentklasse, bestimmt Teile des Layouts
\usepackage[utf8]{inputenc} %Ermöglicht Unicode, absolut notwendig
\usepackage[ngerman]{babel} %Ermöglicht deutsche Sonderzeichen, z. B. Umlaute
\usepackage[T1]{fontenc} %Ermöglicht Unicode, absolut notwendig
\usepackage{amsmath} %Wichtige Mathe-Möglichkeiten
\usepackage{amssymb} %Wichtige Mathe-Möglichkeiten
\usepackage{epstopdf} %Um .eps-Bilder in Latex verwenden zu können
\usepackage[onehalfspacing]{setspace} %Bestimmt den Zeilenabstand, hier 1,5
\usepackage{graphicx} %Bessere Bilder
\usepackage{array} %Bessere Formatierung / Umgebung für mehrzeilige Formeln
\usepackage{upgreek} %Für nicht kursive griechische Buchstaben, z. B. \upmu statt \mu
\usepackage{float} %Großes H zum fixieren von Abbildungen/Tabellen
%\usepackage{csquotes}% Anführungszeichen etc.
\usepackage{mhchem}% für Chemische Formeln und Reaktionsgleichungen
\usepackage{tikz} %Vektorgrafziken im LaTeX-eigenen Format, z. B. zum Export aus QtiPlot
\usepackage{graphicx}
\usepackage[font=small]{caption} %Bestimmt die Schriftgröße von Tabellenüberschriften und Bildunterschriften, optional
\usepackage[backend=biber, style=chem-angew]{biblatex} %Literaturverzeichnis
\usepackage{tabularx, booktabs, multirow} %Für bessere Tabellen und einfachen Excel to Latex import
\usepackage[
  left=3cm,
  right=2cm
]{geometry}
\usepackage{pgfplots}
\newcommand{\celsius}{^{\circ}\mathrm{C}} %Ermöglicht es, \celsius für die Einheit °C zu verwenden
\geometry{bottom=100pt} \geometry{top=100pt} %Definiert den oberen und unteren Rand, optional
\parindent0pt %Kein Einzug am Anfang von Absätzen
\sloppy %Besserer Blocksatz
\renewcaptionname{ngerman}{\figurename}{Abb.} %Umbenennung Abbildungen, optional
\renewcaptionname{ngerman}{\tablename}{Tab.} %Umbenennung Tabellen, optional
\begin{document}
\begin{titlepage}
\begin{center}
\vspace*{2cm}
\begin{LARGE}
\vspace*{1cm}
\textbf{\textsf{Quantitative Analyse - Aufgabe 1\\}}
\end{LARGE}
\vspace*{1cm}
\textbf{\textsf{Praktikum zur analystischen Chemie}}\\
\vspace*{1.5cm}
\begin{table}[H]
\sffamily
\hspace*{3cm}\begin{tabular}{>{\bfseries}l>{\bfseries}l}
Verfasser: Maxim Gilsendegen\\
Matrikelnummer: 3650677\\
E-Mail-Adresse: 182513@stud.uni-stuttgart.de\\
Assistent: Robert Stelzer\\
Abgabedatum: 19.07.2023\\
\end{tabular}
\end{table}
\end{center}
\end{titlepage}
\renewcommand{\thepage}{\Roman{page}}\setcounter{page}{1}
\tableofcontents %Generiert ein Inhaltsverzeichnis, optional
\newpage
\renewcommand{\thepage}{\arabic{page}}\setcounter{page}{1}

\section{Aufgabe}
Bestimmung der Stoffmenge von \ce{Ca^{2+}} durch Fällung zu Calciumoxalat Monohydrat \ce{Ca(C2O4)\cdot H2O}.\\
Die zu bestimmende Stoffmenge des Kations soll durch Reaktion zu einer schwerlöslichen Verbindung und folgender Wiegung rechnerisch bestimmt werden.

\section{Durchführung}
Jede der drei Aliquoten an je 25 ml Analyselösung wurde durch etwas Salzsäure angesäuert und mit demineralisiertem Wasser auf 100 ml aufgefüllt.\\
Nachdem die Lösung 10 Minuten auf einer Heizplatte gesiedet hat, wurde diese mit 15 ml an 0.5 M \ce{(NH4)2C2O4}-Lösung und einigen Tropfen Methylrot als Indikator versetzt.\\
Solange die Lösung noch warm blieb wurde 1 M \ce{NH3}-Lösung bis zum Farbwechsel von rot nach gelb unter ständigem Rühren hinzugetropft.\\
Um sicherzugehen, dass das Kation vollständig gefällt wurde, wird noch mit etwas \ce{(NH4)2C2O4}-Lösung nachgespült.\\
Nachdem die Lösung auf Zimmertemperatur abkühlen konnte, wurde mithilfe eines Gewichtskonstanten Tiegels abfiltriert, mit \ce{(NH4)2C2O4} nachgewaschen und mit demineralisiertem Wasser und Ethanol nachgespült.\\
Nach dem Trocknen bei 80$\celsius$ wurde der Feststoff als \ce{Ca(C2O4)\cdot H2O} ausgewogen und damit die Stoffmenge berechnet.\\

\newpage

\section{Auswertug}

Der Fällung von \ce{Ca^{2+}} liegt Reaktionsgleichung 1 zu Grunde.\\
\begin{equation}
  \ce{Ca^{2+}_{(aq)} + C2O4^{2-}_{(aq)} -> CaC2O4_{(s)} \cdot H2O}
\end{equation}

Die Tiegel wurden zu Beginn des Versuches auf Gewichtskonstanz gewogen.
\begin{center}
  Tab.1: Messwerte für die Gewichtsmessungen vor der Durchführung. Die Werte in eckigen Klammern wurden für den Mittelwert ausgelassen.\\
  \begin{tabular}{l l l l}
      \hline
      Messung & Tiegel 1 [g] & Tiegel 2 [g] & Tiegel 3 [g]\\
      \hline
      1.1&[29.4859]&29.4730&29.4726\\
      1.2&[29.4771]&29.4726&29.4729\\
      1.3&[29.4771]&29.4730&29.4722\\
      2.1&29.4788&29.4700&29.4726\\
      2.2&29.4786&29.4701&29.4720\\
      2.3&29.4785&29.4706&29.4729\\
      3.1&29.4780&29.4726&[29.4714]\\
      3.2&29.4784&29.4727&[29.4702]\\
      3.3&29.4784&29.4720&[29.4698]\\
      \hline
      Mittelwert&29.47845&29.47184&29.47253\\
      \hline
  \end{tabular}
\end{center}

Nach dem Ausfällen und Abfiltern des ausgefallenen Feststoffes wurden die Tiegel nochmal gewogen.
\begin{center}
  Tab.2: Messwerte des Gewichts der Tiegel mit dem Feststoff.\\
  \begin{tabular}{l l l l}
    \hline
    Messung & Tiegel 1 [g] & Tiegel 2 [g] & Tiegel 3 [g]\\
    \hline
    1&29.6055&30.2731&29.5967\\
    2&29.6057&30.2732&29.5968\\
    3&29.6053&30.2736&29.5973\\
    \hline
    Mittelwert&29.6055 & 30.2733 & 29.5969\\
    \hline
  \end{tabular}
\end{center}

Zur Bestimmung der Masse des Produktes werden nun die Mittelwerte der Messungen voneinander abgezogen.\\
\begin{align*}
  \Delta m &= m(\text{Mit Produkt}) - m(\text{Vor Beginn})\\
  &= 29.6055\,\mathrm{g} - 29.47845\,\mathrm{g}\\
  &= 0.12705\,\mathrm{g}
\end{align*}
Für die anderen Tiegel wird analog zu dieser Rechnung gerechnet.\\
Damit ergibt sich für $\Delta m$:\\
\begin{center}
  Tab.3: Massendifferenzen der Tiegel anhand der Mittelwerte aus Tabelle 1 und 2.\\
  \begin{tabular}{c c}
    \hline
    Tiegel&$\Delta m$ [g]\\
    \hline
    1&0.12705\\
    2&0.80146\\
    3&0.12437\\
    \hline
  \end{tabular}
\end{center}

Mit der berechneten Massendifferenz kann durch die zuvor zu bestimmende Molare Masse des ausgewogenen Stoffes dessen Stoffmenge berechnet werden.
Die molare Masse von \ce{Ca(C2O4)\cdot H2O} wird wie folgt berechnet:
\begin{align*}
  M(\ce{Ca(C2O4)\cdot H2O}) &= M(\ce{Ca}) + 2 M(\ce{C}) + 2 M(\ce{H}) + 5 M(\ce{O})\\
  &= 40.078\,\mathrm{\frac{g}{mol}} + 2\cdot  12.011\\,\mathrm{\frac{g}{mol}} + 2\cdot 1.008\,\mathrm{\frac{g}{mol}} + 5\cdot 16\,\mathrm{\frac{g}{mol}}\\
  &= 146.12\,\mathrm{\frac{g}{mol}}
\end{align*}
Nun kann die Stoffmenge berechnet werden.
\begin{align*}
  n(\ce{Ca(C2O4)\cdot H2O}) &= \frac{m}{M}\\
  &= \frac{0.12705\,\mathrm{g}}{146.12\,\mathrm{\frac{g}{mol}}}\\
  &= 0.000869491\,\mathrm{mol}
\end{align*}
Um nun von einem 25 ml Aliquoten auf die Gesamtstoffmenge der 100 ml zu kommen muss die berechnete Stoffmenge mit dem Faktor 4 multipliziert werden.
\begin{align*}
  n(\text(Gesamt)) &= 4 \cdot n(\text{25 ml  Aliquot})\\
  &= 5 \cdot 0.000869491\,\mathrm{mol}\\
  &= 0.00347796\,\mathrm{mol}\\
  &= 3.47796\,\mathrm{mmol}
\end{align*}
Analog dazu werden auch die anderen Aliquoten berechnet.\\
\begin{center}
  Tab.4: Berechnete Gesamtstoffmengen nach den Massendifferenzen der verschiedenen Tiegel. Für die Berechnung des Mittelwertes werden Werte in eckigen Klammern nicht miteinbezogen.\\
  \begin{tabular}{l l}
    \hline
    Tiegel & Gesamtstoffmenge $n$ [mmol]\\
    \hline
    1&3.47796\\
    2&[21.93978]\\
    3&3.40460\\
    \hline
    Mittelwert & 3.44128\\
    \hline
  \end{tabular}
\end{center}

Die gesamte Messung mit Tiegel 2 wurde verworfen, da hier eine sehr hohe Abweichung zu den anderen beiden Werten vorliegt. Die kam vermutlich zustande, da hier die Gefilterte Flüssigkeit nochmal durch denselben Tiegel abgefiltert wurde, da hier Feststoff gesehen werden konnte.\\
Somit ergibt sich für die Gesamtstoffmenge dieses Versuches $n(\ce{Ca^{2+}}) = 3.44128\,\mathrm{mmol}$

\section{Literatur}
[1] Skript zum Praktikum im Modul AC I: 19.06.2023
\end{document}
