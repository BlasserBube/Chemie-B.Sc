\documentclass{article}
\usepackage{chemfig}
\usepackage{mhchem}
\usepackage{xcolor}
\begin{document}
\section{Einleitung}
\subsection{Chemie der Nichtmetalle}
\subsubsection{Was ist ein Metall?}
\begin{itemize}
\item Duktil
\item Metallischer Glanz
\item Temperatur Abhängigkeit der elektrischen Leitfähigkeit
\item Metallatome geben leicht Elektronen ab
\item Zur erreichung der Edelgaskonfiguration
\item Kleine Ionisierungsenergien
\item Chemie dominiert von positiv geadenen Teilchen
\item Metalle haben kleine Elektronegativität
\end{itemize}
\subsection{Chemie der Elektronegativität der Elemente}
\subsubsection{Elektronegativität}
Sehr nützliches Konzept
Es ist keine experimentell observable
Pauling:
\begin{equation*}
\Delta D = D(\chemfig{A-B}) - \frac{1}{2}(D(\chemfig{A-A}) + D(\chemfig{B-B}))
\end{equation*}
$\Delta D$ Maß ionische Anteile der polaren Bindungen
\begin{equation*}
\Delta EN = \sqrt{\Delta D}
\end{equation*}
\begin{equation*}
0 \leq EN \geq 4
\end{equation*}
Metalle: $EN < 1.9(1)$
Nichtmetalle: $EN > 2.1(1)$
\subsection{Grundegende Konzepte/Bindungstheorie}
\subsubsection{Die unpolare kovalente Bindung}
Beispiel: MO-Diagramme
\subsubsection{Die polare kovalente Bindung}
Beispiel: MO-Schema für Fluor-Wasserstoff
\subsubsection{Atom/Kovalenzradien}
\chemfig{E-X}
\begin{itemize}
    \item größerer Radius bei X
\end{itemize}
\chemfig{E-E}
\begin{itemize}
    \item gleichgroß verteilt
\end{itemize}
\begin{equation*}
    r_{kov}(E)+r_{kov}(X)
\end{equation*}
\textbf{1.3.3.1 Kovalenzradientrends}\\
\begin{itemize}
    \item Trend 1: innerhalb einer Gruppe nimmt $r_{kov}$ zu 
    \item Trend 2: innerhalb einer Periode fällt der $r_{kov}$
    \item $Z_{\mathrm{eff}}$ steigt innerhalb einer Periode
    \item Valenzelektronen außen spüren mehr vom Kern 
    \begin{itemize}
        \item Stärkere Kontraktion
    \end{itemize}
\end{itemize}
2.Periode \ce{e-} können nur kleine Konzentrationszahlen \textcolor{gray}{(KZ)} realisieren\\
\ce{AlF6^{3-}}\hspace*{1cm} \ce{BF6^{3-}} gibt es nicht $->$ KZ = 4\\\\
Einschub $Z_{eff}$\\
Real gespürte elektrostatische Anziehung eines Valenzelektrons vom Kern $"Z"$\\
\begin{equation*}
    Z_{\mathrm{eff}} = Z-\sigma
\end{equation*}
\\
\textbf{1.3.3.2 Ionenradien}
\ce{H-}
$r_{ion}=207$ pm Pauling\\
$r_{ion}=139$ pm\\
$r_{ion}\approx 149 \pm 20$ pm\\\\
\textbf{1.3.3.3 Die Bindungsenergien von \chemfig{Element - Element} Einfachbindungen}\\
Freie Elektronenpaare die nahe zur Bindung liegen destabilisieren diese durch elektrostatische Wechselwirkungen.\\\\
\textbf{1.3.3.4 Ionisierungsenergien $\&$ Elektronenaffinitäten}\\
\chemfig{|E:} \ce{Ag -> Ag+ + e-} 5-25eV\\
\chemfig{EA}: \ce{Ag + e- -> Ag-}
\subsection{Der Wasserstoff}
\subsubsection{Allgemeines H hydrogenium}
hydro = Wasser\\
genium = erzeugen\\
\begin{itemize}
    \item Häufigstes Element Massenprozent 70$\%$
    \begin{itemize}
        \item Erdhydrosphäre 0.75$\%$
    \end{itemize}
\end{itemize}
3 Isotope:
\begin{itemize}
    \item \ce{^{1}_{1}H} $\approx$ 99.98$\%$ $r_{kov}=37$ pm
    \item \ce{^{2}_{1}H} = \ce{D} $\approx$ 0.02$\%$
    \item \ce{^{3}_{1}H} = \ce{T} $\tau_{\frac{1}{2}} = 12.5$ Jahre\\ \ce{^{14}N + n -> ^{R}C + ^{3}H / T}\\ \ce{^{6}Li + n -> ^{4}_{e}He + T + 5} MeV
\end{itemize}
\chemfig{H - H} $\Delta$ E = 440 $\frac{\mathrm{kJ}}{\mathrm{mol}}$ Bindungslänge: 74 pm\\
Smp 14K\\
Sdp 20K\\
1765 Cavendish:\\
\ce{2M + 2HCl -> H2 + 2MCl}
\subsubsection{Darstellung}
\ce{2HCl + Zn -> H2 + ZnCl2}\\
Elektrolyse von \ce{H2O} in verdünnten Säuren / Laugen\\
\begin{itemize}
    \item Kathode \ce{H+ + e- -> \frac{1}{2} H2}
    \item Anode \ce{H2O -> \frac{1}{2}O2 + 2e- + 2H+}
\end{itemize}
\subsubsection{großtechnische Produktion von \ce{H2}}

\subsubsection{Reaktivität von \ce{H2}}
\chemfig{H-H} ist sehr stabil, \ce{H2} ist reaktionsträge
Knallgasreaktion:\\
\ce{2H2 + O2 -> H2O + En}\\
Kettenreaktion:\\
Kettenstart \ce{H2 -> 2H$\cdot$}\\
\ce{H$\cdot$ + O2 -> HOO$\cdot$}\\
\ce{HOO$\cdot$ -> HO$\cdot$ + $\cdot$O$\cdot$}\\
\ce{HO$\cdot$ + H2 -> H2O + H$\cdot$}\\
\ce{H$\cdot$ + $\cdot$OH -> H2O}\\
\\\\
\subsubsection{Großtechnische Verfahren}
\begin{itemize}
    \item Elektrolyse
    \begin{itemize}
        \item zu wenig "überflüssige" Kapazität an Strom
        \item Wirkungsgrad \underline{noch} zu gering
        \item fossile \ce{H2}-Täger sind \underline{noch} zu günstig
    \end{itemize}
    \item Steam-Reforming
    \begin{itemize}
        \item[] \ce{CH4 + H2O ->[{\ce{N2} - Katalysator}][{700 - 800 $^\circ$C endotherm}] 3H2 + CO}
        \item[] \ce{CO + H2O ->[{exotherm}][{Wassergas-Stoff Reaktion Fe/Gr Kat.}] H2 + CO2$\downarrow$}
        \item 50 \% des Welt \ce{H2}-Prod kommt durch diese Reaktion
        \item 50/60 \% $\rightarrow$ Haber-Bosch-Verfahren
        \item Kuoerner-Verfahren \ce{C_nH_{2n+2} -> (n + 1)H2 + C}
    \end{itemize}
\end{itemize}
\subsubsection{Die Ionen des \ce{H2}}
\begin{itemize}
    \item[a)] das Hydrid \ce{H-}
    \begin{itemize}
        \item $r_{ion} \approx 149\pm 20\, pm$
        \item Darstellung
        \item[] \ce{2Na + H2 ->[{$\Delta T$}][{500 $^\circ C$}] 2NaH -> NaCl}
        \item[] \ce{Ca + H2 -> CaH2}
        \item \ce{H-} ist eine starke Base
        \item \ce{H-} sind potente Reduktionsmittel
        \item Komplexe Hydride:
        \item[] \ce{4NaH + AlCl3 -> 3 NaCl} - extrem instabil
        \begin{itemize}
            \item Komplexe \ce{H-}
            \begin{itemize}
                \item AC II Vorlesung
            \end{itemize}
        \end{itemize}
    \end{itemize}
    \item[b)] Das Proton \ce{H+}
    \begin{itemize}
        \item extrem instabil in freier Form nicht möglich
        \begin{itemize}
            \item Solvatationsentahlpie von \ce{H+} immens
            \item Triebkraft die berücksichtigt werden muss
        \end{itemize}
    \end{itemize}
\end{itemize}
\begin{center}
    \begin{tabular}{c c c c}
        \hline
        Proton & Oxonium-Ion & Zundel-Ion & Eigen-Ion\\
        \hline
        \ce{H+} & \ce{[H3O+]} & \ce{[H(OH2)2]+} & \ce{[(H3O)(H2O)3]+}\\
        \hline
    \end{tabular}
\end{center}
Einschub: WBB\\
$\hookrightarrow $ wichtige Art der zwischenmolekularen WW\\
VdW $\approx 0-20 \mathrm{\frac{kJ}{mol}}$\\
WB $\approx 20 \mathrm{\frac{kJ}{mol}} - 160\mathrm{\frac{kJ}{mol}}$\\
kovalente Bindungen $\approx 130 \mathrm{\frac{kJ}{mol}} - 580 \mathrm{\frac{kJ}{mol}}$\\
Polare \chemfig{E > H}\\
freie Elektronenpare im Molekül $E_2$:\\
\chemfig{E>H} $\cdots$ \ce{|E_2}

\subsection{Säure-Base-Chemie}
\subsection{SB-Chemie in Wasser}
\begin{itemize}
    \item[a)] \ce{H2O + H2O <=> [H3O]^+ + [OH^-]} $K_{Was} = 10^{-14}$
    \item[] in neutralem \ce{H2O}: \ce{c([H3O]^+_{(aq)}) = c([OH^-])_{(aq)} = 10^{-7} \frac{mol}{l}}
    \item[b)] \ce{HA + H2O <=> H^+_{(aq)} + A^-_{(aq)}}
    \item[] \ce{H2O} ist hier Reaktionspartner und Solvenz
\end{itemize}
\ce{H^+_{(g)} + H2O3 -> [H3O^+]_{(g)}} $\Delta G = -700 \mathrm{\frac{kJ}{mol}}$\\
\ce{[H3O^+]_{(g)} + xH2O -> [H3O]^+_{(aq)}} $\Delta G = -400 \mathrm{\frac{kJ}{mol}}$\\
\\
Einschub: Molare Grenzleitfähigkeit
\begin{itemize}
    \item Maß für die Leitfähigkeit von wässrigen Lösungen
    \item Maß für die Beweglichkeit der Ionen in Lösung
\end{itemize}
\begin{center}
    \begin{tabular}{c c c c c c}
        \hline
        - & \ce{H^+_{(aq)}} & \ce{Li^+_{(aq)}} & \ce{Na^+_{(aq)}} & \ce{K^+_{(aq)}} & \ce{Rb^+_{(aq)}}\\
        \hline
        $\Lambda$ & 350 & 35 & 50 & 74 & 78\\
        $r_{ion(aq)}$ & 450 & 300 & 200 & 150 & 130\\
        \hline
    \end{tabular}
\end{center}
Die Anomalie bei dem Radius des \ce{H^+_{(aq)}} und seine Leitfähigkeit, lässt sich mit dem Grotthuß-Effekt erklären, hierbei "klappt" das Proton zum nächsten Oxonium-Ion \ce{H3O^+_{(aq)}} und "springt" somit sehr schnell durch die Lösung.\\\\
\ce{Ha + H2O <=> [H^+_{(aq)}] + [A^-_{(aq)}]}\\\\
$K(\ce{HA}) = \frac{a\ce{[H^+_{(aq)}]} + \ce{[A^-_{(aq)}]}}{a\ce{(HA)}}$\\\\
$p\ce{H} = -\log a\ce{(H^+_{(aq)})}$\\\\
$p\ce{K\ce{(HA)}} = -\log K\ce{(HA)}$\\\\
$K_{\mathrm{S, Solv}} = \frac{a\ce{[H^+_{(aq)}]} + \ce{[A^-_{(aq)}]}}{a\ce{(HA)}}$\\\\
$pK_{\mathrm{S, Solv}} = -\log(K_{\mathrm{S, Solv}})$\\\\
$pH_{\mathrm{S, Solv}} = -\log(a_{\mathrm{S, Solv}})$\\\\

\subsubsection{Autoprotonolyse konstanten von verchiedenen protonischen Lösungsmitteln}
\ce{2H2O <=> H3O^+ + OH^-} $pK_w = pK_{an}$ = 14\\
\ce{2H2SO4 <=> H3SO4^+ + HSO4^-} $pK_an$ = 3.6\\
\ce{2AcOH <=> ACOH2^+ + ACO^-} $pK_an$ = 14.5\\
\ce{3HF <=> HFH^+ + FHF^- } $pK_an \approx$ 10\\
\ce{2NH3 <=> NH4^+ + NH2^-} $pK_an $= 30\\\\

\subsection{Die Halogene}
\begin{tabular}{c c c c c c}
    \hline\\
    \ce{F2} & \ce{Cl2} & \ce{Br2} & \ce{I2} & \ce{At2} & \ce{Ts2}\\
    \hline\\
    "Fluor" & "Chloros" & "Bromos" & "Iodeios" & "Astatos" & "Teness"\\
    von "Fluorit" & hellgrün & gestank & Veilchenblau & unbeständig & chemisch irrelevant\\
    \hline
\end{tabular}

\subsection{Struktur der Elemente/Halogene im Festkörper}
\subsection*{Polyhalogenide}
Triiodid\\
\ce{KI_{(aq)} + I2 -> K^+_{(aq)} + I3^-_{(aq)}}\\
Jedoch verstößt das Iod in einer linearen Darstellung gegen die Oktettregel\\
$\hookrightarrow$ Hier wäre das Iod "hypervalent"\\
Schreibt man die Iod-Atome nebeneinander, zwischen zweien eine einfach kovalente Bindung, so kann man mit einem "Klappodell" die Bindung näher verständlich machen.\\
Chlor kann auch Polyhalogenide bilden.\\
\ce{[AsPh4]^+ Cl^- ->[{Cl2}] [AsPh4]^+ [Cl3]^-}\\
Warum nicht:\\
\ce{NaCl + Cl2 -> NaCl3}?\\
Da NaCl ein viel stabileres Gitter bildet.\\
Brom:
\ce{[NPr4]Br ->[{4Br2}] [NPr4]^+[Br9]^-}\\
\subsection{Farbigkeit der Halogene MO-Diagramm der Halogene}
\begin{center}
    \begin{tabular}{c c c}
        \hline
        Halogen & Farbendruck & $\lambda$\\
        \hline
        \ce{F2} & schwach gelb & -\\
        \ce{Cl2} & gelbgrün & 330 nm\\
        \ce{Br2} & orangebraun & 430 nm\\
        \ce{I2} & violett & 540 nm\\
        \hline
    \end{tabular}
\end{center}
\subsection{Gewinnung, Darstellung + Verwendung der Halogene}
\subsubsection{Das Fluor}
$\hookrightarrow$ Vorkommen in Erdkruste ist häufig, \ce{CaF2} $\&$ \ce{Na3[AlF6]}\\
$\hookrightarrow$ 1886 Henry Moissau (1906 Nobelpreis)\\
\begin{itemize}
    \item Elektrolyse von \ce{KF$\cdot$ (HF) -> KHF2}
\end{itemize}
\subsubsection{Das Chlor}
Vorkommen: Meerwasser, Steinsalz, \ce{KCl} (Sylvin), \ce{KMgCl3 $\cdot$ H2O} Caballit\\
\end{document}