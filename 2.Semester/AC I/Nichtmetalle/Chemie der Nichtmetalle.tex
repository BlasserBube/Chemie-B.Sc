\documentclass{article}
\usepackage{chemfig}
\usepackage{mhchem}
\usepackage{xcolor}
\begin{document}
\section{Einleitung}
\subsection{Chemie der Nichtmetalle}
\subsubsection{Was ist ein Metall?}
\begin{itemize}
\item Duktil
\item Metallischer Glanz
\item Temperatur Abhängigkeit der elektrischen Leitfähigkeit
\item Metallatome geben leicht Elektronen ab
\item Zur erreichung der Edelgaskonfiguration
\item Kleine Ionisierungsenergien
\item Chemie dominiert von positiv geadenen Teilchen
\item Metalle haben kleine Elektronegativität
\end{itemize}
\subsection{Chemie der Elektronegativität der Elemente}
\subsubsection{Elektronegativität}
Sehr nützliches Konzept
Es ist keine experimentell observable
Pauling:
\begin{equation*}
\Delta D = D(\chemfig{A-B}) - \frac{1}{2}(D(\chemfig{A-A}) + D(\chemfig{B-B}))
\end{equation*}
$\Delta D$ Maß ionische Anteile der polaren Bindungen
\begin{equation*}
\Delta EN = \sqrt{\Delta D}
\end{equation*}
\begin{equation*}
0 \leq EN \geq 4
\end{equation*}
Metalle: $EN < 1.9(1)$
Nichtmetalle: $EN > 2.1(1)$
\subsection{Grundegende Konzepte/Bindungstheorie}
\subsubsection{Die unpolare kovalente Bindung}
Beispiel: MO-Diagramme
\subsubsection{Die polare kovalente Bindung}
Beispiel: MO-Schema für Fluor-Wasserstoff
\subsubsection{Atom/Kovalenzradien}
\chemfig{E-X}
\begin{itemize}
    \item größerer Radius bei X
\end{itemize}
\chemfig{E-E}
\begin{itemize}
    \item gleichgroß verteilt
\end{itemize}
\begin{equation*}
    r_{kov}(E)+r_{kov}(X)
\end{equation*}
\textbf{1.3.3.1 Kovalenzradientrends}\\
\begin{itemize}
    \item Trend 1: innerhalb einer Gruppe nimmt $r_{kov}$ zu 
    \item Trend 2: innerhalb einer Periode fällt der $r_{kov}$
    \item $Z_{\mathrm{eff}}$ steigt innerhalb einer Periode
    \item Valenzelektronen außen spüren mehr vom Kern 
    \begin{itemize}
        \item Stärkere Kontraktion
    \end{itemize}
\end{itemize}
2.Periode \ce{e-} können nur kleine Konzentrationszahlen \textcolor{gray}{(KZ)} realisieren\\
\ce{AlF6^{3-}}\hspace*{1cm} \ce{BF6^{3-}} gibt es nicht $->$ KZ = 4\\\\
Einschub $Z_{eff}$\\
Real gespürte elektrostatische Anziehung eines Valenzelektrons vom Kern $"Z"$\\
\begin{equation*}
    Z_{\mathrm{eff}} = Z-\sigma
\end{equation*}
\\
\textbf{1.3.3.2 Ionenradien}
\ce{H-}
$r_{ion}=207$ pm Pauling\\
$r_{ion}=139$ pm\\
$r_{ion}\approx 149 \pm 20$ pm\\\\
\textbf{1.3.3.3 Die Bindungsenergien von \chemfig{Element - Element} Einfachbindungen}\\
Freie Elektronenpaare die nahe zur Bindung liegen destabilisieren diese durch elektrostatische Wechselwirkungen.\\\\
\textbf{1.3.3.4 Ionisierungsenergien $\&$ Elektronenaffinitäten}\\
\chemfig{|E:} \ce{Ag -> Ag+ + e-} 5-25eV\\
\chemfig{EA}: \ce{Ag + e- -> Ag-}
\subsection{Der Wasserstoff}
\subsubsection{Allgemeines H hydrogenium}
hydro = Wasser\\
genium = erzeugen\\
\begin{itemize}
    \item Häufigstes Element Massenprozent 70$\%$
    \begin{itemize}
        \item Erdhydrosphäre 0.75$\%$
    \end{itemize}
\end{itemize}
3 Isotope:
\begin{itemize}
    \item \ce{^{1}_{1}H} $\approx$ 99.98$\%$ $r_{kov}=37$ pm
    \item \ce{^{2}_{1}H} = \ce{D} $\approx$ 0.02$\%$
    \item \ce{^{3}_{1}H} = \ce{T} $\tau_{\frac{1}{2}} = 12.5$ Jahre\\ \ce{^{14}N + n -> ^{R}C + ^{3}H / T}\\ \ce{^{6}Li + n -> ^{4}_{e}He + T + 5} MeV
\end{itemize}
\chemfig{H - H} $\Delta$ E = 440 $\frac{\mathrm{kJ}}{\mathrm{mol}}$ Bindungslänge: 74 pm\\
Smp 14K\\
Sdp 20K\\
1765 Cavendish:\\
\ce{2M + 2HCl -> H2 + 2MCl}
\subsubsection{Darstellung}
\ce{2HCl + Zn -> H2 + ZnCl2}\\
Elektrolyse von \ce{H2O} in verdünnten Säuren / Laugen\\
\begin{itemize}
    \item Kathode \ce{H+ + e- -> \frac{1}{2} H2}
    \item Anode \ce{H2O -> \frac{1}{2}O2 + 2e- + 2H+}
\end{itemize}
\subsubsection{großtechnische Produktion von \ce{H2}}

\subsubsection{Reaktivität von \ce{H2}}
\chemfig{H-H} ist sehr stabil, \ce{H2} ist reaktionsträge
Knallgasreaktion:\\
\ce{2H2 + O2 -> H2O}\\
Kettenreaktion:\\
Kettenstart \ce{H2 -> 2H$\cdot$}\\
\ce{H$\cdot$ + O2 -> HOO$\cdot$}\\
\ce{HOO$\cdot$ -> HO$\cdot$ + $\cdot$O$\cdot$}\\
\ce{HO$\cdot$ + H2 -> H2O + H$\cdot$}\\
\ce{H$\cdot$ + $\cdot$OH -> H2O}\\
\end{document}