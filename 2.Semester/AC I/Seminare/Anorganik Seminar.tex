\documentclass{article}
\usepackage{mhchem}
\begin{document}
\section{Qualitative Analyse}
\subsection{Wie sieht meine Probe aus?}
\begin{itemize}
    \item Welche Farben sind zu erkennen?
    \item Kann ich unter dem Mikroskop Kristallformen bestimmen?
\end{itemize}
\subsection{Vorproben}
\begin{itemize}
    \item Flammenfärbung
    \item Spektroskopie
    \item Borax- bzw. Phosphorsalzperle
    \item Glühröhrchen
    \item Marsche-Probe
\end{itemize}
\subsection{Kationenanalyse}
Immer erst kalt dann heiß:
\begin{itemize}
    \item Wasser
    \item Salzsäure (verd. \& konz.)
    \item Salpetersäure (komz.)
    \item Königswasser (ggf. erhitzen)
    \item Schwefelsäure (konz.)
    \item Ammoniak (konz.)
\end{itemize}
Was nicht in Lösung geht muss aufgeschlossen werden:
\begin{itemize}
    \item Soda-Pottasche Aufschluss\\Erdalkalimetallsulfate, Silberhalogenide und einige Oxide
    \item Oxidationsschmelze\\Schwerlösliche Chrom- ud Manganverbindungen
    \item Saurer Aufschluss\\Einige Oxide
    \item Freiberger Aufschluss\\\ce{SnO2}
\end{itemize}
\end{document}