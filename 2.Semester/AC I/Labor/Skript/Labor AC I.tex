\documentclass{article}

\usepackage{xcolor}
\usepackage{mhchem}
\usepackage{scrlayer-scrpage}

%%%%%%%%%%%%%%%%%%%%%%%%%%%%%%%%%%%%%%%%%%%%%%

\pagestyle{scrheadings}
\ihead{Maxim Gilsendegen}
\ohead{\textbf{Kolloquium $-$ Chemie-Bibel}} 
\cfoot{\thepage}

%%%%%%%%%%%%%%%%%%%%%%%%%%%%%%%%%%%%%%%%%%%%%%

\begin{document}

%%%%%%%%%%%%%%%%%%%%%%%%%%%%%%%%%%%%%%%%%%%%%%

\begin{titlepage}
    \author{Maxim Gilsendegen} 
    \title{\textbf{Die Chemie-Bibel}} 
    \date{April 2023}
    \maketitle
    \thispagestyle{empty}
    \end{titlepage}

\section{Kolloquium}

%%%%%%%%%%%%%%%%%%%%%%%%%%%%%%%%%%%%%%%%%%%%%%
%%%%%%%%%%%%%%%%%%%%%%%%%%%%%%%%%%%%%%%%%%%%%%

\subsection{Essentielles Vorwissen}
\subsubsection{Autoprotolyse des Wassers}
Wasser durchläuft konstant folgende Reaktion
\begin{equation}
    \ce{2H2O -> H3O+ + OH-}
\end{equation}
hierbei ist zu beachten, dass $c_{\ce{OH-}}=c_{\ce{H3O+}}=10^{-7}\frac{\mathrm{mol}}{\mathrm{l}}$, dadurch kommt auch der $p$H-Wert von Wasser.
\begin{equation}
    \ce{p\mathrm{H}=-\log_{10}(c)}=-\log_{10}(10^{-7})=7
\end{equation}
Somit ist der $p$H-Wert von Wasser $p$H(\ce{H2O})=7

%%%%%%%%%%%%%%%%%%%%%%%%%%%%%%%%%%%%%%%%%%%%%%

\subsection{Sichere Arbeitsweise im Labor}


\subsubsection{Schutzkleidung}
Das betreten des Labors ohne Schutzkleidung stellt eine große Gefahr da (siehe unter anderem (1.3) Besonders giftige Chemikalien) und kann somit zu einem sofortigen Ausschlus aus dem Labor als Konsequenz tragen.\\
Somit ist das tragen eines Laborkittels, einer langer Hose, welche die Beine komplett bedeckt, geschlossener Schuhe und einer Schutzbrille zu jeder Zeit verpflichtend. Alle getragenen Materialien sollten schwer brennbar sein und bei Arbeit im Labor nicht behindern.\\

\subsubsection{Laborplatz}
Vor dem Arbeiten am Laborplatz sollte sichergestellt sein, dass keine Reste von (unbekannten) Chemikalien auf der Arbeitsplatze oder in unmittelbarer Nähe sind.\\
Beim Reinigen des Arbeitsplatzes sollte vor allem bei unbekannten Chemikalien mit großer Vorsicht vorgegangen werden, da diese potentiell giftig, krebserregend oder erbgut verändernd sein könnten.\\
Demnach ist daraus auch zu schließen, dass der Arbeitsplatz immer sauber zu hinterlassen ist um Gefahren vorzubeugen und den anderen Laborierenden einen sicheren Arbeitsplatz zu bieten.\\
Beim Wahrnehmen von Gasen, anhand von Farbe oder Geruch, ist umgehend, sofern nicht auszuschließen ist, dass diese eine potentielle Gefahr sein könnten, die Gefahrenzone zu verlassen, das Einatmen der potentiell gefährlichen Gase zu vermeiden und einen Assistenten oder die Laborleitung darauf hinzuweisen.\\
Beim Arbeiten mit Gefahrstoffen sollte immer der Abzug verwendet werden, vor allem wenn eine Gasbildung mit potentieller giftigen Wirkung nicht auszuschließen ist.\\
Somit ist auch immer sicherzustellen, dass der Abzug angeschalten ist und ein Luftaustausch gesichert ist, hierzu sollte beim Betreten des Labors der Status des Lüftungssystems geprüft und dieses angeschalten werden, sofern dieses nicht bereits angeschalten ist.\\
Es sollte auch eine sichere Arbeitsatmosphäre für alle Laborierenden bestehen, somit ist immer volle Aufmerksamkeit erwartet und eine Behinderung anderer Laborierenden zu vermeiden.\\   

\subsubsection{Handhabung von Chemikalien}
Chemikalien sind immer in geschlossenen beziehungsweise nicht komplett gefüllten offenen Behältern zu transportieren. Beim Laufen mit diesen sollte aufmerksam vorgegangen werden, um einen Zusammenstoß mit anderen, das Stolpern und das Resultierende Verschütten der Chemikalien zu vermeiden.\\
Beim transportieren von Chemikalien in (Voll-/Tropf-)Pipetten soll diese immer auf den Boden zeigen, um das Tropfen auf den Boden zu verhindern, kann ein Uhrglas unter dieser gehalten werden. Die Pipette sollte niemals zur Decke gerichtet sein, da nicht auszuschließen ist, dass am Peleusball oder am Gummi-Aufsatz eine komplette Dichtung besteht, wodurch Chemikalien leicht auf die Haut geraten könnten.\\

\subsubsection{Verzehr von Genuss- und Lebensmitteln}
Jeglicher Konsum von Genuss- und Lebensmitteln ist im gesamten Labor aufgrund von Sicherheitsanweisungen verboten. Dazu gehört der Konsum von Wasser, jegliches Essen, das Rauchen und auch das kauen von Kaugummies, so wie der Konsum von Bonbons.

%%%%%%%%%%%%%%%%%%%%%%%%%%%%%%%%%%%%%%%%%%%%%%
\newpage
%%%%%%%%%%%%%%%%%%%%%%%%%%%%%%%%%%%%%%%%%%%%%%

\subsection{Besonders giftige Chemikalien}
Die folgenden Chemikalien sind mit Vorsicht und Respekt zu handhaben, eine Aussetzung in die Umwelt, vor allem in das Wasser, durch zum Beispiel den Ausguss, zu vermeiden.\\
Feststoffe sind in der Feststofftonne zu entsorgen.


\subsubsection{\ce{HCN} $-$ Cyanwasserstoff \textcolor{gray}{(Blausäure)}}
HCN, Cyanwasserstoff oder auch bekannt als Blausäure, ist hoch giftig für Menschen und kann bereits in einer kleinen Dosis tödlich sein.\\
Symptome einer Blausäuren-Vergiftung sind Schwindel, Atemnot, Unwohlsein und die Rötung der Adern.\\
Diese Rötung der Adern ist ein resultat der gehemmten Zellatmung, weshalb der Sauerstoff aus dem Eisen-Hämoglobin-Sauerstoff-Komplex nicht zur Zellatmung verwendet werden kann.\\
Der Eisen-Hämoglobin-Sauerstoff-Komplex ist rot gefärbt, weshalb sich diese Vergiftung durch die Färbung der Adern beobachten lässt.\\
Der Betroffene sollte umgehend an die frische Luft gebracht werden und es sollte ärztliche Hilfe geholt werden,  beziehungsweise die Giftnotrufzentrale angerufen werden.\\

\subsubsection{\ce{KCN} $-$ Kaliumcyanid}
Kaliumcyanid dissoziiert wie folgt:\\
\begin{equation}
    \ce{KCN -> K+ + CN-}
\end{equation}
Giftig hierbei sind sie \ce{CN-}-Ionen, die bei der Dissoziation entstehen, welches mit unter anderem mit Wasser \textcolor{gray}{(Autoprotolyse des Wassers)} bzw. \ce{H+}-Ionen basisch reagieren.\\
\begin{equation}
    \ce{CN- + H+ -> HCN}
\end{equation}

\subsubsection{\ce{K2Cr2O7} $-$ Kaliumdichromat}
Kaliumdichromat gilt als starkes Oxidationsmittel und beschleunigt somit Brände.\\
Kaliumdichromat wird hauptsächlich über die Atemwege aufgenommen und aknn damit die Schleimhäute schädigen.\\
Der Betroffene, sollte bei Unwohlsein oder längerer Exposition umgehen an die frische Luft gebracht werden. Wenn eine akute Lebensgefahr nicht auszuschließen ist, kann ein Arzt oder die Giftnotrufzentrale verständigt werden.\\

\subsubsection{\ce{K2CrO4} $-$ Kaliumchromat}
Kaliumchromat wird hauptsächlich über die Atemwege aufgenommen und wirkt Schleimhaut reizend und schädigend.\\
Auch durch längeren Hautkontakt kann Kaliumchromat schädigend wirken, was sich unter anderem in Nierenschädigungen äußert.\\

\subsubsection{\ce{Hg(NO3)2} $-$ Quecksilber(II)-Nitrat}
Quecksilber(II)-Nitrat reagiert in einer hydrolitischen Spaltung wie folgt
\begin{equation}
    \ce{Hg(NO3)2 + H2O -> HgO + 2 HNO3}
\end{equation}
\ce{HNO3} oder auch Salpetersäure genannt und \ce{HgO} oder auch Quecksilber(II)-Oxid entstehen.\\
Somit können Verätzungen auftreten, wobei bei Hautkontakt Rötungen der Haut entstehen, bei Aufnahme über die Atemwege kommt es zu starkem Husten.
Bei Verschlucken können abdominale Schmerzen und Durchfall ab einer bestimmten Dosis als Symptome auftreten.\\
Quecksilberverbindungen stehen im Verdacht chromosomenschädigend und kanzerogen zu wirken.\\

\subsubsection{\ce{HgCl2} $-$ Quecksilber(II)-Chlorid}
Dieser Stoff kann sowohl (auch trotz geringer Staubbildung) über die Atemwege, als auch über die Haut aufgenommen werden, wodurch sich Vergiftungssymptome äußern und erhöhte Quecksilber-Werte im Urin festzustellen sind.\\

\subsubsection{\ce{NH4VO3} $-$ Ammoniumvanadat}
Ammoniumvanadat ist bei Verschlucken hoch giftig und schädigt auch die Augen, das Einatmen des Stoffes ist aufgrund von potentiell schwerer Schädigung nach wiederholter oder längerer Exposition zu vermeiden.\\

\subsubsection{\ce{H2S} $-$ Schwefelwasserstoff}
Schwefelwasserstoff ist durch seinen besonders penetranten Geruch nach verfaulten Eiern einfach zu erkennen und wahrzunehmen.\\
Die Gefahr beim Einatmen von Schwefelwasserstoff ist das ersticken von innen, anders als bei der Blausäure (1.2.1), wird nicht die Zellatmung gehemmt, sondern gleich die Bindung von Sauerstoff an den Hämoglobin-Eisen-Komplex.\\
Somit kann von einer $"$Erstickung von innen$"$ gesprochen werden. Akute Gefahr besteht beim Einatmen einer Menge von Schwefelwasserstoff, bis der zuvor beschriebene Geruch nicht mehr wahrnehmbar ist.\\
Spätestens, sobald der Geruch nicht mehr wahrnehmbar ist, sollte der Betroffene so schnell wie möglich an die frische Luft. Ärztlicher Rat bzw. die Giftnotrufzentrale sollte auch zunächst zur Hilfe gezogen werden, wenn eine akute Lebensgefahr nicht auszuschließen ist.\\

\subsubsection{\ce{SO2} $-$ Schwefeldioxid}
Schwefeldioxid reagiert mit Wasser stark sauer und ist somit beim Einatmen sehr giftig. Zudem sind sowohl reduzierende als auch oxidierende Eigenschaften zu erkennen.\\

\subsubsection{\ce{Cl2} $-$ Chlorgas}
Das leicht gelbliche Chlorgas wurde unter anderem im 1. Weltkrieg sehr effektiv eingesetzt um Soldaten schnell außer Gefecht zu setzen, hierbei wirkt das hoch reaktive Halogengas über die Atemwege stark reizend und sollte umgehend behandelt werden.\\

\subsubsection{\ce{Br2} $-$ Bromgas}
Bromgas hat wie Chlorgas (1.2.10) eine stark reizende Wirkung auf die Schleimhäute über die Atemwege, ist jedoch etwas rötlich-brauner und kann Erbgutverändert wirken, die Fruchtbarkeit schädigen (vor allem bei Frauen permanent), so wie das Kind im Mutterleib schädigen.\\
Beim experimentieren mit Bromgas sei daher große Vorsicht ratsam und längere Exposition zu vermeiden.\\

\subsubsection{\ce{H2} $-$ Wasserstoffgas}
Wasserstoffgas ist ein hoch entzündliches, flüchtiges, geruchsloses und unsichtbares Gas, welches im richtigen Verhältnis zu Sauerstoff explosivartig reagieren kann, bei Exposition gegenüber einer Zündquelle.\\

\subsubsection{\ce{O2} $-$ Sauerstoff}
Sauerstoff kann hoch konzentriert starke Brände und Explosionen herbeiführen und stellt somit eine unsichtbare und geruchslose Gefahr da.\\

\subsubsection{\ce{AsH3} $-$ Arsenwasserstoff}
Dieses farblose, unangenehm nach Knoblauch riechende Gas reagiert in Mischung mit Sauerstoff explosiv, eine Freisetzung in die Atmosphäre ist zu vermeiden.\\

\subsubsection{\ce(C2H5OH) $-$ Methanol}
Methanol gilt als Kontaktgift und ist unter anderm durch seine Entstehung beim Schwerzbrennen und den somit verbundenen Erblindungs- und Todesfällen verbunden.\\
Als Kontaktgift kann dieses auch durch die Haut aufgenommen werden und seine giftige Wirkung freisetzen.

%%%%%%%%%%%%%%%%%%%%%%%%%%%%%%%%%%%%%%%%%%%%%%
\newpage
%%%%%%%%%%%%%%%%%%%%%%%%%%%%%%%%%%%%%%%%%%%%%%

\subsection{Umgang mit Säuren und Laugen}


\subsubsection{Säuren}
Bei Säuren ist erstmal ein grundlegender Spruch essentiell (auch wenn dieser nicht immer zutrifft).\\
\begin{center}
    \textbf{\textit{"Erst das Wasser, dann die Säure, sonst geschieht das Ungeheure"}}
\end{center}
Dies ist auf die Hitzeentwicklung bei der Dissoziation von starken Säuren in wässriger Lösung zurückzuführen, bei welcher Temperaturen erreicht werden können, um das Wasser zum sieden zu bringen.\\
Die Gefahr hierbei ist, dass das siedende Wasser in den gasförmigen Aggregatzustand übergeht und beim Entweichen durch die Lösung Säure mit aus dem Behälter reißt.\\
Somit sollte Säure im Wasser gelöst werden und nicht Wasser in der Säure.\\
Ist jedoch der Kontakt mit der Säure nicht zu verhindern, so gilt Tabelle 1.\\\\
Tab.1 Verhalten bei Kontakt mit Säure der aufgelisteten Objekte oder Körperteile.
\begin{center}
    \begin{tabular}{c | p{8cm}}
        Objekt/Körperstelle & Verhaltensregel\\
        \hline
        Arbeitsbereich & Da andere Experimentierende durch benetzung der Arbeitsplatte durch Säuren in gefahr gebracht werden, ist diese umgehend zu reinigen.\\
        \hline
        Haut & Die benetzte Stelle sollte umgehend mit genügend Seife und Wasser gründlich gewaschen werden. Ist dies aufgrund der Körperstelle nciht möglich, ist eine Notdusche zu verwenden.\\
        \hline
        Augen & Es sollte sofort eine Augendusche aufgesucht werden und die Augen gründlich ausgewaschen werden. Bei Verlust der Sehqualität sollte ein Arzt aufgesucht werden.\\
        \hline
        Kleidung & Die benetzte Kleidung sollte umgehen entfernt und gewaschen werden (den Laborkittel nicht ablegen).
    \end{tabular}
\end{center}

\subsubsection{Laugen}
Prizipiell gelten für Laugen dieselben Regeln, wie für die Säuren, somit ist für die Gefahren (1.4.1), so wie für das Verhalten bei Kontakt Tabelle 1 zu lesen.\\
Wichtig ist, dass bei Kontakt mit Augen sehr schnell eine irreperable Schädigung der Augen auftritt, weshalb eine unverzügliche Augendusche durchzuführen ist.\\ 

%%%%%%%%%%%%%%%%%%%%%%%%%%%%%%%%%%%%%%%%%%%%%%
\newpage
%%%%%%%%%%%%%%%%%%%%%%%%%%%%%%%%%%%%%%%%%%%%%%

\section{Quantitative Analysen}
Die Aufgabe bei den quantitativen Analysen besteht darin, die Stoffmenge $n$ einer Lösung experimentell zu bestimmen.\\
Hierzu sind essentielle Grundkenntnisse über den Zusammenhang von Masse $m$, molarer Masse $M$, Dichte $\rho$, Konzentration $c$, Volumen $V$ und der Stoffmenge $n$ vorrausgesetzt.\\
Die genannten Parameter sind hier nur Beispiele und stellen keine Vollständige Liste dar.\\



\subsection{Ausgabe von Analysen}
Für die quantitativen Analysen werden bis 8:30 Uhr saubere und trockene 100 ml Maßkolben vor der Analysenausgabe samt einem Zettel mit Namen des Laboranten, dem Datum der Abgabe, dem Probennamen \textcolor{gray}{("Quant 4 - \ce{X+}")} und dem Namen des Assistenten (um den Hals des Maßkolben) und dem Zettel des Assistenten in das Blech an die Richtige Position gestellt.\\
Die Lösung kann dann pünktlich zu Laborbeginn \textcolor{gray}{(Sofern alle Richtlinien beachtet wurden)} vor der Analysenausgabe abgeholt und in das Labor gebracht werden. Ein sicherer Transport ist zu gewährleuisten\\


\subsection{Preparation der Analyse}
Zur Preparation muss lediglich der Maßkolben bis zur Markierung auf 100 ml  mit demineralisiertem Wasser aufgefüllt werden.


\subsection{Proben}


\subsubsection{Quantität 1 $-$ Gravimetrische Analyse}
Bei diesem Versuch soll durch eine Gravimetrie von einem der folgenden Ionen die Stoffmenge dieser bestimmt werden.\\
\begin{itemize}
    \item \ce{Al^{3+}}
    \item \ce{Bi^{3+}}
    \item \ce{Ni^{2+}}
    \item \ce{Ca^{2+}}
    \item \ce{Pb^{2+}}
    \item \ce{SO4^{3+}}
    \item \ce{Cl-}
\end{itemize}

\subsubsection{Quantität 4 $-$ Komplexometrische Doppelbestimmung mit EDTA}
Durch den Chelatkomplexbildner \ce{EDTA} \textcolor{gray}{(\underline{E}thylen\underline{D}iammin\underline{T}etra\underline{E}ssigsäure)} soll in der Probe eine Mischung aus zwei Ionen maßanalytisch bestimmt werden.
\begin{itemize}
    \item \ce{Ca^{2+}}/\ce{Mg^{2+}}
    \item \ce{Fe^{3+}}/\ce{Al^{3+}}
    \item \ce{Zn^{2+}}/\ce{Mg^{2+}}
    \item \ce{Bi^{3+}}/\ce{Pb^{2+}}
\end{itemize}

\subsubsection{Quantität 5 $-$ Potentiometrisch verfolgte Säure-Base-Titration}
Während einer einfachen Säure-Base-Titration soll die Titrationskurve mit einem $p$H-Hand-Messgerät aufgenommen werden, wodurch eine der folgenden Säuren oder Basen nachgewiesen wird.
\begin{itemize}
    \item Salzsäure \ce{HCl}
    \item Schwefelsäure \ce{H2SO4}
    \item Essigsäure \ce{CH3COOH}
    \item Oxalsäure \ce{C2H2O4}
    \item Malonsäure \ce{C3H4O4}
    \item Phosphorsäure \ce{H3PO4}
    \item Kaliumhydrogenphtalat \ce{C8H5KO4}
    \item Soda \ce{Na2CO3}
    \item Borax \ce{Na2B4O7}
\end{itemize}


\subsection{Aufgaben}


\subsubsection{Quantität 1 $-$ Gravimetrie}

Bestimmung von \ce{Ca^{2+}} als Caliumoxalat Monohydrat \ce{CaC2O4 $\cdot$ H2O}\\\\
Benötigt werden:
\begin{itemize}
    \item 0.5 M und 0.01 M \ce{(NH4)2C2O3}-Lösung
    \item Salzsäure \ce{HCl}
    \item Methylrot(-Natriumsalz)
    \item \ce{H2O} chloridfrei
    \item 1 M \ce{NH3}-Lösung
    \item Titrierfritte (hier Tiegel mit Fritte)
    \item Herdplatte mit Magnetrührer
    \item stark verdünnte \ce{H2SO4}-Lösung\\\textcolor{gray}{1ml \ce{H2SO4} auf 100 ml demineralisiertes Wasser}
    \item Ethanol
\end{itemize}

Ein Aliquot von 25 ml wird aus der Analyselösung genommen, in einem Becherglas mit einigen Tropfen \ce{HCl} angesäuert und auf 100ml aufgefüllt.\\
Dies wird nun auf der Heizplatte 10 Minuten zum sieden gebracht und daraufhin mit 15 ml der 0.5 M Ammoniumoxalat\textcolor{gray}{\ce{(NH4)2C2O4}}-Lösung versetzt.\\
Nach dem hinzugeben von einigen Tropfen Methylrot als Indikator, wird dies nurn auf 70$^\circ$ gehalten, währen tropfenweise die 1 M \ce{NH3}-Lösung hinzugetropft wird, bis die Farbe von rot hin zu gelb übergeschlagen ist.\\
Das Becherglas wird nun von der Heizplatte genommen und mit der \ce{(NH4)C2O3}-Lösung so lange versetzt bis sich kein Feststoff mehr bei der Zugabe bildet.\\
Sobald das \ce{CaC2O4} vollends ausgefallen ist, wird das Gemisch durch das Stehenlassen auf Raumtemperatur gebracht.\\
Nun sollte ein klarer Feststoff auf dem Boden des Becherglases zu erkennen sein.\\
Nachdem der Tiegel abgewogen und sein Gewicht notiert wurde, wird der Feststoff durch diesen von der Flüssigkeit getrennt, viermal mit stark verdünnter \ce{H2SO4}-Lösung und zweimal mit Ethanol gewaschen.\\
Der Feststoff wird bei 130$^\circ$ im Trockenschrank getrocknet und dann gewogen, wobei die Differenz zwischen Tiegelgewicht ohne und mit Feststoff die Masse des Feststoffes ergibt.\\
Da ein Aliquot nur $\frac{1}{4}$ der Stoffmenge $n$ beinhaltet, muss der berechnete Wert (hier ist zu beachten, dass die molare Masse $M$ von \ce{CaC2O4}$\cdot$\ce{H2O} zu nehmen ist) mit dem Faktor 4 multipliziert werden.

\subsubsection{Quantität 5 $-$ Potentiometrisch verfolgte Säure-Base-Titration}

Ein 25 ml Aliquot wird in ein 250ml Becherglas samt Magnetfisch gegeben, auf 100 ml verdünnt und auf den Magnetrührer gestellt.\\
Ein Hand-$p$H-Meter kann im Assistentenzimmer geholt werden, vor dem Gebrauch wird die Elektrode mit demineralisiertem Wasser abgespült.\\
Die Titration wird noch zweimal wiederholt.\\

%%%%%%%%%%%%%%%%%%%%%%%%%%%%%%%%%%%%%%%%%%%%%%

\end{document}