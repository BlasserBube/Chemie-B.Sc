\documentclass[a4paper, fleqn]{article}
\usepackage{chemfig}
\usepackage{mhchem}
\usepackage{xcolor}
\usepackage{amsmath}
\usepackage{tikz}
\usepackage{chemfig}
\usepackage{amssymb}
\usepackage{hyperref}
\usepackage{makecell}
\usepackage[
  left=1cm,
  right=1cm,
  top=2cm,
  bottom=2cm,
]{geometry}
\begin{document}
\tableofcontents
\newpage
\section{Nebengruppenmetalle}
\subsection{Die siebte Gruppe}
\subsubsection{Vorkommen}
Mangan: Oxide: \ce{MnO2, Mn2O3, Mn3O4}\\
Technetium: vom Kernbrennstab\\
Rhenium: vergesellschaftet mit \ce{MoS2}
\subsubsection{Herstellung}
Mangan: Aluminothermisch aus \ce{Mn3O4}\\
Technetium: \ce{^{89}Mo + ^1_0 n -> ^{99}Mo - $\beta$^- -> ^{99*}Tc - $\gamma$ -> ^{99}Tc}\\
Rhenium: Nebenprodukt in den Röstgasen der Molybdänherstellung aus \ce{MoS2} $\rightarrow$ \ce{Re2O7} $\rightarrow$ Reduktion mit \ce{H2}

\subsubsection{Verbindungen}
Halogene:\\
Mn - Halogenide nur in den niedrigen Oxidationsstufen des Mangans.\\
Tc- und Re- Halogenide auch für höhere Oxidationsstufen (\ce{TcF6} oder \ce{TcCl4}); bei Re: Clusterbildung\\\\
Sauerstoffverbindung:\\
\ce{Mn^{2+}}: \ce{Mn(OH)2 - H2O -> MnO}\\
\ce{Mn^{3+}}: \ce{Mn2O3}\\
\ce{Mn^{4+}}: \ce{MnO2} oder \ce{MnO(OH)2}\\
\ce{Mn^{5+}}: \ce{MnO4^{3-}} nur im stark alkalischen, hellblau\\
\ce{Mn^{6+}}: \ce{MnO4^{2-}} lakalisch, dunkelgrün\\
\ce{Mn^{7+}}: \ce{MnO4^-} violett\\
\vspace*{2cm} \ce{MnO4^- + H^+ -> HMnO4}\\
\vspace*{3cm} \ce{2HMnO4 - H2O -> Mn2O3}\\\\
Technetium und Rhenium:\\
hohe Ox-Stufen \ce{Tc2O7, Re2O7,ReO3}\\

\subsubsection{Technische Verwendung}
Léclanché-Element $\rightarrow$ \textbf{siehe Folie}\\
Zinkbecher: \ce{Zn+ 4NH4^+ -> [Zn(NH3)4]^{2+} + 2e^- + 4H^+}\\
oder \ce{Zn+ 4OH^- -> [Zn(OH)4]^{2-} + 2e^- + 4H^+}\\
Braunsteinpulver: \ce{MnO2 + H2O + e^- -> MnO(OH) + OH^-}\\
Ergibt ca. $1.5\,\mathrm{V}$

\subsection{Die achte, neunte und zehnte Gruppe, Eisen Cobalt und Nickel}
\subsubsection{Vorkommen}
Eisen: \ce{Fe2O3} (Hämatit); \ce{FeO(OH)} (Goethit); \ce{Fe3O4} (Magnetit); \ce{FeS2} (Pyrit/ Markasit)\\
Cobalt und Nickel: \ce{CoAsS, CoAs3, NiAs, (Ni/Fe)9S8, NiS}\\
\ce{NiAs} ist hexagonal Analog zur \ce{NaCl}-Struktur\\

\subsubsection{Herstellung}
Hochofenprozess von Eisen und Stahl von \ce{Fe2O3} zu Roheisen $\rightarrow$ \textbf{siehe Folie}\\
Roheisen enthält bis zu 4 \% C\\
Aufarbeiten mit "Schrott" $\rightarrow$ Rost \ce{Fe2O3}\\
Eisen veredler mit \ce{Cr, Mo, V, ...}\\
Cobalt/Nickel: Rösten\\
Reinigung von Nickel $\rightarrow$ Mond-Verfahren\\
\ce{Ni + 4CO ->[{353.15 K}] [Ni(CO)4] ->[{433.15 K}] Ni + 4CO}

\subsubsection{Verbindungen}
Halogenide:
\begin{itemize}
    \item Eisen: für \ce{Fe^{2+}} und \ce{Fe^{3+}} gibt es alle Halogenide.
    \item Cobalt: für \ce{Co^{2+}} alle Halogenide bekannt\\für \ce{Co^{3+}} nur das Fluorid bekannt.
    \item Nickel: für \ce{Ni^{2+}} alle Halogenide bekannt.
\end{itemize}
Oxide:
\begin{itemize}
    \item Eisen: \ce{Fe2O3} (Hämatit); \ce{FeO_$1-x$}; \ce{Fe3O4} (Magnetit)
    \item Cobalt: \ce{CoO}; \ce{Co3O4} ; \ce{Co2O3} (Alle Schwarz wegen Metal-to-Metal-Charge-Transfer)
    \item Nickel: \ce{NiO}; \ce{Ni2O3}/\ce{Ni3O4} (beide nicht rein erhältlisch); \ce{NiO2$\cdot x$H2O}
\end{itemize}
Komplexchemie:\\
Eisen:\\
\begin{center}
    \begin{tabular}{c c c}
        \hline
        \ce{Fe^{2+}} ($d^6$) & vs & \ce{Fe^{3+}} ($d^5$)\\
        \hline
        alle Orbitale einfach besetzt ein Orbital zweifach (ls) & & alle Orbitale einfach besetzt\\
        Aqua-Komplexe: leicht grün & & gelb (sollte eigentlich farblos sein)aber \ce{[Fe(H2O)5OH]} Kationensäure\\
    \end{tabular}
    Zusammen in einer Verbindung: Berliner/Turnbulls/Preußisch Blau
\end{center}
Maximal 4 \ce{SCN^{-}} Liganden um ein \ce{Fe^{3+}}\\
\ce{Fe^{3+}} ist mit \ce{F^{-}} maskierbar $\rightarrow$ \ce{[FeF6]^{3-}} stabil aufgrund hoher Bindungsenergie\\\\
Cobalt: \ce{Co^{2+}} ($d^7$) $\rightarrow$ rosa/rot alle Orbitale einfach besetzt 2 Oben 3 Unten, zwei Orbitale unten doppelt. blaue Komplexe gleich aber 2 Unten 3 Oben\\
\ce{Co^{2+}} lowspin $\rightarrow$ 2 Oben 3 unten, alle unteren Orbitae doppelt besetzt, der obere einfach.\\
Es entsteht hierbei ein Radikal, das durch Dimerisierung zu einer Bildung zweier Komplexe führt, welche um 45 Grad zueinander verschoben sind.\\
Mit dem Zusatz eines Oxidationsmittel: \ce{[Co(Cn)6]^{3-}}\\\\
Nickel: \ce{Ni^{2+}} ($d^8$) 2 Orbitale oben, 3 unten, alle unteren doppelt befüllt, obere einfach.\\
Mit sehr starken Liganden kommt es zu einem quadratisch-planaren Feld.
\subsubsection{Die 8.,9. und 10. Gruppe}
\ce{Ru, Rh, Pd, Os, Ir, Pt}\\
\subsubsection{Vorkommen}
"Platinmetalle"\\
\begin{itemize}
    \item gediegene Elemente\\ $\hookrightarrow$ Überführung in Oxide und Destillation.\\$\hookrightarrow$ Überführung in Hexachloridometallat\\$\hookrightarrow$ Ionenaustausch/Komplextitration\\$\hookrightarrow$ "Urban Mining"
\end{itemize}
\subsubsection{Verbindungen}
Oxide: Oxidationsstufen der Metalle von +4 und höher:\\
\ce{RuO2}; \ce{RhO2}\\
Maixmal:\\
\ce{RuO4} / \ce{OsO4}\\
Auch ternäre (dreikomponentige) Oxide\\
\ce{BaRuO3}; \ce{Na3RuO4}\\\\
Komplexchemie:\\
alles lowspin\\
bei $d^8$ (\ce{Pd^{2+}}; \ce{Pt^{2+}}) $\rightarrow$ quadratisch-planaren

\section{Die Seltenerdelemente, Lanthanoide \& Actinoide}
\subsection{Eigenschaften}
\begin{itemize}
    \item gute Reduktionsmittel ($E^0 = 2,3-2,5\,\mathrm{V}$)
    \item Lanthanoide $\rightarrow$ alle Oxidationsstufe + 3\\ \ce{Ce, Tb, (Pr)} +4; \ce{Eu,Yb,Sm,Tm} +2
    \item Elektronenkonfiguration (\textbf{siehe Folie})
    \item Lanthanoidenkontraktion\\ $f$-Orbitale sind kernnah und bieten damit schlechte Abschirmung der Kernladung.\\ $\hookrightarrow$ die Atome "schrumpfen" kontinuierlich
\end{itemize}
Konsequenzen:
\begin{itemize}
    \item Koordinationszahl um die \ce{Lanthanoide^{3+}}-Kationen sind von 9-10 um die vorderen zu 6-7 um die hinteren Lanthanoiden
    \item Die Härte der \ce{Lanthandoide^{3+}}-Kationen nimmt von \ce{La} bis \ce{Lu} zu
    \item Die Hydratationsenthalpie bimmt von \ce{La} bis \ce{Lu} zu
    \item Die Basizität der Oxide sinkt von \ce{La2O3} bis \ce{Lu2O3}
\end{itemize}

\subsection{Vorkommen}
\subsubsection{Die Seltenerdelemente}
\begin{itemize}
    \item Phosphate: \ce{Lanthanoide^{3+}PO4} - Monazit für große \ce{Lanthanoide^{3+}}\\ - Xenotin für kleine \ce{Lanthanoide^{3+}}
    \item Bastnäsit: \ce{Lanthanoid^{3+}F(CO3)} - Fluorid-Carbonat - für große \ce{Lanthanoide^{3+}}
\end{itemize}

\subsubsection{Die Actinoide}
\ce{U} als \ce{UO2} und \ce{Th} als \ce{ThO2}

\subsection{Herstellung}
\subsubsection{Die Seltenerdelemente}
\begin{itemize}
    \item Erze + \ce{H2SO4} + Druck \ce{-> Lanthanoide2(SO4)3}$\cdot x$\ce{H2O}
\end{itemize}
Trennung:\\
Früher: Fraktionierte Kristallisation/Fällung/Zersetzung oder Fraktionierte Lösungsextraktion\\
Das bis zu 10,000 mal hintereinander\\\\
Heute: Ionentausch und Komplexbildung.\\
Anlagerung an Ionentauscherharz geht nacheinanger, abhängig von \ce{Lanthanoid^{3+}}-Größe.\\
1. Anlagerung \ce{La^{3+}} deutlich besser als \ce{Lu^{3+}}\\
2. Komplexbildner \ce{Lu^{3+}} deutlich besser als \ce{La^{3+}}\\
$\rightarrow$ Überführung in Chloride/Fluoride\\
$\hookrightarrow$ Schmelzflusselektrolyse/Metallothermie mit \ce{Ca}

\subsubsection{Die Actinoide}
1.Rösten der Uranerze zu \ce{UO3}
\begin{equation*}
    \ce{UO3 + 3 H2SO4 + H2O -> 4 H3O^{+} + [UO2(SO4)3]^{4-}}
\end{equation*}
2. Zugabe von \ce{NaOH} oder \ce{NH3}
\begin{equation*}
    \rightarrow \ce{UO3 \cdot 2 H2O}; \ce{(NH4)2U4O13 \cdot 6 H2O} \dots
\end{equation*}
3. Lösen mit \ce{HNO3}; Extraktion mit Tributylphosphat
\begin{equation*}
    \rightarrow \ce{[UO2(NO3)2TBP2]}
\end{equation*}
Zersetzung zu \ce{UO3}
4. Reduktion von \ce{UO3} mit \ce{H2} zu \ce{UO2}\\
\begin{equation*}
    \ce{+ HF -> UF4}
\end{equation*}
"grünes Salz"\\
5. \ce{UF4 + F2 -> UF6}\\
Zentrifuge (Abtrennung von \ce{^{235}UF6}) $\rightarrow$ \ce{UO3} $\rightarrow$ \ce{UO2} (Brennelement)\\
Aber: \ce{UFe + 2 H2O -> UO2F2 + 4 HF}

\subsection{Verbindungen}
\subsubsection{Seltenerdelemente}
Halogenide:\\
\ce{LanthanoideCl3}\\
$\rightarrow$ bei kleinen \ce{Lanthanoide^{3+}} $\rightarrow$ \ce{AlCl3}-Typ\\
$\rightarrow$ bei großem \ce{Lanthanoide^{3+}} $\rightarrow$ \ce{UCl3}-Typ\\\\
Oxide und Oxidoverbindungen\\
\ce{Lantahnoid2O3}\\
$\rightarrow$ bei kleinen \ce{Lanthanoide^{3+}} $\rightarrow$ Koordinationszahl von 6 analog zu \ce{Al2O3}\\
$\rightarrow$ bei großem \ce{Lanthanoide^{3+}} $\rightarrow$ Koordinationszahl von 7\\\\
\ce{Lanthanoid(OH)3} $\rightarrow$ besische Hydroxide , lösen sich in Säuren, von Lantah zu Lutetium immer amphoterer\\
Komplexe Anionen $\rightarrow$\\
Andere Oxidationsstufen\\
\ce{Ce^[4+], CeO2}, \ce{Pr^{4+}, Pr6O11}, \ce{Tb^{4+}, Tb4O7}\\
\ce{Eu^{2+}}, \ce{Yb^{2+}}, \ce{Sm^{2+}}, \ce{Tm^{2+}}\\
\ce{2 Ce^{4+} + 2 I^{-} -> I2 + 2Ce^{3+}}
Uranverbindungen:\\
\ce{U^{6+} + O^{2-} -> UO2^{2+}}\\
2 $\sigma$-Bindungen\\
4 $\pi$-Bindungen

\subsection{Technische Verwendung}
$\ce{YBa2Cu3O_{7-x}} \rightarrow$ Supraleiter\\
\ce{Eu^{2+}/Eu^{3+}} $\rightarrow$ Leuchtstoffe
\end{document}