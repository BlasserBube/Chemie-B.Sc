\documentclass{article}
\usepackage{xcolor}
\usepackage{amsmath}
\usepackage{mhchem}
\usepackage{chemfig}
\usepackage{tabularx}
\usepackage{makecell}
\usepackage{hyperref}
\usepackage[
  left=1cm,
  right=1cm,
  top=2cm,
  bottom=2cm,
]{geometry}

\begin{document}
\tableofcontents
\newpage
\section{Metalle mit Ingo}
\subsection{Eigenschaften metallischer Elemente}
Physikalische Eigenschaften
\begin{itemize}
    \item Leitfähigkeit
    \begin{itemize}
        \item elektrischen
        \item thermische
    \end{itemize}
    \item Metallischer Glanz
    \item Duktilität (Formbarkeit)
    \item Nicht Lichtdurchlässig
\end{itemize}
Chemische Eigenschaften
\begin{itemize}
    \item niedrige Elektronegativität
    \item bildet bevorzugt Kationen
    \item Meist basische Hydroxide!?
    \begin{itemize}
        \item niedrige Oxidationsstufe: JA \\Beispiel: \ce{Cr(OH)2 + H2O -> Cr^{2+} + 2OH- + H2O}
        \item hohe Oxidationsstufe: NEIN \\Beispiel: \ce{Cr(OH)6} (gibt's nicht) wird zu \ce{CrO2(OH)2} $\rightarrow$\ce{H2CrO4}\\ \ce{H2CrO4 + 2H2O -> CrO4^{2-} + 2H3O+}
    \end{itemize}
\end{itemize}
\subsection{Elektrisches Verhalten}
\subsubsection{Betrachtung des spezifischen Widerstands}
\begin{itemize}
    \item Metalle: $10^{-4}$ bis $10^{-6} \Omega\cdot$ cm$^{-1}$
    \item Halbleiter: $10^{1}$ bis $10^{4} \Omega\cdot$ cm$^{-1}$
    \item Isolator: $> 10^{10} \Omega\cdot$ cm$^{-1}$
\end{itemize}
\subsubsection{Betrachtung der thermischen Verhaltens der Leitfähigkeit}
Siehe Folie
\subsection{Definition des metallischen Zustands}
Phänomenologisch: schwierig, da makroskopische Eigenschaften wie Glanz, Duktilität verändert werden können.\\
Temperaturabhängigkeit der elektrischen Leitfähigkeit: schwierig, da andere Stoffklassen ähnliche Eigenschaften aufweisen.
\subsection{Die chemische Bindung in Metallen}
\subsubsection{Ketelaar-Diagramm}
Man stelle sich ein Dreieck vor mit den Eckenbeschriftungen ionische Bindung \ce{NaCl}, kovalente Bindung \ce{Cl2} und metallisch \ce{Na}
\subsubsection{Das Elektronengasmodell}
\begin{itemize}
    \item Die Metallatome geben eine gewisse Zahl an Valenzelektronen ab, es verbleiben positiv geladene Atomrümpfe
    \item Die Elektronen sind zwischen den Atomrümpfen frei beweglich, ähnlich eines Gases $\rightarrow$ Elektronengas \textcolor{red}{(versagt bei der Beschreibung der Wärmekapazität von Metallen)}
\end{itemize}
\subsubsection{Das Bändermodell}
\begin{itemize}
    \item Elektronen können nur bestimmte Energien aufweisen\\$\rightarrow$ Orbitale (hier Atomorbitale)
    \item Beim Übergang von Ein- zu Mehratomsystemen\\$\rightarrow$ Übergang von Atom- zu Molekülorbitalen
\end{itemize}
\ce{Li3}: $+\hspace*{0.25cm}+\hspace*{0.25cm}+\hspace*{0.25cm} = \sigma_b$\\
\hspace*{0.85cm}$+\hspace*{0.25cm}-\hspace*{0.25cm}+\hspace*{0.25cm} = \sigma_{ab}$\\
\hspace*{0.85cm}$+\hspace*{0.45cm}|\hspace*{0.45cm}+\hspace*{0.25cm} = \sigma_{nb}$
\begin{itemize}
    \item Beim Übergang von Mehr- zu Vielatomsystemen\\$\rightarrow$ Übergang von Molekülorbital zu (Orbital-) Bändern\\$\rightarrow$ Valenzband: mit Valenzelektronen besetzt, höchster besetzte Zustand: HOMO\\$\rightarrow$ Leitungsband: frei, niedrigste unbesetzte Zustand: LUMO
\end{itemize}
Fermikante = Ort zwischen Besetzt und Unbesetzt\\
\subsection{Strukturen der Metalle}
Übersicht:
\begin{itemize}
    \item kubisch-innenzentriert
    \item hexagonal dichteste Packung
    \item kubisch dichteste Packung
    \item eigener Strukturtyp
    \item unbekannt
\end{itemize}
\subsubsection{Die kubisch-innenzentrierte Kugelpackung}
\textbf{(bcc = body-centered cubis), W(olfram)-Typ}
\underline{C}oordination\underline{N}umber = 8 + 6\\
Koordinationspolyeder = Rhombododecaeder\\
Raumerfüllung = 68\%\\
Siehe Folie für näheres.
\newpage
\subsubsection{Die dichtesten Packungen}
\textbf{Hexagonal-dichteste Kugelpackung\\(hcp = hexagonal close packed), M(a)g(nesium)-Typ}
CN=12\\
Koordinationspolyeder = Antikuboktaeder\\
Raumerfüllung = 74\%\\\\
\textbf{Kubisch-dichteste Kugelpackung\\(ccp=cubic close packed), Cu(pfer)-Typ}
CN = 12\\
Koorinationspolyeder = Kuboktaeder\\\\
\textbf{Varianten der dichtesten Kugelpackungen}\\
hc-Typ\\
hhc-Typ\\
Kommen vor und nach einer Schicht dieselbe Schicht, so ist diese hexagonal umgeben. (Kurz: h)\\
Sind die Schichten vor und nach der betrachteten Schicht nicht gleich, so ist die betrachtete Schicht kubisch umgeben. (Kurz: c)\\\\
Siehe Folie.\\\\
\textbf{Variation der Kristallstruktur der Metalle.\\(Abhängig von Druck und Temperatur)}\\\\
\ce{Fe}: $\alpha \left(\mathrm{bcc}\right) \rightarrow \gamma \left(\mathrm{ccp}\right) \rightarrow \delta \left(\mathrm{bcc}\right)$\\
Erster Schritt bei ca. 900$^\circ$, zweiter schritt bei ca. 1400$^\circ$\\\\
\ce{Na}: bcc \ce{->} ccp \ce{-> -> ->} transparente Modifikation, kein Metall mehr\\
Dabei läuft der erste Schritt bei 656 Pa ab und der letzte bei $>$ 100 GPa
\subsubsection{Aufgefüllte dichteste Packungen}
\begin{itemize}
    \item Oktaederlücken\\hcp-Abfolge: A c B (A,B = Schichten, c = Lücken)\\$N(\mathrm{Oktaederl\text{ü}cken})=N(\mathrm{Packungsteilchen})$\\
            ccp Abfolge: A c B a C b A (A,B,C = Schichten, a,b,c = Lücken)
    \item Tetraederlücken\\hcp:Abfolge: A $\beta$ $\alpha$ B $\alpha$ $\beta$ A $\beta$ (A,B = Schichten, $\alpha, \beta$ = Lücken)\\$N(\mathrm{Tetraederl\text{ü}cken})=2N(\mathrm{Packungsteilchen})$\\Tetraederlücken\\
            ccp:Abfolge: A $\beta$ c $\alpha$ B $\gamma$ a $\beta$ C $\alpha$ b $\gamma$ A (A,B,C = Schichten, $\alpha, \beta, \gamma$ = Tetraederlücken, a, b, c = Oktaederlücken)
\end{itemize}

\section{Die Elemente der ersten und elften Periode (-\ce{H}\&\ce{Rg})}
\begin{itemize}
    \item[1. Gruppe] Alkalimetalle
    \item[11. Gruppe] Münzmetalle 
\end{itemize}
\subsubsection{Vorkommen}
\begin{itemize}
    \item[Alkalimetalle]:
    \begin{itemize}
        \item kationisch in salzartigen Verbindungen \ce{NaCl} - Halit, \ce{KCl} -Sylvin
        \item kationisch eingelagert in Alumosilicaten (\ce{LiAlSi2O6})
    \end{itemize}
    \item[Münzmetalle]:
    \begin{itemize}
        \item[Kupfer:] hauptsächlich sulfidisch: \ce{Cu2S}, \ce{CuFeS2}, $\dots$\\auch: gediegen (elementar)
        \item[Silber:] hauptsächlich gediegen\\auch: sulfidisch
        \item[Gold:] hauptsächlich gediegen\\selten: Goldtelluride  
    \end{itemize}
\end{itemize}
\subsubsection{Herstellung}
\begin{itemize}
    \item[Alkalimetalle]:
    \begin{itemize}
        \item[\ce{Li} und  \ce{Na}:] Schmelzflusselektrolyse aus Salz(-mischungen)
        \item[\ce{K}:] Reduktion mit metallischem \ce{Na}
        \item[\ce{Rb} und \ce{Cs}:] Reduktion mit metallischem \ce{Ca} und anschließender Destillation
    \end{itemize}
    \item[Münzmetalle]:
    \begin{itemize}
        \item[\ce{Cu}:] Rösten der sulfideischen Kupfererze\\Rösten: \ce{6CuFeS2 + 13O2 -> 3Cu2S + 2Fe3O4 + 9SO2}\\Schlacke: \ce{2Fe3O4 + 2CO + 3 SiO2 -> 3Fe2SiO4 + 2 CO2}\\$\rightarrow$(Abtrennug des Eisenanteils)\\\begin{center}\ce{2Cu2S + 3O2 -> 2 Cu2O + 2SO2}\end{center}\begin{tabular}{c | c}Röstreaktion & Röstreduktion\\\ce{2Cu2O + Cu2S -> 6Cu + SO2$\uparrow$}&\ce{Cu2O + CO -> 2Cu + CO2$\uparrow$}\end{tabular}\\Reinigung des Rohkupfers durch elektrolytische Kupferaffinition
        \item[\ce{Ag} und \ce{Au}:] Reinigung der gediegenen Metalle
        \begin{itemize}
            \item Recycling aus Anodenschlamm (Reinigung des Rohkupfers)
            \item Amalgamierung vom Gold, Goldwäsche
            \item Cyanidlaugerei\\\ce{Ag2S + 4NaCN -> 2Na[Ag(CN)2] + Na2S}\\\ce{2Ag + H2O + 1/2O2 + 4NaCN -> 2Na[Ag(CN)2] + 2 NaOH}\\\\\ce{Ag^+ + 2CN^- -> [Ag(CN)2]^-} $K_K \approx 10^{21}\mathrm{\frac{mol^2}{l^2}}$\\$K_K=\frac{\ce{[[Ag(CN)2]^-]}}{[\ce{Ag^+}]\cdot\ce{[CN^-]^2}} \rightarrow \ce{[Ag^+]} = \frac{\ce{[[Ag(CN)2]^-]}}{K_K\cdot\ce{[CN^-]^2}}$\\$E=E^o_{(\ce{Ag}/\ce{Ag^+})}+\frac{RT}{zF}\ln([\ce{Ag^+}])$\\\\Rückgewinnung des Silbers\\\ce{2Na[Ag(CN)2] + Zn -> 2Ag + Na2[Zn(CN)4]}
        \end{itemize}
    \end{itemize}
\end{itemize}
\subsubsection{Verbindungen}
\begin{itemize}
    \item[Halogenide]:
    \begin{itemize}
        \item Alkalimetallhalogenide: \ce{A} = \ce{Li} bis \ce{Cs} $\rightarrow$ \ce{AX} $\leftarrow$ \ce{X} = \ce{F} bis \ce{I}
        \item[] \ce{NaCl}-Struktur: ccp mit allen Oktaederlücken gefüllt
        \item[] \ce{CsCl}-Struktur: kubisch-primitiver Aufbau der Packungsteilchen, Lückensitzer im Zentrum des Würfels 
    \end{itemize}
    \item[Münzmetalle]:
    \begin{itemize}
        \item[] Cu(I)-Halogenide vom \chemfig{Cl-I}
        \item[] Cu(II)-Halogenide $\rightarrow$ schwache Oxidationsmittel
        \item[] \ce{CuCl2 + Cu -> CuCl} $\ce{->[\mathrm{mehr} \ce{Cl^-}] CuCl_{2/3/4}^{1/2/3 -}}$
        \item[] \ce{CuCl2 + Fe^{2+} -> CuCl + Fe^{3+} + Cl^-}
        \item[] \ce{CuI2 -> CuI + 1/2I2}
        \item[] \ce{Cu^{2+} + 2 CN^- -> CuCN + 1/2 (CN)2}
        \item[] Oxidation organischer Verbindungen $\rightarrow$ Fehling-Probe
        \item[] \ce{Ag^+} + Halogenide $\rightarrow$ \ce{AgF}, \ce{AgCl}, \ce{AgBr}, \ce{AgI} 
    \end{itemize}
\end{itemize}
\subsubsection{Sauerstoff-Verbindungen}
\begin{itemize}
    \item[] \ce{4Li + O2 -> 2Li2O}
    \item[] \ce{6Li + N2 -> 2Li3N}
    \item[] \ce{2Na + O2 -> 2NaO ->[{besser}] Na2O2} - natriumperoxid (\ce{O2^{-2}})
    \item[] \ce{A + O2 -> AO2} mit \ce{A} = \ce{K}, \ce{Rb}, \ce{Cs} \\ Der Name des \ce{AO2} lautet: "Alkalimetallsuperoxid" $\rightarrow$ \ce{O2^-}
    \item[] Umsetzung mit mehr \ce{O2}:
    \item[] \ce{A4O6} $\rightarrow$ 1$\times$ \ce{O2^{-2}} + 2 $\times$ \ce{O2^-}
    \item[] Umsetzung mit Metallüberschuss $\rightarrow$ Alkalimetallsuboxide
\end{itemize}
Münzmetalle
\begin{itemize}
    \item[] \ce{Cu2O} rot; \ce{CuO} schwarz
    \item[] \ce{Ag2O}, \ce{AgO} aber \ce{Ag^I Ag^{II} O2}
    \item[] \ce{AuO} -"- \ce{Au2O3}
\end{itemize}

\subsubsection{Hydroxide}
\begin{itemize}
    \item Alkalimetallhydroxide
    \begin{itemize}
        \item stark basisch
        \item ziehen \ce{CO2} aus der Luft
    \end{itemize}
    \item Herstellung durch Elektrolyse aus NaCl-Lösung
    \begin{itemize}
        \item Chloralkalielektrolyse
        \item[] \ce{2NaCl + 2H2O ->[{Strom}] 2Na^+ + 2OH^- + Cl2 + H2}
        \item[Probleme:] \ce{Cl2} disproportioniert in Lauge\\\ce{H2 + Cl2 ->} Chlorknallgas 
    \end{itemize}
\item Münzmetallhydroxide
    \begin{itemize}
        \item \ce{Cu(OH)2}
        \item \ce{Au(OH)3}
        \item[] \ce{2A + 2H2O -> A^+ + OH^- + H2}
    \end{itemize}
\end{itemize}

\subsubsection{Alkalimetall-Elektrode und Alkalide}
\begin{center}
    \ce{A -> A^+ + e^-}\\
    $\hookrightarrow$ \ce{+ 3-4 NH3 -> [e(NH3)_{3-4}]^-}
\end{center}
auch möglich:\\
\ce{A} + $\begin{array}{c}Kronenether\\ Cryptant\end{array}$ \ce{->} [A(Kronenether)]$^+$ + e$^-$ \ce{->[{+A}]} [A(Kronenether)]$^+$ + A$^-$
\subsubsection{Stickstoffverbindungen}
\begin{itemize}
    \item[$\rightarrow$] Nitride \ce{N^{3-}}
    \item[$\rightarrow$] Imide \ce{NH^{2-}} (vgl. \ce{O^{2-}})
    \item[$\rightarrow$] Amide \ce{NH2^-} (vgl. \ce{OH^- H^-})
    \item[$\rightarrow$] Ammoniak \ce{NH3} (vgl. \ce{H2O HF})
    \item[$\rightarrow$] Ammonium \ce{NH4^+} (vgl. \ce{H3O^+ H2F^+} \textcolor{gray}{$\rightarrow$ \ce{CH4}})
    \item[$\rightarrow$] Azide \ce{N3^-} (isoelektronisch zu \ce{N2O CO2 NO2^+})
\end{itemize}

\subsection{Oxidationsstufen der Münzmetalle}
\subsubsection{Allgemeines}
$\hookrightarrow$ Siehe Folie\\
Wiederholung der Kristallfeldtheorie\\
$\hookrightarrow$ Siehe Folie\\

\subsubsection{Verbindungen von \ce{Cu} und \ce{Ag} in hohen Oxidationsstufen}
\ce{CuF3, K3[CuF6], 4Ba2Cu3O_{7-x}} (Supraleiter)\\
\ce{K[AgF4], Cs2[AgF4]}

\subsection{Die Chemie der Golds}
\subsubsection{Relativistische Effekte}
Kontraktion von 6s und 6p; Expansion von 5d
\begin{itemize}
    \item $r$(Au) $\approx r$(Ag) $\rightarrow$ höhere dichte
    \item höhere Elektronenaffinität $\rightarrow$ \ce{Au^-} aber kein \ce{Ag^-}
    \item aurophile Wechselwirkungen $\rightarrow$ \chemfig{Au-Au}-Bindungen in der Gasphase
    \item Farbigkeit $\rightarrow$ elektronische Übergäng eim sichtbaren Bereich
\end{itemize}

\subsubsection{Goldverbindungen}
Oxidation von Gold durch Königswasser
\begin{itemize}
    \item[] \ce{HNO3 + 3HCl -> NO4 + 2H2O + 2Cl$\cdot$}\\\ce{Cl$\cdot$} ist das naszierende Chlor
    \item[] \ce{Au + 3Cl$\cdot$ + Cl^- -> [AuCl4]^-} \textcolor{gray}{(Tetrachloridoaurat)}
    \item \ce{Au^{2+}} $\mathrm{5d^9}$-System $\rightarrow$ \ce{Au2^{4+}}
\end{itemize}

\section{Elemente der 2. und 12. Gruppe}

\subsection{Vorkommen}
\subsubsection{Erdalkalimetalle}
\ce{Be}: in (Alumo-)Silicaten: z.B. \ce{Be3Al2[Si6O18]}\\
\ce{Mg + Ca}: \begin{itemize}
    \item Carbonate z.B. \ce{CaCO3}
    \item Sulfate
    \item Halogenide
\end{itemize}
\ce{Sr + Ba}: \begin{itemize}
    \item Carbonate
    \item Sulfate
\end{itemize}

\subsubsection{Elemente der Zink-Gruppe}
\ce{Zn+Cd}: \begin{itemize}
    \item Sulfide 
    \item Carbonate (untergeordnet)
\end{itemize}
\ce{Hg} \begin{itemize}
    \item Sulfide  (Farben durch ostwaldsche Stufenregel)
    \item Gediegen
\end{itemize}

\subsection{Herstellung}
\subsubsection{Erdalkalimetalle}
\ce{Be}: \ce{BeF2 + Mg -> Be + MgF2}\\
\ce{Mg}: Schmelzflusselektrode\\
\ce{Ca, Sr, Ba}: Aluminothermie: \ce{4MO + 2Al -> 3M + MAl2O4}
\subsubsection{Zinkgruppe}
\ce{M = Zn + Cd}: \ce{MS + O2 -> MO + SO2}\\
1. Röstreduktion: \ce{ZnO + CO -> Zn + CO2}\\
2. "Im Nassen": \ce{ZnO + H2SO4 -> Zn^{2+} + SO4^{2-} + H2O}\\\\
\ce{2HgS + O2 -> 2Hg + SO2}

\subsection{Verbindungen}
\subsubsection{Halogenide \ce{MX2}}
Metall in Tetraederlücken aus \ce{X}\\
z.B. \ce{BeCl2} oder \ce{ZnCl2} $\rightarrow$ siehe Folie\\
Metall in Oktaederlücke aus \ce{X}:\\
z.B. \ce{CaCl2, MgI2, CdCl2} $\rightarrow$ siehe Folie\\
Metall in kubischen Lücken aus \ce{X}\\
z.B. \ce{CaF2} $\rightarrow$ siehe Folie

\subsubsection{Chalkogenide}
\ce{ZnS} in Zinkblende und Wurzit-Typ $\rightarrow$ siehe Folie\\
Kalk: \ce{CaCO3} (Kalkstein)\\
\ce{CaCO3 ->} CaO + CO2\\
\ce{CaO} ist gebrannter Kalk\\
\ce{CaO + H2O -> Ca(OH)2}\\
\ce{Ca(OH)2} ist gelöschter Kalk\\
\ce{Ca(OH)2 + CO2 -> CaCO3 + H2O}\\\\
Gips: \ce{CaSO4$\cdot$2H2O -> CaSO4$\cdot$0.5H2O}\\
$\cdot\ce{0.5H2O}$ nennt man Hemihydrat\\
Anhydrit: \ce{CaSO4} wasserfrei\\\\\\

\textbf{EINSCHUB:} Wasserhärte: Gesamtmenge an zweiwertiger Kationen im Wasser.\\
Temporäre Härte:\\
\ce{Ca^{2+} + 2OH^- + CO2 -> CaCO3 + H2O}\\
\ce{CaCO3 + H2O + CO2 -> Ca^{2+} + 2HCO3^-}\\
Edukte schwerlöslich, Produkte leichtlöslich\\
Enthärtung von \ce{H2O}:\\
\begin{itemize}
    \item Ionenaustausch: Harz mit Sulfonsäurengruppen, belegt mit \ce{Na^+} $\rightarrow$ Austausch gegen \ce{Ca^{2+}}
    \item Komplexbildner: EDTA, Zeolith
    \item Umkehrosmose
    \item Kristallisationskeim
\end{itemize}

\subsubsection*{Grimm-Sommerfeld-Verbindungen}
Kation aus der $N-k$-ten Gruppe + Anion aus der $N+k$-ten Gruppe = Struktur, die einen Element aus der $N$-ten Gruppe des PSE\\\\
Beispiele:\\
\begin{itemize}
    \item[1.] \ce{BN} $\rightarrow$ Struktur von \ce{C} \textcolor{gray}{(Diamant, Graphit)} (14.Gruppe)
    \item[2.] \ce{CdSn} $\rightarrow$ Struktur von \ce{C} \textcolor{gray}{(Diamant)} (14.Gruppe)
    \item[3.] \ce{GeSe} $\rightarrow$ Struktur von \ce{As} \textcolor{gray}{(auch möglich: Struktur von \ce{P} oder \ce{Sb})} (15.Gruppe)
\end{itemize}
Wichtige Vertreter:\\
\ce{CdS},\ce{CdSe} und \ce{CdTe} $\rightarrow$ wichtige Farbpigmente\\
\ce{ZnSe},\ce{CdSe},\ce{CdTe} $\rightarrow$ Halbleitermaterialien\\
\ce{CdS} $\rightarrow$ Fotohalbleiter\\
\ce{ZnS}:\ce{M} $\rightarrow$ Phosphoreszenzmaterial (:\ce{M} heißt dotiert mit \ce{M})

\subsubsection*{Hydroxide}
\begin{center}
    \ce{Be(OH)2 -> Ba(OH)2}\\
    höherer \ce{->[{kovalenter Bindungsanteil}]} niedrigerer\\
    niedrige \ce{->[{Löslichkeit}]} hohe
\end{center}

\begin{center}
    \ce{Zn(OH)2 + 2H^+ -> Zn^{2+} + 2H2O} Säure\\
    \ce{Zn(OH)2 + 2OH^- -> [Zn(OH)4]^{2-}} Base\\
    Cd- und Hg-Hydroxide sind basisch
\end{center}

\subsection{Die Chemie des Quecksilbers}
\subsubsection{Besonderheiten}
\begin{itemize}
    \item relativistische Effekte\\$\hookrightarrow$ keine "$sp^3$-Hybridisierung", maximal sp $\rightarrow$ lineare Koordination
    \item pseudo-Edelgaskonfiguration\\$\hookrightarrow$ schwache Bindungskräfte zwischen den Atomen $\rightarrow$ flüssig bei Zimmertemperatur
    \item Ox-Stufe +1 in Form von \ce{Hg2^{2+}}-Kationen
\end{itemize}

\subsubsection{Halogenide}
\ce{Hg2Cl2,Hg2Br2,Hg2I2} molekular aufgebaut\\
Kalomel-Reaktion:\\
\begin{center}
    \ce{Hg2Cl2 +2NH3 -> Hg + [Hg(NH2)]Cl + Cl^- + NH4^+}\\
    \ce{HgCl2, HgBr2, HgI2} molekular, \ce{HgF2} ionisch\\
    \ce{HgCl2 + 2NH3 -> [Hg(NH3)2]^{2+}+2Cl^-}\\
    zwischen \ce{Hg} und \ce{I} besonders starke Bindung\\
    \ce{HgCl2 + 2 I^- -> HgF2 + 2Cl^-}\\
    \ce{HgI2 + 2I^- -> [HgI4]^{2-}}\\
    \ce{[HgI4]^{2-} + NH4^+ + 4OH^- -> [Hg2N]I + 7I^- + 4H2O}
\end{center}

\subsubsection{Chalkogenide}
\begin{center}
    \ce{Hg2O -> Hg+HgO} Disproportionierung\\
    \ce{HgO -> Hg + 1/2 O2}\\
    \ce{HgO} zeigt Thermochromie (Farbwechsel bei Temperaturerhöhung)\\
    \ce{HgS}:
    \begin{itemize}
        \item Metacinnabarit (\ce{ZnS}-Struktur, schwarz)
        \item Cinnabarit/Zinnoger (\ce{HgS}, rot)
    \end{itemize}
\end{center}

\subsubsection{Amalgame}
Metallverbindungen mit Quecksilberbeteiligung
\begin{itemize}
    \item[1.] Stöchiometrische Amalgame (intermet. Verbindungen)\\z.B. \ce{NaHg2}, \ce{BaHg11}
    \item[2.] Amalgame mit Phasenbreiten (intermet. Verbindung)\\$\mathrm{HgIn}_{1 +- x} \mathrm{Hg}_{2 +- x}\mathrm{Tl}$
    \item[3.] Amalgame mit Löckenlose Mischbarkeit (farbe Lösung)\\$\mathrm{Hg}_x\mathrm{Au}_{1-x}$
\end{itemize}

\section{Die Metalle des p-Blocks}
\subsection{Eigenschaften}
\subsubsection{Tabelle}
\subsubsection{Grnazbereich Metalle-Nichtmetalle}
\begin{itemize}
    \item \ce{Al} $\rightarrow$ ccp
    \item \ce{Ga} $\rightarrow$ spezieller Strukturtyp
    \item \ce{In} $\rightarrow$ verzerrte ccp
    \item \ce{Tl} $\rightarrow$ hcp
    \item \ce{Sn} $\rightarrow$ verzerrte dichteste Kugelpackung
    \item \ce{Pb} $\rightarrow$ verzerrte dichteste Kugelpackung
    \item \ce{Sb} $\rightarrow$ Arsenstruktur
    \item \ce{Bi} $\rightarrow$ Arsenstruktur
\end{itemize}
\subsection{Vorkommen}
\subsubsection{Erdmetalle}
\ce{Al}: 3.häufigstes Element in der Erdkrust\\
$\hookrightarrow$ Al-Oxiden, - Hydroxiden, - Silcaten, -Alumosilicaten\\
\ce{Ga}\ce{In}\ce{Tl}
\begin{itemize}
    \item \ce{Ga} Begleiter von \ce{Al}
    \item \ce{In}\ce{Tl} Begleiter von \ce{Sn}\ce{Pb}
\end{itemize}

\subsubsection{Zinn,Blei,Actino-,Bismut}
\ce{Sn}\ce{Pb}: oxidisch (\ce{Sn}) und sulfidisch (\ce{Pb})\\
\ce{Sb}: \ce{Sb2S3} (Grauspießerglanz)\\
\ce{Bi}: \ce{Bi2S3} aber auch \ce{Bi2O3}

\subsection{Herstellung}
\subsubsection{Erdmetalle}
Aluminiumherstellung: 
\begin{center}
    1. \ce{Al(OH)3 + NaOH -> Na[Al(OH)4]} (löslich)\\
    \ce{Fe,Ti,Si}-Verbindungen unlöslich\\
   2. \ce{Na[Al(OH)4] ->[{H2O}] Al(OH)3$\downarrow$ + NaOH_{(aq)}}\\
    Dieses \ce{Al(OH)3} ist nun rein\\
   3. \ce{2Al(OH)3 ->[Temperatur] Al2O3 + 3H2O}\\
   4. \ce{Al2O3 -> 2Al+ 3/2 O2}\\
    \ce{3C + 3/2 O2 -> 3CO}\\
    \ce{Al2O3 + 3C -> 2Al + 3Co}
\end{center}
Galliumherstellung*: Reichert sich im ersten Schritt der Aluminiumherstellung an.\\
Indium- / Thalliumherstellung*: Aus den Röstgasen bei der Pb-Herstellung\\
* Urban-Mining

\subsubsection{Zinn, Blei, Antimon, Bismut}
Zinn: \ce{SnO2 + 2C -> Sn + 2CO}, Reinigugn über "seigen"
Blei, Antimon, Bismut: Rösten.

\subsection{Verbindungen}
\subsubsection{Halogenide}
$\hookrightarrow$ Trihalogenide z.B. \ce{AlCl3}
$\hookrightarrow$ Auch fpr Gallium, Indium\\
ABER $\rightarrow$ Thallium am liebsten einwertig: \ce{TlX}\\
Quizfrage: \ce{TlI3} stabil? Nö, reagiert zu \ce{TlI} $\cdot$ \ce{I2}\\
Zinn + Blei: \ce{SnX4} und \ce{PbX4} sind leicht flüchtige und moderat hydrolyseempfindliche Moleküle, aber nur für \ce{X} = Cl,Br,I\\
\ce{SnF4} und \ce{PbF4} siehe Folie\\
für \ce{Pb -> PbX2}\\
\ce{PbI2 + 2I^- -> [Pb(I)_4]^{2-}}\\
\ce{Sb},\ce{Bi}:  \ce{SbX3 + X^- -> [SbX4]^-} und \ce{BiX3 + X^- -> [BiX4]^-}:\\
Bei den Zinnverbindungen sind lonepairs vorhanden, es gibt Lonepairaktivität, Stereochemisch aktiv.\\
Bismutverbindungen sind über Kanten Verknüpft und Pentagonale Dipyramiden, das lonepair ist nicht visualisierbar, nicht stereochemisch aktiv\\

\subsubsection{Chalkogenide}
Aluminium: \ce{Al2O3}\\
Korun, sehr stabil. Passivierung von metallischem Aluminium.\\
\ce{Al2O3}:\ce{Cr^{3+}} $\rightarrow$ Rubin\\
\ce{Al2O3}:\ce{Fe^{2+}}\ce{Ti^{3+}} $\rightarrow$ Saphir\\
$\hookrightarrow$ \ce{Ga2O3}; \ce{In2O3}; aber \ce{Tl2O}\\\\
Zinn und Blei:\\
\ce{SnO2} , \ce{PbO2} , \ce{SnO} , \ce{PbO}\\
\ce{PbO2 -> Pb12O19 -> Pb12O17 -> Pb3O4 -> PbO}
Antimon und Bismut:
\ce{Sb2O3}, \ce{Bi2O3}\\
\ce{Sb2S3}, \ce{Bi2S3}\\
$\downarrow$\\
\ce{[SbS3]^{3-}}

\subsubsection{Aquakomplexe von Aluminium}
\ce{[Al(H2O)6]^{3+} -> [Al(H2O)5OH]^{2+} + H^{+}_{(aq)}}
\ce{Al^{3+}} ist klein hart und hoch geladen, somit schafft es die Elektronenhülle von Sauerstoff leicht zu polarisieren.\\
Dadurch entsteht eine kovalente Bindung zwischen einem Wassermolekül und dem \ce{Al^{3+}}, wodurch ein \ce{H^{+}} abgespalten werden muss!\\
Danach nimmt der Effekt ab.\\
\ce{Al(OH)3 + H^{+} -> Al^{3+}_{(aq)} + 3H2O}\\
\ce{Al(OH)3 + OH^{-} -> [Al(OH)4]^{-}}\\
Amphoteres Verhalten.

\subsubsection{Zintl-Phasen}
Anionen ab der 13. Gruppe sind isoelektronisch zu Elementen derselben Elektronenzahl.\\
\ce{Te^-} $\rightarrow$ \ce{Te2^{2-}} isoelektronisch zu \ce{I2}\\
\ce{Si^-} $\rightarrow$ \ce{Si4^{4-}} isoelektronisch zu \ce{P4}\\
\ce{Tl^-} $\rightarrow$ \ce{Tl4^{4-}} isoelektronisch zu \ce{C4}\\

\section{Grundzüge der Komplexchemie}
\subsection{Allgemeines}
Besteht aus einem Zentralatom, um welches einige Liganden liegen.\\
Zentralatom meist metallisch, die Liganden sind meist nichtmetallisch oder besitzen einen nichtmetallischen Anteil.\\
Komplexbildung $\rightarrow$ Lewis-Säure (Zentralatom) - Base (Liganden) - Reaktion.\\
Die Liganden müssen freie Elektronenpaare mitbringen, das Zentralatom freie Orbitale, so viele, damit ees für die Liganden reicht.\\
\begin{equation*}
    \mathrm{Freie Orbitale} = \mathrm{Ligandenanzahl}
\end{equation*}
Freie Orbitale müssen Valenzorbitale sein (äußersten)\\
Übergangsmetallkationen:
\begin{itemize}
    \item Valentorbitale: $n$ s (1 s-Orbital) ($n = 2,5,6,7$), und $(n-1)$ d Orbitale (5 d-Orbitale), und $n$ p (3 p-Orbitale) $\rightarrow$ 9 Orbitale mit 18 Elektronen $\Rightarrow$ Edelgasschale
    \item homoleptische Komplexe: Selbe Art von Liganden
    \item heteroletische Komplexe: Unterschiedliche Liganden
\end{itemize}

\subsection{Komplexliganden}
Unterschieden durch Anzahl an Koordinationsstellen:
\begin{itemize}
    \item Einzähnige Liganden - eine Koordinationsstellen
    \item Mehrzähnige Liganden - mehrere Koordinationsstellen zum gleichen Zentralatom (Monoatomare Liganden fallen hier aus)\\ Chelatliganden! Bsp: EDTA - EthylenDiamminTetraAcetat
\end{itemize}
\begin{center}
    \begin{tabular}{|c c c|}
        \hline
        \ce{CO3^{2-}} & vs. & \ce{C2O4^{2-}}\\
        \hline
        & \makecell{An den Zwei Sauerstoffen mit\\3 freien Elektronenpaaren\\an das gleiche Zentralatom} &\\
        \hline
        & 90 ° Winkel am Zentralatom & \\
        \hline
        & 120 ° am Molekül am C-Atom & \\
        \hline
        & \makecell{105 ° Winkel am Sauerstoff\\ zwischen den zwei Bindungen\\ (zum Zentralatom und C)} & \\
        \hline
        \makecell{$\sum (\mathrm{Winkel})$\\sollte 360° sein\\ist aber 420} & & \makecell{$\sum (\mathrm{Winkel})$\\sollte 540 ° sein\\ist es auch.}\\
        \hline
    \end{tabular}
\end{center}

\subsection{Arten der Donor-Bindung und Ligandenverbrückung}
\subsubsection{Arten der Donor-Bindung}
$n$-Komplex : \chemfig{Z-L}\\
$\pi$-Komplex : \chemfig{Z-}BENZOL\\
$\sigma$-Komplex : Sigmabindung wird zum Zentralatom hinverlagert, schwächt die Sigmabindung, nur bei Elektronenarmen Bindungen.

\subsubsection{Haptizität ($\pi$-Komplexe) $\eta$}
\textbf{Definition: Wie viele Orbitale des Liganden-$\pi$-Sytsems tragen zur Koordination zum Zentralatom bei}

\subsubsection{Verbrückung $\mu$}
\textbf{Definition: Wie viele Zentralatome kann der Ligand miteinander Verbinden, bzw. zu ie vielen verschiedenen Zentralatomen kann er koordinieren}
\\\\\\
\begin{center}
    \textbf{WICHTIG: NOMENKLATUR VON LIGANDEN LERNEN}
\end{center}
\subsection{Geometrien in Komplexen}
\begin{center}
    \textbf{SIEHE FOLIE}
\end{center}
Ab der Koordinationszahl CN=5 gibt es Äquatorial- und Axial-Liganden

\subsection{Isomerie in Komplexen}
\subsubsection{Geometrische Isomere}
\begin{center}
    \textbf{SIEHE FOLIE}
\end{center}

\subsubsection{Konstitutionsisomere}
\begin{itemize}
    \item Ionisationsisomere:
\end{itemize}
\begin{center}
    \begin{tabular}{|c c c|}
        \hline
        \ce{[Pt(NH3)4Br2]Cl2} & vs. & \ce{C2O4^{2-}}\\
        \hline
        & \makecell{An den Zwei Sauerstoffen mit\\3 freien Elektronenpaaren\\an das gleiche Zentralatom} &\\
        \hline
        & 90 ° Winkel am Zentralatom & \\
        \hline
        & 120 ° am Molekül am C-Atom & \\
        \hline
        & \makecell{105 ° Winkel am Sauerstoff\\ zwischen den zwei Bindungen\\ (zum Zentralatom und C)} & \\
        \hline
        \makecell{$\sum (\mathrm{Winkel})$\\sollte 360° sein\\ist aber 420} & & \makecell{$\sum (\mathrm{Winkel})$\\sollte 540 ° sein\\ist es auch.}\\
        \hline
    \end{tabular}
\end{center}

\subsection{Die Kristallfeld- bzw. Ligandenfeldtheorie}
\textbf{Kristallfeldtheorie}
Elektrostatische Wechselwirkungen zwischen dem Ligand (negativ geladen) und dem Zentralatom (positiv geladen)
\textbf{Ligandenfeldtheorie}
Kristallfeldtheorie + die Erklärung der Bindung mit der Molekülorbitaltheorie (MO-Theorie)
\subsubsection{Allgemeines}
Energetische Aufspaltung der d-Orbitale im Feld der Liganden $\rightarrow$ elektrostatische Gründe\\
Tetraeder: immer Lowspin, da Ligandenfeldaufspaltungsenergie geringer als Spinpaarungsenergie.\\
Oktader: Lowspin oder Highspin, je nach Verhältnis von Ligandenfeldaufspaltungsenergie zu Spinpaarungsenergie.\\
Bei einem quadratischen planaren Feld immer Lowspin, da Ligandenfeldaufspaltungsenergie doppelt so groß wie beim Oktaeder, limitiert auf d$^8$, machmal auch d$^9$, hier aber verzerrt\\
\subsubsection{Die Ligandenfeldstabilisierungsenergie (LFSE)}
\textbf{= Ligandenfeldaufspaltungsenergie}
Beispiel: \ce{MgAl2O4} - Spinell\\
Kubisch-dichteste-Kugelpackung aus \ce{O^{2-}}\\
Dreiwertiges Ion in der Hälfte der Oktaederlücke\\
Zweiwertiges Ion in $\frac{1}{8}$ der Tetraederlücken\\
\ce{Fe3O4} $\rightarrow$ \ce{Fe^{II}Fe^{III}2O4}\\
Tetraeder:\\
3 Orbitale (3 e) werden um 4 Dq angehoben, 2 (3 e) Orbitale werden um -6 Dq abgesenkt:\\
$3\cdot -6 + 3 \cdot 4 = -6 \vert \cdot \frac{4}{9}$ da Tetraeder\\
$\hat{=} -\frac{24}{9} = 2.\bar{6}$\\
Oktaeder:\\
2 (2 e) Anheben um 6 Dq, 3 (3 e) Absenken um -4 Dq:\\
$3\cdot -4 + 2\cdot 6 = 0$\\
Gesamterenergiegewinn: -$2.\bar{6}$\\\\
Inverser Spinell:\\
Zweiwertige Ionen besetzen $\frac{1}{4}$ der Oktaderlücken\\
Dreiwertige Ionen besetzen $\frac{1}{4}$ der Oktaderlücken\\
Dreiwertige Ionen besetzen $\frac{1}{8}$ der Tetraederlücken\\\\
Beim Oktaeder:\\
3 Orbitale (4 e) abgesenkt um -4 Dq, 2 Orbitale (2 e) um 6 Dq angehoben:\\
$= -4$\\
Tetraeder:\\
3 Orbitale (3 e) anheben um 4 Dq, 2 (2 e) absenken um -6 Dq\\
$= 0$\\
Gesamtgewinn: -4 Dq\\\\
Somit eher im inversen Spinell.\\
Oxidationsstufen werden \underline{NICHT} verändert.

\subsubsection{Die spektrochemischen Reihen}
1. Größe des Zentralatoms\\
\begin{center}
    \begin{tabular}{c c c}
        LFSE groß & 5d $>$ 4d $>$ 3d & LFSE klein\\
        & bei 4d \& 5d kein Highspin &\\
    \end{tabular}
\end{center}
2. Oxidationsstufe des Zentralatoms\\
\begin{center}
    \begin{tabular}{c c c}
        LFSE groß & +5 $>$ +4 $>$ +3 $>$ +2 $>$ +1 & LFSE klein\\
    \end{tabular}
\end{center}
3. Ligandenstärke\\
\ce{I^{-}} $<$ \ce{Br^{-}} $<$ \ce{Cl^{-}} $<$ \ce{F^{-}} $<$ \ce{O^{2-}} $<$ \ce{OH^{-}} $<$ \ce{H2O} $<$ \ce{NH3} $<<$ \ce{CN^{-}} $<$ \ce{CO}\\
Bis \ce{O^{2-}} schwach\\

\subsection{Physikalische Eigenschaften}
\subsubsection{Optische Eigenschaften}
1. Wellenlänge $\hat{=}$ Energie\\
Bei Komplexen: LFSE\\
2. Intensität (wie oft findet der Prozess statt?)\\
d - d Übergänge sind verboten (quantenmechanisch)\\
Übergang mit Spinumkehr sind verboten (quantenmechanisch)\\
Laporte-Verbot $\rightarrow$ Übergänge unter Inversionssymmetie sind verboten.\\
\subsubsection{Magnetismus in Komplexen}
Alle Elektronen gepaart $\rightarrow$ Diamagnetismus, Lowspin gerade Anzahl an Elektronen\\
Ungepaarte Elektronen $\rightarrow$ Paramagnetismus\\
$\hookrightarrow$ Je mehr ungepaarte Elektronen, desto höher das magnetische Moment*\\
* - Gilt nur in der ersten Übergangsmetallreihe, da die sogenannte "Spin-Bahn-Kopplung" vernachlässigt wird.\\

\section{Übergangsmetalle}
\subsection{Allgemeines}
\subsubsection{Verschiedene Trends im Vergleich Hauptgruppenmetalle / Übergangsmetalle}
\textbf{Siehe Folie.}
\subsection{Die vierte Gruppe}
\subsubsection{Vorkommen}
Titan: $\rightarrow$ oxidisch: \ce{TiO2}, Perwoskit-Strukturtyp: \ce{CaTiO3}\\
Zirkonium und Hafnium: oxiditsch $\rightarrow$ \ce{ZnO2} / \ce{HfO2}, \ce{ZrSiO4}
\subsubsection{Herstellung}
Carbochlorierung:\\
\ce{TiO2 + 2C + 2Cl2 -> TiCl4 + 2CO}\\
Kroll-Prozess:\\
\ce{TiCl4 + 2Mg -> 2MgCl2 + Ti} (SChwamm)\\
van Arlid-de Boer\\
\ce{Ti + 2I2 ->[{873.15 K}] TiI4}\\
Chemische Transport\\
\textbf{Siehe Folie}\\
\subsubsection{Verbindungen}
Halogene:\\
\ce{TiX4 ->[{+H2O}] Ti_{(aq)}^{4+}}\\
\ce{[Ti(H2O)6]^{4+} - 2H^{+} -> TiO^{2+}_{(aq)}}\\
\ce{TiO^{2+} + O2^{2-} -> [TiO(O2)]_{(aq)}}\\
\ce{Ti^{3+} + 3X^{-} -> TiX3}\\
$\rightarrow$ fest, \ce{BiI3}-Struktur.\\
Chalkogenide:\\
\ce{TiO2} $\rightarrow$ Rutil, Anatas, Brookit\\
\ce{Ti3O5}; \ce{TiO_{2-x}}\\
\ce{Zn + HCl -> Zn^{2+} + HCl^{-} + 2H^{-}}\\
\ce{Ti^{4+}_{(aq)} + H -> Ti^{3+}_{(aq)} + H^{+}}\\
Percowsit \ce{CaTiO3}\\
\textbf{Siehe Folie}
\subsection{Die fünfte Gruppe}
\subsubsection{Vorkommen}
Vanadium:\\
Vanadit: \ce{Pb5Cl[VO4]3}\\
Patronit: \ce{VS4}\\
Niob + Tantal: \ce{(Fe,Mn)[NbO3]} bzw. \ce{(Fe,Mn)TaO3}\\
\subsubsection{Herstellung}
Vanadium: \ce{V2O5 + 5Ca -> 2V + 5CaO} metallothermisch\\
Niob / Tantal: \ce{MsO5 + 5C -> 2M + 5CO}\\
\subsubsection{Verbindungen}
Halogenide: \ce{VX5}; \ce{NbX5};\ce{TaX5c}\\
verknüpfe Aktaeder in der Krsitallstruktur\\
niedrigere Oxidationsstufen bei \ce{Nb} und \ce{Ta}\\
\ce{Nb6X8} oder \ce{Nb6X12}\\
Linkes ist ein Cluster, \textbf{siehe Folie}\\
Oxide:\\
\ce{V2O5} analog zu \ce{P2O5} \ce{->[{H2O}] H3VO4}\\
\ce{VO2} Rutil-Typ\\
\ce{V2O3} Korund-Struktur\\
\ce{VO} \ce{NaCl}-Typ\\
Nach unten hin immer dunkler richtung schwarz\\
\ce{2VO4^{3-} ->[{2 Protonen}] V2O3^{4-} -> V3O10^{5-} ... V2O5}
Bei den Punkten handelt es sich um isopolysäuren.
\subsection{Die sechste Gruppe}
\subsubsection{Vorkommen}
Chrom:\\
+ III: Chromeisenstein \ce{FeCr2O4}, Chromocker \ce{Cr2O3}\\
+VI: Krokoit \ce{Pb[CrO4]}\\
Molybdän \& Wolfram:\\
+ VI: Powellit \ce{Ca[MoO4]}, Scheelit \ce{Ca[WO4]}, Wulferit \ce{Pb[MoO4]}, Wolfrenit \ce{(Fe,Mn)WO4}\\
+ IV: Molybdänglanz \ce{MoS2}

\subsubsection{Herstellung}
Chrom:\\
\ce{4FeCr2O4 + 8Na2CO3 + 7 O2 -> 8Na2CrO4 + 2Fe2O3 + 8CO2}\\
\ce{Na2CrO4 + H2SO4 -> Na2Cr2O7 + Na2SO4}\\
\ce{Na2Cr2O7 + 2C -> Cr2O3 + Na2CO3 + CO}\\
\ce{Cr2O3 + 2Al -> Al2O3 + 2Cr} (Rohchrom)\\
van Arhd-de Boer\\
\ce{Cr + I2 <->[{1173 K}][{1273-1573 K}] CrI2}\\
Molybdän/Wolfram:\\
nur Mo: \ce{2NoS2 + 7O2 -> 4SO2 + 2MoO3}\\
Mo/W:\\
\ce{CaWO4 + Na2CO3 -> CaO + Na2WO4 + CO2}\\
\ce{Na2WO4 + 2HCl -> WO3 \cdot x H2O + 2NaCl}\\
Reduktion:\\
\ce{Wo3/MoO3 + 3H2 -> W/Mo + 3H2O}

\subsubsection{Verbindungen}
Halogene:\\
\textbf{Siehe Folie}\\
Sauerstoffverbindungen des Chroms:\\
\ce{CrO} mit \ce{Cr^{2+}} schwarz, NaCl-Typ $\rightarrow$ halbleitend\\
\ce{Cr2O3} mit \ce{Cr^{3+}} grün, Korund-Typ\\
\ce{CrO2} mit \ce{Cr^{4+}} schwarz, Rutil-Typ, ferromagnetisch\\
\ce{Cr2O5} mit \ce{Cr^{5+}} molekular\\
\ce{CrO3} mit \ce{Cr^{6+}} molekular (siehe \ce{SO3})\\
$\downarrow$\\
Anhydrid der Schromsäure\\
\ce{H2CrO4 - 2H^{+} -> 2CrO4^{2-} + 2H^{+} -> Cr2O7^{2-} + H2O}\\
Umsetzung von \ce{Cr2O2^{2-}} mit \ce{H2O2 -> CrO5 } = \ce{Cr(O2)2O}\\
Stoffverbindungen von Mo und W:\\
\begin{center}
    \begin{tabular}{c c c}
        \hline
        \ce{MoO3}& und & \ce{WO3}\\
        \ce{H2MoO4}& (waren Anhydride werden zu diesen) & \ce{H2WO4}\\
        \ce{MoO4^{2-}}& und & \ce{WO4^{2-}}\\
        & Kondensation zu z.B. Heteropolysäuren & \\
        z.B. \ce{P [Mo12O40]}& & \\
        \hline 
    \end{tabular}
\end{center}
\ce{MoO3 / WO3} $\rightarrow$ Reduktion mit Wasserstoff $\rightarrow$ Wolframbraun\\
$\rightarrow$ Reduktion mit Alkalimetallen $\rightarrow$ Wolframbronze\\
Sulfide: \ce{MoS2 ->} \textbf{Siehe Folie}

\subsubsection{Struktur von Molybdänblau}
\textbf{Siehe Folie}

\subsubsection{Komplexe mit M-M.Vierfach- und Fünffachbindung}
$\sigma$-Bindung: \textbf{siehe Folie}\\
$\pi$-Bindung: \textbf{siehe Folie}\\
$\delta$-Bindung: \textbf{siehe Folie}

\subsubsection{Die siebte Gruppe}
\subsubsection{Vorkommen}
Mangan: Oxide: \ce{MnO2, Mn2O3, Mn3O4}\\
Technetium: vom Kernbrennstab\\
Rhenium: vergesellschaftet mit \ce{MoS2}
\subsubsection{Herstellung}
Mangan: Aluminothermisch aus \ce{Mn3O4}\\
Technetium: \ce{^{89}Mo + ^1_0 n -> ^{99}Mo - $\beta$^- -> ^{99*}Tc - $\gamma$ -> ^{99}Tc}\\
Rhenium: Nebenprodukt in den Röstgasen der Molybdänherstellung aus \ce{MoS2} $\rightarrow$ \ce{Re2O7} $\rightarrow$ Reduktion mit \ce{H2}

\subsubsection{Verbindungen}
Halogene:\\
Mn - Halogenide nur in den niedrigen Oxidationsstufen des Mangans.\\
Tc- und Re- Halogenide auch für höhere Oxidationsstufen (\ce{TcF6} oder \ce{TcCl4}); bei Re: Clusterbildung\\\\
Sauerstoffverbindung:\\
\ce{Mn^{2+}}: \ce{Mn(OH)2 - H2O -> MnO}\\
\ce{Mn^{3+}}: \ce{Mn2O3}\\
\ce{Mn^{4+}}: \ce{MnO2} oder \ce{MnO(OH)2}\\
\ce{Mn^{5+}}: \ce{MnO4^{3-}} nur im stark alkalischen, hellblau\\
\ce{Mn^{6+}}: \ce{MnO4^{2-}} lakalisch, dunkelgrün\\
\ce{Mn^{7+}}: \ce{MnO4^-} violett\\
\vspace*{2cm} \ce{MnO4^- + H^+ -> HMnO4}\\
\vspace*{3cm} \ce{2HMnO4 - H2O -> Mn2O3}\\\\
Technetium und Rhenium:\\
hohe Ox-Stufen \ce{Tc2O7, Re2O7,ReO3}\\

\subsubsection{Technische Verwendung}
Léclanché-Element $\rightarrow$ \textbf{siehe Folie}\\
Zinkbecher: \ce{Zn+ 4NH4^+ -> [Zn(NH3)4]^{2+} + 2e^- + 4H^+}\\
oder \ce{Zn+ 4OH^- -> [Zn(OH)4]^{2-} + 2e^- + 4H^+}\\
Braunsteinpulver: \ce{MnO2 + H2O + e^- -> MnO(OH) + OH^-}\\
Ergibt ca. $1.5\,\mathrm{V}$

\subsection{Die achte, neunte und zehnte Gruppe, Eisen Cobalt und Nickel}
\subsubsection{Vorkommen}
Eisen: \ce{Fe2O3} (Hämatit); \ce{FeO(OH)} (Goethit); \ce{Fe3O4} (Magnetit); \ce{FeS2} (Pyrit/ Markasit)\\
Cobalt und Nickel: \ce{CoAsS, CoAs3, NiAs, (Ni/Fe)9S8, NiS}\\
\ce{NiAs} ist hexagonal Analog zur \ce{NaCl}-Struktur\\

\subsubsection{Herstellung}
Hochofenprozess von Eisen und Stahl von \ce{Fe2O3} zu Roheisen $\rightarrow$ \textbf{siehe Folie}\\
Roheisen enthält bis zu 4 \% C\\
Aufarbeiten mit "Schrott" $\rightarrow$ Rost \ce{Fe2O3}\\
Eisen veredler mit \ce{Cr, Mo, V, ...}\\
Cobalt/Nickel: Rösten\\
Reinigung von Nickel $\rightarrow$ Mond-Verfahren\\
\ce{Ni + 4CO ->[{353.15 K}] [Ni(CO)4] ->[{433.15 K}] Ni + 4CO}
\end{document}